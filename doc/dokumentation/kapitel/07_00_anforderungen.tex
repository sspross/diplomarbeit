Bei der Aufnahme der Bedürfnisse der verschiedenen Stakeholdern, zusammen mit 
der Geschäftsleitung der allink, wird darauf geachtet, dass sie so einfach und 
eindeutig wie möglich formuliert werden. Zusätzlich wird jedes Bedürfnis beschrieben
und fortlaufend nummeriert, damit in den Anforderungen wieder darauf verwiesen
werden kann.

\section{Bedürfnisse der Stakeholder}
\newcounter{bcounter}

\subsection{Kunde}
Die in der Tabelle \ref{tab:beduerfnisse_stakeholder_kunde} aufgelisteten 
Bedürfnisse beziehen sich auf die Sicht des Kunden, wurden aber von der
Geschäftsleitung der allink formuliert.

\begin{center}
    \begin{longtable}{lp{3cm}p{10cm}}
        \toprule \textbf{Nr.} & \textbf{Bedürfnis} & \textbf{Beschreibung} \\
        \midrule \addtocounter{bcounter}{1}B\arabic{bcounter} & Klare Timings & 
            Der Kunde möchte klare Timings haben, die von allink auch eingehalten 
            werden können.\\
        \midrule \addtocounter{bcounter}{1}B\arabic{bcounter} & Gute Beratung & 
            Der Kunde möchte geführt werden. Die Beratung muss dem Kunden das 
            Gefühl vermitteln, dass er genau das bekommt, was er benötigt.
            Dabei können auch kleine Zwischenmenschliche Tipps oder Gefälligkeiten
            sehr förderlich sein.\\
        \midrule \addtocounter{bcounter}{1}B\arabic{bcounter} & Gute Qualität & 
            Qualität ist dem Kunden sehr wichtig. Fehler sollte er nicht selbst 
            entdecken.\\
        \midrule \addtocounter{bcounter}{1}B\arabic{bcounter} & Vertraulichkeit & 
            Je nach Daten die in einem Projekt verwendet werden müssen, muss
            auch eine gewisse Vertraulichkeit an den Tag gelegt werden, so dass
            der Kunden darauf Vertrauen kann, dass die Daten bei der Agentur 
            sicher aufbewahrt und für keine anderen Dinge verwendet werden.\\
        \midrule \addtocounter{bcounter}{1}B\arabic{bcounter} & Prestige der Agentur & 
            Nicht zu vernachlässigen ist auch der Bekanntheitsgrad von allink
            selbst. Viele Kunden arbeiten gerne mit einer in der Szene bekannteren
            Agentur zusammen.\\
        \bottomrule
        \caption[Bedürfnisse an das neue Projektmanagement seitens des Kunden]{Bedürfnisse 
            an das neue Projektmanagement seitens des Kunden\footnotemark}
        \label{tab:beduerfnisse_stakeholder_kunde}
    \end{longtable}
\end{center}
\footnotetext{Eigene Darstellung}

\subsection{Geschäftsleitung}
Die in der Tabelle \ref{tab:beduerfnisse_stakeholder_partner} aufgelisteten 
Bedürfnisse beziehen sich auf die Sicht der Geschäftsleitung.

\begin{center}
    \begin{longtable}{lp{3cm}p{10cm}}
        \toprule \textbf{Nr.} & \textbf{Bedürfnis} & \textbf{Beschreibung} \\
        \midrule \addtocounter{bcounter}{1}B\arabic{bcounter} & Überblick Projektverlauf & 
            Die Geschäftsleitung wünscht sich einen aussagekräftigen Überblick 
            über die jeweiligen Projekte und deren Verläufe.\\
        \midrule \addtocounter{bcounter}{1}B\arabic{bcounter} & Finanzieller Überblick & 
            Auch wünscht sich die Geschäftsleitung einen finanziellen Überblick 
            über alle Projekte und die Liquidität der Unternehmung.\\
        \midrule \addtocounter{bcounter}{1}B\arabic{bcounter} & Entlastung von Wiederkehrendem & 
            Bei wiederkehrenden Projekten und Abläufe soll die Geschäftsleitung
            soweit wie möglich entlastet werden, da die wichtigsten Informationen
            schon existieren.\\
        \midrule \addtocounter{bcounter}{1}B\arabic{bcounter} & Verantwortungs- träger & 
            Die Geschäftsleitung möchte mehr Verantwortung für weniger kritische
            Projekte vollständig an Berater oder Projektleiter abgeben können,
            die die Kompetenz haben, eigene Entscheidungen darin zu fällen.\\
        \bottomrule
        \caption[Bedürfnisse an das neue Projektmanagement seitens der Geschäftsleitung]{Bedürfnisse 
            an das neue Projektmanagement seitens der Geschäftsleitung\footnotemark}
        \label{tab:beduerfnisse_stakeholder_partner}
    \end{longtable}
\end{center}
\footnotetext{Eigene Darstellung}

\subsection{Mitarbeiter}
Die in der Tabelle \ref{tab:beduerfnisse_stakeholder_mitarbeiter} aufgelisteten 
Bedürfnisse beziehen sich auf die Sicht des Mitarbeiters von allink, wurden aber 
von der Geschäftsleitung formuliert.

\begin{center}
    \begin{longtable}{lp{3cm}p{10cm}}
        \toprule \textbf{Nr.} & \textbf{Bedürfnis} & \textbf{Beschreibung} \\
        \midrule \addtocounter{bcounter}{1}B\arabic{bcounter} & Klare Briefings & 
            Der Mitarbeiter wünscht sich klare Briefings, woraus hervorgeht, was
            das Ziel des Projektes und die Aufgabe des einzelnen Mitarbeiters ist.\\
        \midrule \addtocounter{bcounter}{1}B\arabic{bcounter} & Geregelte Arbeitszeiten  & 
            Ausserordentliche Einsätze zu Randzeiten sollten durch eine bessere
            Planung möglichst vermieden werden. Die Mitarbeiter sind sich zwar
            bewusst, dass dies vorkommen kann und auch bereit ihren Teil beizutragen.
            Trotzdem sollte es auf ein Minimum reduziert werden.\\
        \midrule \addtocounter{bcounter}{1}B\arabic{bcounter} & Definierte Ablagestruktur & 
            Zum Wohle aller wünscht sich der Mitarbeiter eine klarere Ablagestruktur
            von den aktuellen und archivierten Projekten. Dies kann viele Abläufe
            vereinfachen und unnötige Kommunikation vermeiden.\\
        \midrule \addtocounter{bcounter}{1}B\arabic{bcounter} & Klare Ansprechpartner & 
            Innerhalb eines Projektes, wie auch im Unternehmen, wünscht sich der
            Mitarbeiter klare Ansprechpartner.\\
        \midrule \addtocounter{bcounter}{1}B\arabic{bcounter} & Minimum an administrativen Arbeiten & 
            Damit sich der Mitarbeiter mit seinen Talenten möglichst gut auf 
            seine Aufgaben konzentrieren kann, sollten administrative Arbeiten
            wie das Rapportieren der Arbeitszeiten auf ein Minimum reduziert
            werden.\\
        \bottomrule
        \caption[Bedürfnisse an das neue Projektmanagement seitens der Mitarbeiter]{Bedürfnisse 
            an das neue Projektmanagement seitens der Mitarbeiter\footnotemark}
        \label{tab:beduerfnisse_stakeholder_mitarbeiter}
    \end{longtable}
\end{center}
\footnotetext{Eigene Darstellung}

\subsection{Drittanbieter}
Die in der Tabelle \ref{tab:beduerfnisse_stakeholder_drittanbieter} aufgelisteten 
Bedürfnisse beziehen sich auf die Sicht eines Drittanbieters, wie zum Beispiel
einer Druckerei. Sie wurden ebenfalls von der Geschäftsleitung der allink 
formuliert.

\begin{center}
    \begin{longtable}{lp{3cm}p{10cm}}
        \toprule \textbf{Nr.} & \textbf{Bedürfnis} & \textbf{Beschreibung} \\
        \midrule \addtocounter{bcounter}{1}B\arabic{bcounter} & Klare Aufträge & 
            Die Aufträge an den Drittanbieter sollten sauber definiert sein und
            alle nötigen Unterlagen zur Ausführung beinhalten.\\
        \midrule \addtocounter{bcounter}{1}B\arabic{bcounter} & Klare Ansprechpartner & 
            Wie auch der Mitarbeiter wünscht sich ein Drittanbieter für einen 
            Auftrag einen klaren Verantwortlichen innerhalb von allink.\\
        \midrule \addtocounter{bcounter}{1}B\arabic{bcounter} & Rechnungen bezahlen & 
            Die Rechnungen sollten rechtzeitig bezahlt werden. Wo möglich sollte
            man auch vorhanden Skonto\footnote{Als Skonto bezeichnet man einen 
            Preisnachlass auf einen Rechnungsbetrag bei einer frühen Bezahlung.}
            Optionen nutzen, wovon beide Parteien profitieren.\\
        \bottomrule
        \caption[Bedürfnisse an das neue Projektmanagement seitens der Drittanbieter]{Bedürfnisse 
            an das neue Projektmanagement seitens der Drittanbieter\footnotemark}
        \label{tab:beduerfnisse_stakeholder_drittanbieter}
    \end{longtable}
\end{center}
\footnotetext{Eigene Darstellung}

\clearpage

\section{Anforderungen}
\newcounter{acounter}
Die aufgenommenen und beschriebenen Bedürfnisse werden nun in der Formulierung
der Anforderungen berücksichtigt und in der Tabelle \ref{tab:anforderungen_projektmanagement} 
dargestellt. Es liegt in der Natur der Sache, dass Bedürfnisse von mehreren 
Anforderungen abgedeckt und nicht alle Bedürfnisse eins zu eins berücksichtigt 
werden können. Sie werden jedoch bei der Ausarbeitung des Konzeptes im Kapitel \ref{chap:konzept}
erneut beachtet und zum Ende kontrolliert, welche Bedürfnisse abgedeckt wurden.

Jede Anforderung wird beschrieben und die darin abgedeckten Bedürfnisse aufgelistet. 
Die Anforderungen werden zudem priorisiert, indem sie in Muss- und Kann-Anforderungen 
kategorisiert werden.\footnote{\citealp*[Vgl.][S. 32]{hobel2006gabler}}

\begin{center}
    \begin{longtable}{lp{3cm}p{6cm}p{3cm}l}
        \toprule \textbf{Nr.} & \textbf{Anforderung} & \textbf{Beschreibung} & \textbf{Bedürfnisse} & \textbf{Prio} \\
        \midrule \addtocounter{acounter}{1}A\arabic{acounter} & ... & ... & B1 bis B4 & Muss \\
        \midrule \addtocounter{acounter}{1}A\arabic{acounter} & ... & ... & B1 bis B4 & Muss \\
        \bottomrule
        \caption[Anforderungen an das neue Projektmanagement]{Anforderungen an das 
            neue Projektmanagement\footnotemark}
        \label{tab:anforderungen_projektmanagement}
    \end{longtable}
\end{center}
\footnotetext{Eigene Darstellung}

\clearpage

\section{Kennzahlen}
\newcounter{kcounter}

\subsection{Projekt}
Die in der Tabelle \ref{tab:proj_kennzahlen_anforderungen_projektmanagement} abgebildeten
Projekt-Kennzahlen basieren einerseits auf den Anforderungen und auf zusätzlichen
Diskussionen mit der Geschäftsleitung. Die Kennzahlen sollen in Zukunft auf
Projektebene gemessen werden können.

\begin{center}
    \begin{longtable}{lp{2cm}p{3cm}p{8cm}}
        \toprule \textbf{Nr.} & \textbf{Stufe} & \textbf{Kennzahl} & \textbf{Beschreibung} \\
        \midrule \addtocounter{kcounter}{1}K\arabic{kcounter} & Projekt & Bruttokosten &
            Die Bruttokosten eines Projektes setzen sich aus allen dem Kunden 
            gegenüber offerierten Beträgen zusammen.\\
        \midrule \addtocounter{kcounter}{1}K\arabic{kcounter} & Projekt & Nettobudget &
            Das Nettobudget eines Projektes wird berechnet, indem man die Bruttokosten
            abzüglich den externen Kosten von Drittanbietern rechnet.\\
        \midrule \addtocounter{kcounter}{1}K\arabic{kcounter} & Projekt & Zielstundensatz &
            Der Zielstundensatz ist der durchschnittlich zu erreichende Stundensatz,
            den man in diesem Projekt erreichen will.\\
        \midrule \addtocounter{kcounter}{1}K\arabic{kcounter} & Projekt & Stundenmaximum &
            Das Stundenmaximum eines Projektes wird berechnet, indem man das Nettobudget
            durch den Zielstundensatz teilt.\\
        \midrule \addtocounter{kcounter}{1}K\arabic{kcounter} & Projekt & Total Stunden &
            Die Total Stunden setzten sich aus allen rapportierten Stunden
            von allen Mitarbeitern, die am Projekt mitgearbeitet haben, zusammen.\\
        \midrule \addtocounter{kcounter}{1}K\arabic{kcounter} & Projekt & Eigentlicher Stundensatz &
            Der eigentliche Stundensatz ist der durchschnittlich erreichte Stundensatz im Projekt
            und berechnet sich, indem man das Nettobudget durch die tatsächlich
            rapportierten Stunden teilt.\\
        \bottomrule
        \caption[Projekt-Kennzahlen-Anforderungen an das neue Projektmanagement]{Projekt-Kennzahlen-Anforderungen 
            an das neue Projektmanagement\footnotemark}
        \label{tab:proj_kennzahlen_anforderungen_projektmanagement}
    \end{longtable}
\end{center}
\footnotetext{Eigene Darstellung}

\subsection{Liquiditätsplanung}
Die in der Tabelle \ref{tab:liq_kennzahlen_anforderungen_projektmanagement} abgebildeten
Liquidität-Kennzahlen basieren ebenfalls auf den Anforderungen und auf zusätzlichen
Diskussionen mit der Geschäftsleitung. Diese Kennzahlen sollen in Zukunft
projektübergreifend gemessen und zur Liquiditätsplanung des Unternehmens
verwendet werden können.

\begin{center}
    \begin{longtable}{lp{2cm}p{3cm}p{8cm}}
        \toprule \textbf{Nr.} & \textbf{Stufe} & \textbf{Kennzahl} & \textbf{Beschreibung} \\
        \midrule \addtocounter{kcounter}{1}K\arabic{kcounter} & Kalender- monat & Bruttokosten &
            Die Bruttokosten pro Monat lassen sich durch die Summe aller Bruttokosten
            aller Projekte berechnen.\\
        \midrule \addtocounter{kcounter}{1}K\arabic{kcounter} & Kalender- monat & Nettokosten &
            Die Nettokosten pro Monat lassen sich durch die Summe aller Nettokosten
            aller Projekte berechnen.\\
        \midrule \addtocounter{kcounter}{1}K\arabic{kcounter} & Kalender- monat & Fremdkosten &
            Die Fremdkosten pro Monat werden durch die Summe der Differenz der Brutton-
            und Nettokosten aller Projekte berechnet.\\
        \midrule \addtocounter{kcounter}{1}K\arabic{kcounter} & Kalender- monat & Offen &
            Die noch offenen Beträge pro Monat werden durch die Summe in diesem
            Monat geplanter, aber noch nicht in Rechnung gestellter Kosten berechnet.\\
        \midrule \addtocounter{kcounter}{1}K\arabic{kcounter} & Planung & Nächste 30 Tage &
            asdf\\
        \midrule \addtocounter{kcounter}{1}K\arabic{kcounter} & Planung & Weiter 30 Tage &
            asdf\\
        \midrule \addtocounter{kcounter}{1}K\arabic{kcounter} & Planung & Auf sicher ungeplant &
            asdf\\
        \midrule \addtocounter{kcounter}{1}K\arabic{kcounter} & Planung & Offeriert &
            asdf\\
        \bottomrule
        \caption[Liquidität-Kennzahlen-Anforderungen an das neue Projektmanagement]{Liquidität-Kennzahlen-Anforderungen 
            an das neue Projektmanagement\footnotemark}
        \label{tab:liq_kennzahlen_anforderungen_projektmanagement}
    \end{longtable}
\end{center}
\footnotetext{Eigene Darstellung}