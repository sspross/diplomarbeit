Die Erarbeitung des Lösungsansatzes wird anhand den Erkenntnissen
aus den theoretischen Grundlagen des Projektmanagements aus Kapitel \ref{chap:theorie_teil},
den gewonnenen Informationen aus der Analyse des IST-Zustandes aus Kapitel \ref{chap:analyse} 
und des Branchenvergleiches aus Kapitel \ref{chap:branchenvergleich} erstellt.
Das Resultat ist ein überarbeiteter Projektablauf, der die definierten Anforderungen 
aus Kapitel \ref{chap:anforderungen} erfüllt.

\section{Projektablauf}
Die nachfolgende Grafik \ref{pic:01_projektablauf} zeigt den neuen Projektablauf 
mit einer Aufteilung in vier Hauptabschnitte. Jedem Abschnitt wurden Resultate
hinzugefügt, die während dieser Phase erarbeitet oder durchgeführt werden sollen.

\begin{figure}[htbp]
\begin{center}
\includegraphics[width=0.99\textwidth,angle=0]{./bilder/loesung/01_projektablauf.pdf}
\caption[Aufteilung des Projektablaufes mit Resultaten]{Aufteilung des Projektablaufes mit Resultaten\footnotemark}
\label{pic:01_projektablauf}
\end{center}
\end{figure}
\footnotetext{Eigene Darstellung}

In den folgenden Kapiteln wird einzeln auf die vier Hauptabschnitte und deren
Resultate eingegangen. Es wird aufgezeigt, wie die jeweils erwarteten Resultate 
erarbeitet werden sollen.

Zusätzlich werden die Stellen hervorgehoben, wo der Einsatz und die Verwendung 
von Instrumenten sinnvoll ist. Die Wahl dieser Instrumente wird später in einem
separaten Kapitel \ref{chap:instrumentenwahl} behandelt. Zum besseren Verständnis
wird an den jeweiligen Stellen das nötigen Instrument mit einer fortlaufenden
Nummer (\textbf{I}n) versehen. So kann darauf im separaten Kapitel wieder Bezug 
genommen werden.

\newcounter{icounter}
\subsection{Projektdefinition}
In der Grafik \ref{pic:02_01_projektdefinition} ist der Ablauf der Projektdefinition 
ersichtlich. Die potenziellen Projekte (\textbf{0}) gelangen weiterhin zu den
Partnern (\textbf{1.1}), die entscheiden ob ein Projekt angenommen werden soll
oder nicht (\textbf{1.2}). Im Falle einer Absage, wird das dem Kunden von einem
Partner mitgeteilt (\textbf{2.1} bis \textbf{2.3}).

Wird das Projekt angenommen, bestimmen die Partner einen Hauptverantwortlichen
Partner und einen Verantwortlichen Projektleiter (\textbf{3.1}). Der Projektleiter
erstellt darauf den Projektbrief (\textbf{3.2}) und danach kümmert er sich um
die Projektorganisation (\textbf{3.3}).

\begin{figure}[htbp]
\begin{center}
\includegraphics[width=0.99\textwidth,angle=0]{./bilder/loesung/02_01_projektdefinition.pdf}
\caption[Ablauf der Projektdefinitionsphase]{Ablauf der Projektdefinitionsphase\footnotemark}
\label{pic:02_01_projektdefinition}
\end{center}
\end{figure}
\footnotetext{Eigene Darstellung}

\subsubsection{Projektbrief}
Der Projektbrief beinhaltet alle wichtigen Angaben zum Projekt. Der Projektbrief 
sollte eine A4 Seite nicht überschreiten, da er allen Mitarbeitern im Projekt 
möglichst einfach und klar den Kern des Projektes nähre bringen soll.

In der nachfolgenden Tabelle \ref{tab:projektbrief} sind
die Elemente des Projektbriefs dargestellt und zusätzlich beschrieben.

\begin{longtable}{lp{10cm}}
    \toprule \textbf{Element} & \textbf{Beschreibung} \\
    \midrule Kunde, Kontaktperson &
        Namen des Kunden und Kontaktperson \\
    \midrule Projekt &
        Namen des Projektes \\
    \midrule Datum &
        Erstellungsdatum des Projektbriefs \\
    \midrule Hintergrund &
        Beschreibung der Ausgangslage. Was müssen wir wissen? \\
    \midrule Aufgabe &
        Was möchte der Kunde von uns? \\
    \midrule Ziel &
        Was wollen wir mit dem Projekt erreichen? \\
    \midrule Zielgruppe &
        Mit wem kommunizieren wir? Wer ist die Zielgruppe? \\
    \midrule Botschaft und USP &
        Was muss das Produkt kommunizieren und was ist das Verkaufsargument? \\
    \midrule Tonalität &
        Wie kommunizieren wir das Produkt? \\
    \midrule Vorgaben, Obligatorisches &
        Was muss verwendet und beachtet werden? Gibt es Vorgaben? \\
    \bottomrule
    \caption[Aufbau eines Projektbriefs]{Aufbau eines Projektbriefs\footnotemark}
    \label{tab:projektbrief}
\end{longtable}
\footnotetext{Eigene Darstellung}

Der Projektbrief kann der Projektleiter in Zusammenarbeit mit dem verantwortlichen 
Partner erarbeiten. Sobald dieser erstellt ist, sollte klar sein, was für 
Ressourcen für das Projekt benötigt werden und der Projektleiter kann sich um 
die Projektorganisation kümmern.

\subsubsection{Projektorganisation}
Beim Aufbau der Projektorganisation macht sich der Projektleiter Gedanken dazu,
was für Ressourcen er für die Umsetzung des Projektes benötigt. Dazu wählt er,
wenn nötig unter Absprache mit dem verantwortlichen Partner, die geeigneten 
Ressourcen. Dies ergibt dann, wie in der Grafik \ref{pic:03_projektorganisation} 
abgebildet, eine Auftrags-Projektorganisation.

\begin{figure}[htbp]
\begin{center}
\includegraphics[width=0.7\textwidth,angle=0]{./bilder/loesung/03_projektorganisation.pdf}
\caption[Auftrags-Projektorganisation]{Auftrags-Projektorganisation\footnotemark}
\label{pic:03_projektorganisation}
\end{center}
\end{figure}
\footnotetext{Eigene Darstellung}

Der Projektleiter sucht sich die geeignetsten Mitarbeiter aus, um eine saubere 
Projektplanung erstellen zu können. Gewisse Mitarbeiter können auch nur für die 
Projektplanung, also zum Beispiel als Experten für die Aufwandsschätzung, 
eingeplant werden. Es ist wichtig, dass zu diesem Zeitpunkt noch keine 
Ressourcenplanung erstellt wird. Ist zu diesem Zeitpunkt auch nicht möglich,
da noch keine Projektplanung existiert.

Der Projektleiter eröffnet nun das Projekt im Projektmanagement-Tool 
(\textbf{\addtocounter{icounter}{1}I\arabic{icounter}}) mit einer fortlaufenden
Projektnummer und einer eindeutigen Projektbezeichnung, damit die Aufwände
der Projektplanung bereits auf das Projekt rapportiert werden können.

\subsection{Projektplanung}
In der Grafik \ref{pic:02_02_projektplanung} ist der Ablauf der Projektplanung 
ersichtlich. Als erstes überprüft und sammelt der Projektleiter alle nötigen
Unterlagen, die zur Planung des Projektes notwendig sind (\textbf{3.5}). Falls
noch Unterlagen oder Informationen von Seiten des Kunden fehlen (\textbf{3.6})
fordert er diese beim Kunden ein (\textbf{4.1}). Sobald alle Informationen
vorhanden sind, werden die Arbeitspakete zusammen mit den Mitarbeitern und 
Experten erstellt (\textbf{5.1}). Sind die Arbeitspakete definierte, werden
diese ebenfalls vom Projektleiter in Zusammenarbeit mit den Mitarbeitern und
Experten geschätzt (\textbf{5.2}). Alle Mitarbeiter müssen die hierfür aufgewendete
Zeit auf das Projekt rapportieren (\textbf{\addtocounter{icounter}{1}I\arabic{icounter}}).

Der Projektleiter erstellt danach die Offerte und macht einen Vorschlag mit
realistischen Timings und Meilensteinen (\textbf{5.4}). Diese bespricht er
mit dem verantwortlichen Partner (\textbf{5.4}). Der Partner wiederum bespricht
danach die Offerte mit dem Kunden (\textbf{5.6}). Wenn weitere Änderungen oder
Anpassungen nötig sind (\textbf{5.7}), nimmt diese der Projektleiter vor. Sobald
die Offerte vom Kunden abgesegnet ist, startet das eigentliche Projekt (\textbf{6.1}).

\begin{figure}[htbp]
\begin{center}
\includegraphics[width=0.99\textwidth,angle=0]{./bilder/loesung/02_02_projektplanung.pdf}
\caption[Ablauf der Projektplanungsphase]{Ablauf der Projektplanungsphase\footnotemark}
\label{pic:02_02_projektplanung}
\end{center}
\end{figure}
\footnotetext{Eigene Darstellung}

\subsubsection{Arbeitspakete}
Die in Zusammenarbeit mit den Mitarbeitern, Experten und sofern nötig mit dem
verantwortlichen Partner erstellten Arbeitspakete, werden vom Projektleiter
in Todo Listen im Projektmanagement-Tool erfasst (\textbf{\addtocounter{icounter}{1}I\arabic{icounter}}).
Wo sinnvoll und zu diesem Zeitpunkt schon offensichtlich ordnet er den Todos
auch schon die zuständigen Mitarbeiter zu.

\subsubsection{Aufwandsschätzung}
Mit Team, Experten

\subsubsection{Terminplanung}
Realistische Zeitplanung, Meilensteine, Kundenvorgaben?

\subsection{Projektkontrolle}
\subsubsection{Terminkontrolle}
Meilensteine, Todos

\subsubsection{Kostenkontrolle}
Kostenübersicht, rapportierte Stunden

\subsection{Projektabschluss}
\subsubsection{Qualitätskontrolle}
Verantwortlicher, Tests

\subsubsection{Produktabnahme}
Go live, Kundenübergabe

\subsubsection{Reflektion}
Was kann gelernt werden

\section{Instrumentenwahl}\label{chap:instrumentenwahl}
Im neuen Projektablauf des Lösungsansatz wurde an diversen Stellen auf Instrumente
verwiesen, für die in diesem Kapitel noch die richtige Wahl getroffen werden
muss. Als erstes werden die einzelnen Verwendungen aufgelistet und beschrieben.
In einem weiteren Schritt werden verschiedene Varianten verglichen. Danach
kann die Geschäftsleitung der allink entscheiden, welche Instrumente sie
in Zukunft einsetzen möchte.

\subsection{Anforderungen aus dem Projektablauf}
In der nachfolgenden Tabelle \ref{tab:projektablauf_instrumente} sind alle
Anforderungen an Instrumente, auf die im Projektablauf verwiesen wurde, aufgelistet 
und beschrieben.

\begin{longtable}{lp{14cm}}
    \toprule \textbf{Nr.} & \textbf{Beschreibung} \\
    \midrule AI1 & Ein Instrument, um ein neues Projekt zu eröffnen und um eine 
        eindeutige Projektnummer zu vergeben. \\
    \midrule AI2 & Ein Instrument, mit dem alle Projektmitarbeiter ihre 
        aufgewendeten Stunden rapportieren können. \\
    \midrule AI3 & Ein Instrument, um die Arbeitspakete bzw. Todos eines Projektes
        verwalten und überwachen zu können. \\
    \midrule AI4 & Ein Instrument, um die Ressourcenplanung erstellen zu können. \\
    \midrule AI5 & Ein Instrument, um die Meilensteine eines Projektes verwalten 
        und überwachen zu können.\\
    \midrule AI6 & Ein Instrument, um die Geldflüsse wie Offerten, Rechnungen 
        und Teilzahlungen verwalten und überwachen zu können. \\
    \midrule AI7 & Ein Instrument, um alle Projektdaten archivieren zu können. \\
    \bottomrule
    \caption[Im Projektablauf benötigte Instrumente]{Im Projektablauf benötigte 
        Instrumente\footnotemark}
    \label{tab:projektablauf_instrumente}
\end{longtable}
\footnotetext{Eigene Darstellung}

Es können natürlich mehrere Instrumente mit einer Variante abgedeckt werden.
Zum Beispiel mit der Projektmanagementsoftware Metronom\footnote{Metronom ist die 
Projektmanagementsoftware laut Branchenvergleich im  die FEINHEIT GmbH einsetzt.
Vgl. Kapitel \ref{chap:branchenvergleich}.}. Sie wurden jedoch absichtlich getrennt 
aufgelistet, damit für jedes Instrument verschieden Varianten aufgezeigt werden können.

\subsection{Instrumentenvergleich}
\newcounter{vcounter}
In der nachfolgenden Tabelle \ref{tab:instrumenten_varianten} sind alle
Instrumente und mögliche Varianten, die zur Auswahl stehen, aufgelistet und
beschrieben.

\begin{longtable}{lllp{9cm}}
    \toprule
    \textbf{Anf.} & \textbf{Inst.} & \textbf{Beschreibung} \\
    
    \midrule AI1 
    & \addtocounter{vcounter}{1}I\arabic{vcounter} & Metronom & Mit Metronom bla Mit Metronom bla Mit Metronom bla Mit Metronom bla Mit Metronom bla \\
    & \addtocounter{vcounter}{1}I\arabic{vcounter} & Metronom & Mit Metronom bla \\
    
    \midrule AI2 
    & \addtocounter{vcounter}{1}I\arabic{vcounter} & Metronom & Mit Metronom bla \\
    & \addtocounter{vcounter}{1}I\arabic{vcounter} & Metronom & Mit Metronom bla \\
    
    \bottomrule
    \caption[Zur Auswahl stehende Varianten]{Zur Auswahl stehende Varianten\footnotemark}
    \label{tab:instrumenten_varianten}
\end{longtable}
\footnotetext{Eigene Darstellung}

\subsection{Variantenentscheid}
Welche Instrumente werden gewählt. Make or buy...

\section{Anforderungsabdeckung}
Da der Lösungsansatz mit dem neuen Projektablauf und der Wahl der Instrumente
abgeschlossen ist, muss noch überprüft werden, ob die Anforderungen an den
neuen Projektablauf abgedeckt werden konnten.

Dazu werden die im Kapitel \ref{chap:akzeptanzkriterien} definierten 
Akzeptanzkriterien durchlaufen und geprüft. Dies ist in der nachfolgenden Tabelle 
\ref{tab:akzeptanzkriterien_test} ersichtlich. Zur besseren Lesbarkeit wurde zur 
Akzeptanzkriteriumsnummer das Kriterium ein weiteres mal aufgelistet. Dazu wird 
der Status der Erfüllung gezeigt und in kurzen Worten beschrieben, wodurch das
Kriterium erfüllt wird.

\clearpage

\begin{longtable}{lp{6cm}p{5cm}l}
    \toprule \textbf{Nr.} & \textbf{Kriterium} & \textbf{Beschreibung} & \textbf{Status} \\
    \midrule AK1 &
        Wurde für das Projekt ein Hauptverantwortlicher Projektleiter definiert? &
        Wird in der Projektdefinitionsphase Schritt 3.1. definiert. &
        erfüllt \\
    \midrule AK2 &
        Wurde für das Projekt ein Hauptverantwortlicher Partner definiert? &
        Wird ebenfalls in der Projektdefinitionsphase Schritt 3.1 definiert. &
        erfüllt \\
    \midrule AK3 &
        Existiert für das Projekt ein Projektbrief? &
        Wird in der Projektdefinitionsphase Schritt 3.2 definiert. &
        erfüllt \\
    \midrule AK4 &
        Können für das Projekt Meilensteine und Arbeitspakete definiert werden? &
        Wird mit der Software S8 abgedeckt. &
        erfüllt \\
    \midrule AK5 &
        Können die Mitarbeiter auf das Projekt Stunden rapportieren? &
        Wird mit der Software S9 abgedeckt. &
        erfüllt \\
    \midrule AK6 &
        Sind die definierten Projektkennzahlen in einem System ersichtlich? &
        Wird mit der Software S10 abgedeckt. &
        erfüllt \\
    \midrule AK7 &
        Sind die definierten Liquiditäts-Kennzahlen in einem System ersichtlich? &
        Wird mit der Software S10 abgedeckt. &
        erfüllt \\
    \midrule AK8 &
        Existiert eine klare Struktur zur Ablage der Projektdaten? &
        Wurde Projektübergreifend definiert, Abbildung \ref{pic:05_ablagestruktur} &
        erfüllt \\
    \midrule AK9 &
        Wurde für das Projekt ein Hauptverantwortlicher für die Qualitätssicherung definiert? &
        Wurde in der Projektabschlussphase Schritt 7.2 definiert. &
        erfüllt \\
    \bottomrule
    \caption[Überprüfung der Akzeptanzkriterien der definierten Anforderungen]{Überprüfung 
        der Akzeptanzkriterien der definierten Anforderungen\footnotemark}
    \label{tab:akzeptanzkriterien_test}
\end{longtable}
\footnotetext{Eigene Darstellung}

\clearpage

Es wurden alle Akzeptanzkriterien erfüllt. Ob nun der in der Theorie geplante
Lösungsansatz auch in der Praxis umgesetzt werden kann und die Akzeptanzkriterien
in der Realität auch tatsächlich erfüllt werden, wird der im nächsten und
letzten Kapitel durchgespielte ``Proof of Concept'' zeigen.
