Der Branchenvergleich mit den anderen Unternehmen mit ähnlicher Grösse und
Dienstleistungen war sehr interessant und aufschlussreich. Bei der Panter IIc
ist vor allem der Ansatz, dass alle Mitarbeiter bei Entscheidungen des 
Unternehmens mitreden dürfen sehr interessant. Sie befinden sich jedoch noch
in einer Unternehmensgrösse, bei der sich viele Herausforderungen, denen die allink 
gegenüber steht, noch nicht ergeben haben. Auch hat allink gewisse Herausforderungen,
wie zum Beispiel die Beschäftigung von Praktikanten und Lehrlingen, schon gemeistert.

Die FEINHEIT GmbH bietet da mehr interessante Ansätze und Lösungen. Vor allem
die Struktur scheint gut zu skalieren und ermöglicht anscheinend ein gesundes Wachstum.
Auch ist die Erschaffung von Stellen wie die Qualitätssicherung und das Sekretariat,
die keinen direkten Profit für das Unternehmen generieren, ein Indiz für ein
gutes Vorgehen und besseres Projektmanagement. Laut eigenen Aussagen machen sie
überwiegend den konsequenten Einsatz des Projektmanagement Tools Metronom dafür
verantwortlich. Dies ist jedoch nur ein Mittel zum Zweck und kann auf
diverse Arten umgesetzt werden.

Bei der Erarbeitung des Konzeptes des neuen Projektmanagements für allink 
im Kapitel \ref{chap:konzept} werden diese Erkenntnisse sicherlich einfliessen.
