%
%  Antrag, Aufgabenstellung Diplomarbet
%
%  Created by Silvan Spross on 2011-02-03.
%
\documentclass[]{scrreprt}
\usepackage[ngerman]{babel}

% Use utf-8 encoding for foreign characters
\usepackage[utf8]{inputenc}

% Setup for fullpage use
\usepackage{fullpage}

% Package for including code in the document
\usepackage{listings}

% If you want to generate a toc for each chapter (use with book)
\usepackage{minitoc}

% This is now the recommended way for checking for PDFLaTeX:
\usepackage{ifpdf}

\ifpdf
    \usepackage[pdftex]{graphicx}
\else
    \usepackage{graphicx}
\fi

\title{Aufgabenstellung\\
    Diplomarbeit in Informatik}
    
\author{Studierender - Silvan Spross\\
    Projektbetreuer - Beat Seeliger\\
    Auftraggeber - Michael Walder, allink.creative\\
    \\
    HSZ-T - Technische Hochschule Zürich}
    
\date{26. Februar 2011}

\begin{document}

    \ifpdf
        \DeclareGraphicsExtensions{.pdf, .jpg, .tif}
    \else
        \DeclareGraphicsExtensions{.eps, .jpg}
    \fi

    \maketitle

    \pagenumbering{arabic}

    % \tableofcontents

    \chapter{Aufgabenstellung Diplomarbeit}

    \section{Thema}
    Definition und Optimierung der Projektprozesse bei allink.creative

    \section{Ausgangslage}
    Die Agentur allink.creative ist im letzten Jahr stark gewachsen. Von zehn
    Mitarbeitern im Februar 2010 auf siebzehn Mitarbeiter im Februar 2011. Dies hat 
    zur Auswirkung, dass gewisse Funktionen und Prozesse neu definiert und bestehende
    überarbeitet werden müssen, um weiterhin effizient, oder wenn möglich noch 
    effizienter, arbeiten zu können. Die Agentur arbeitet überwiegend mit Apple
    Computern und setzt gewisse Software ein, die die Geschäftsleitung beibehalten 
    möchte. Die konkreten Vorstellungen und Vorgaben müssen von dem Studierenden
    in der Arbeit erfasst werden.

    \section{Ziel der Arbeit}
    Bereiche wie die Stundenrapportierung, die Projektplanung und das Projektcontrolling 
    können mit Hilfe von IT-Lösungen massgebend optimiert und vereinfacht werden. 
    In dieser Arbeit sollen die Herausforderungen, die der Auftraggeber in der 
    Planung und im Controlling eines Projektes zu bewältigen hat, erfasst und 
    Lösungsvorschläge evaluiert werden. Der Fokus liegt dabei auf der besseren 
    Messbarkeit des finanziellen Erfolges eines Projektes und des gesamten 
    Unternehmens.
    
    Die Arbeit grenzt sich ganz klar von der Finanzbuchhaltung ab, da
    diese keinen direkten Einfluss auf die Projekte hat und bei allink.creative 
    bei einen Treuhänder ausgelagert wurde.
    
    % Folgende Ziele sollen erreicht werden:
    % 
    % \begin{itemize}
    %     \item Bla
    % \end{itemize}
    % 
    % Folgende Punkte werden abgegrenzt, da es den Rahmen der Arbeit sprengen 
    % würde:
    % 
    % \begin{itemize}
    %     \item Bla
    % \end{itemize}

    \section{Aufgabenstellung}
    Folgende Aufgaben soll der Studierende während dieser Arbeit bewältigen:
    
    \begin{itemize}
        \item Ist-Situation im Bereich Projektablauf der allink.creative erfassen
        \item Kennzahlen definieren, die in Zukunft auf Projektebene gemessen 
            werden sollen
        \item Eine Recherche der Prozesse in ähnlich funktionierenden KMUs durchführen
        \item Neue Prozesse definieren und bestehende, sofern sinnvoll, überarbeiten
        \item Evaluation von IT-Lösungen, die diese Prozesse möglichst passend 
            für den Auftraggeber abbilden und die definierten Kennzahlen generieren können
    \end{itemize}

    \section{Erwartete Resultate}
    Der Studierende soll dem Auftraggeber ein Dokument erstellen, das folgende 
    Punkte beinhaltet: 
    
    \begin{itemize}
        \item Beschreibung der Ist-Situation im Bereich Projektablauf
        \item Übersicht der bestehenden Software beim Auftraggeber
        \item Kennzahlen, die auf Projektebene gemessen werden sollen
        \item Darstellung der neuen und überarbeiteten Prozesse
        \item Übersicht der bestehenden Software in der neuen Prozesslandschaft
        \item Softwareempfehlungen für die komplette Prozessabbildung
        \item ``Make or Buy''-Entscheid mit dem Auftraggeber
    \end{itemize}
    
    Während der Arbeit sollen als ``Proof of Concept'' die selben zur Zeit verwendeten
    Prozesse und Tools verwendet werden. Sobald die neuen Prozesse und Tools
    definiert sind und der Auftraggeber einen Entscheid gefällt hat, sollen die
    Informationen ebenfalls in die neue Prozess- und Toollandschaft übertragen werden.

    \section{Geplante Termine}
    Die Termine können zum Zeitpunkt des Antrages noch nicht definitiv 
    festgelegt werden. Sofern jedoch die Planung eingehalten werden kann und 
    freie Termine zur Verfügung stehen, sollten die Termine innerhalb der 
    angegebenen Monate liegen.

    \begin{tabbing}
        \hspace*{4cm}\= \kill
    	Kick-Off:               \> Anfangs März 2011\\
    	Review:                 \> Anfangs April 2011\\
    	Abgabe:                 \> Anfangs Juni 2011\\
    	Schlusspräsentation:    \> Mitte Juni 2011 \\
    \end{tabbing}

    \section{Genehmigung}
    Der Studierende, sein Projektbetreuer und der Studiengangsleiter 
    Informatik erklären sich mit der Aufgabenstellung einverstanden und geben 
    die Arbeit frei zur Erfassung im Einschreibesystem der Hochschule für 
    Technik Zürich.

    \begin{tabbing}
        \hspace*{10cm}\= \kill
    	Silvan Spross, Studierender \> Beat Seeliger, Projektbetreuer \\\\\\
        \line(1,0){150} \> \line(1,0){150} \\\\\\
    	Dr. Olaf Stern, Studiengangsleiter Informatik \\\\\\
        \line(1,0){150}
    \end{tabbing}
    
    \bibliographystyle{plain}
    \bibliography{literaturverzeichnis}
    
\end{document}