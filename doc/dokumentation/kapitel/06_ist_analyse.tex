\section{Vorgehensmodell}
Zur Erstellung einer Ist-Analyse existieren verschiedene Vorgehensmodelle, die
in der Praxis angewendet werden. Ich werde mir nun einige Modelle anschauen und 
mich für ein Vorgehen entscheiden, um die Ist-Analyse des aktuellen
Projektablaufes bei allink zu erstellen.

\subsection{Grochla}
Für meine Analyse könnte ich die ersten drei Phasen des Vorgehensmodell von 
Erwin Grochla\footnote{Erwin Grochla war ein deutscher Ordinarius für Betriebswirtschaftslehre.}
anwenden, welche er in seinem Buch ``Grundlagen der organisatorsichen Gestaltung''\cite[S. 44-74]{grochla1982grundlagen} 
beschreibt.

\subsubsection{Phasen}

\paragraph{1. Phase: Voruntersuchung}
In der Voruntersuchung, auch Pilotstudie genannt, stellt man sich Fragen wie
``Was soll geändert werden?'' und ``Welches sind die Ziele?''. Man verschafft
sich einen Grobüberblick über das eigentliche Problem und die Rahmenbedingungen
möglicher Lösungen. Auch entscheidet man in der Voruntersuchung, ob der Prozess
fortgesetzt oder abgebrochen werden soll.

\paragraph{2. Phase: Ist-Aufnahme}
In der zweiten Phase erfasst man den aktuellen Zustand indem 
man das zu Untersuchende aus möglichst vielen Betrachtungswinkeln analysieren.
Hierbei können verschiedene Techniken zum Zuge kommen, die zu einem späteren
Zeitpunkt erläutert werden.

\paragraph{3. Phase: Ist-Kritik}
In der dritten Phase widment man sich den erhobenen Daten aus der 2. Phase, 
der ``Ist-Aufnahme''. Die genauen Ursachen der Probleme sollen herausgeschält
werden, damit nicht nur Symptome behandelt werden.

\subsubsection{Techniken}
Für die drei Phasen existieren wiederum verschiedene Techniken,
um diese durchzuführen und darzustellen.

\paragraph{Techniken der Voruntersuchung}
tbd

\paragraph{Techniken der Ist-Aufnahme}
tbd

\paragraph{Techniken der Ist-Kritik}
tbd

\subsection{3-Phasen-Modell von Lewin}
Das 3-Phasen-Modell von Kurt Lewin\footnote{Kurt Lewin gilt als einer der einflussreichsten Pioniere der Psychologie}, 
welches er im Buch ``Frontiers in group dynamics''\cite[S. 5-41]{lewinfrontiers} 
beschreibt, ist ein Vorgehensmodell des Veränderungsmanagements (``change management'').
Daraus könnte ich die erste Phase ``unfreezing'' (Auftauphase) verwenden, um die Ist-Situation
zu analysieren.

Folgende acht Schritte sollte man laut John Kotter\cite{kotter2006pinguin} im Veränderungsprozess
des ``change management'' durchlaufen:

\begin{enumerate}
    \item Bewusstsein für die Dringlichkeit schaffen
    \item Verantwortliche mit Veränderungsbereitschaft gewinnen und zusammenbringen
    \item Die Zukunftsvision ausformulieren und eine Strategie entwickeln, wie Sie dahin kommen
    \item Die Zukunftsvision bekannt machen
    \item Handeln im Sinne der neuen Vision und der Ziele ermöglichen
    \item Kurzfristige Erfolge planen und gezielt herbeiführen
    \item Erreichte Verbesserungen systematisch weiter ausbauen
    \item Das Neue fest verankern
\end{enumerate}

Dieses Modell ist für meinen Einsatzzweck zu übergreifend. Meiner Meinung nach
befindet sich allink während der Erstellung meiner Arbeit bereits in Schritt drei.

\subsection{MindMap}
verschaffe mir mit Hilfe von MindMaps einen besseren Überblick über
den Inhalt und den Umfang der einzelnen Themen die ich in dieser Arbeit 
abhandeln werde.

\begin{quotation}
Eine Mind-Map beschreibt eine besonders von Tony Buzan geprägte kognitive 
Technik, die z.B. zur Erschliessung und visuellen Darstellung eines Themengebietes, 
zur Planung oder für Mitschriften genutzt werden kann. Hierbei soll das Prinzip 
der Assoziation helfen, Gedanken frei zu entfalten und die Fähigkeiten des Gehirns 
zu nutzen. Die Mind-Map wird nach bestimmten Regeln erstellt und gelesen. Den 
Prozess bzw. das Themengebiet bzw. die Technik bezeichnet man als Mind-Mapping.
\cite{wikipedia_mindmap}
\end{quotation}

Ich verwende die Technik der MindMaps schon seit einigen Jahren und verwende
sie sehr oft in Projekten um mir einen Überblick über Themen zu verschaffen.

\subsection{Entscheidung}
Ich entscheide mich für

\section{Aktueller Projektablauf}

\subsection{Aktueller Projektablauf}
Zur Zeit existiert bei allink kein offizieller Projektablauf. Ich bezeichne
den heutigen Zustand als ``natürliches Vorgehen''. Ich verstehe darunter,
dass jede Person vor zu selbst beurteilt, was als nächstes geschehen soll bzw.
Sinn macht und dieses mit anderen Entscheidungsträgern bei Bedarf abspricht.
Dies birgt natürlich einige Schwächen und Gefahren.

\begin{figure}[htbp]
\begin{center}
\includegraphics[width=0.95\textwidth,angle=0]{./mindmaps/ist_analyse_projektablauf.pdf}
\caption{MindMap Ist-Analyse allink.creative des aktuellen Projektablaufes}
\label{pic:ist_analyse_projektablauf}
\end{center}
\end{figure}

Ich werde in der eigentlichen Ist-Analyse genauer auf die einzelnen Vor- und
Nachteile des aktuellen Projektablaufes eingehen und deren Schwächen und 
Gefahren erläutern.

\subsection{Eingesetzte Software}
Bei der Analyse der zur Zeit eingesetzten Sofware beschränke ich mich auf jene,
die einen direkten Einfluss auf den aktuellen Projektablauf bei allink haben.
Ich liste keine Software auf, die zur Erbringung der eigentlichen Dienstleistungen
von allink eingesetzt werden.

\begin{figure}[htbp]
\begin{center}
\includegraphics[width=0.95\textwidth,angle=0]{./mindmaps/ist_analyse_software.pdf}
\caption{MindMap Ist-Analyse allink.creative der eingesetzten Software}
\label{pic:ist_analyse_software}
\end{center}
\end{figure}

Den genauen Einsatzzweck der verschiedenen Software im aktuellen Projektablauf
erläutere ich ebenfalls in der eigentlichen Ist-Analyse genauer.

\subsection{Voruntersuchung}
\subsection{Ist-Aufnahme}
\subsection{Ist-Kritik}
\subsubsection{Wenig Standards}
\subsubsection{Unorganisiert}
\subsubsection{Ineffizient}
\subsubsection{Keine Qualitätssicherung}
\subsubsection{Minimale Messbarkeit}
\subsection{Gefahren}
\subsubsection{Kostenrisiko}
\subsubsection{Keine Nachvollziehbarkeit}
\subsubsection{Ressourcenüberbelastung}
\subsubsection{Reputationsverlust}
\section{Zurzeit eingesetzte Software}
\section{Konklusion}