%
%  Kickoff Protokoll
%
%  Created by Silvan Spross on 2010-10-21.
%
\documentclass[]{scrreprt}
\usepackage[ngerman]{babel}

% Use utf-8 encoding for foreign characters
\usepackage[utf8]{inputenc}

% Setup for fullpage use
\usepackage{fullpage}

% Running Headers and footers
%\usepackage{fancyhdr}

% Multipart figures
%\usepackage{subfigure}

% More symbols
%\usepackage{amsmath}
%\usepackage{amssymb}
%\usepackage{latexsym}

% Surround parts of graphics with box
\usepackage{boxedminipage}

% Package for including code in the document
\usepackage{listings}

% If you want to generate a toc for each chapter (use with book)
\usepackage{minitoc}

% This is now the recommended way for checking for PDFLaTeX:
\usepackage{ifpdf}
\usepackage{url}

\ifpdf
    \usepackage[pdftex]{graphicx}
\else
    \usepackage{graphicx}
\fi

\title{Kick-off Protokoll}
    
\author{Studierender - Silvan Spross\\
    Projektbetreuer - Beat Seeliger\\
    Auftraggeber - Michael Walder, allink.creative\\
    \\
    HSZ-T - Technische Hochschule Zürich}
    
\date{11. März 2011}

\begin{document}

    \ifpdf
        \DeclareGraphicsExtensions{.pdf, .jpg, .tif}
    \else
        \DeclareGraphicsExtensions{.eps, .jpg}
    \fi

    \maketitle

    \pagenumbering{arabic}

    % \tableofcontents

    \chapter{Kick-off Protokoll}

    \section{Diplomarbeit}
    Definition und Optimierung der Projektprozesse bei allink.creative

    \section{Beschlüsse anhand Ziele Kick-Off Meeting}
    Alle Fragen\footnote{\url{https://github.com/sspross/diplomarbeit/blob/master/doc/reglemente/Ablauf-Bachelorarbeit_Studiengang-Informatik-der-HSZ-T_V1.3.pdf}} 
    aus dem Kick-Off Meeting konnten geklärt werden:
    \begin{enumerate}
        \item Steht der Auftraggeber hinter dieser Arbeit? \\
            {\bf Ja} und der Auftraggeber wir den Studierenden so gut wie
            möglich unterstützen.
        \item Sind die fachliche Kompetenz und die Verfügbarkeit des Betreuers 
            sichergestellt? \\
            {\bf Ja}, da der Betreuer ebenfalls selbständig und Partner in 
            einem KMU ist.
        \item Sind die Urheberrechte und Publikationsrechte geklärt? \\
            {\bf Ja}, die Arbeit kann vollständig öffentlich zugänglich gemacht
            werden.
        \item Bekommt der Studierende die notwendige logistische und beratende 
            Unterstützung des Auftraggebers? \\
            {\bf Ja}
        \item Entspricht die Arbeit den Anforderungen für eine Diplomarbeit? \\
            {\bf Ja}, jedoch muss die Aufgabenstellung noch um den Punkt 
            ``Recherche'' und ein zusätzliches Resultat ``Proof of Concept'' 
            ergänzt werden.
        \item Ist die Arbeit klar abgegrenzt und terminlich entkoppelt von den 
            Prozessen des Auftraggebers? \\
            {\bf Ja}
        \item Ist eine Grobplanung vorhanden? \\
            {\bf Ja}, auch wenn sich diese möglicherweise noch verändern wird.
        \item Ist die Arbeit technisch und terminlich umsetzbar? \\
            {\bf Ja}, jedoch ist der Zeitplan ziemlich eng.
        \item Sind die nächsten Schritte klar formuliert? \\
            {\bf Ja}
    \end{enumerate}
    
\end{document}