Im neuen Projektablauf des Lösungsansatz wurde an diversen Stellen auf Instrumente
verwiesen, für die in diesem Kapitel noch die richtige Wahl getroffen werden
muss. Als erstes werden die einzelnen Verwendungen aufgelistet und beschrieben.
In einem weiteren Schritt werden verschiedene Varianten verglichen. Danach
kann die Geschäftsleitung der allink entscheiden, welche Instrumente sie
in Zukunft einsetzen möchte.

\subsection{Anforderungen aus dem Projektablauf}
In der nachfolgenden Tabelle \ref{tab:projektablauf_instrumente} sind alle
Anforderungen an Instrumente, auf die im Projektablauf verwiesen wurde, aufgelistet 
und beschrieben.

\begin{longtable}{lp{14cm}}
    \toprule \textbf{Nr.} & \textbf{Beschreibung} \\
    \midrule AI1 & Ein Instrument, um ein neues Projekt zu eröffnen und um eine 
        eindeutige Projektnummer zu vergeben. \\
    \midrule AI2 & Ein Instrument, mit dem alle Projektmitarbeiter ihre 
        aufgewendeten Stunden rapportieren können. \\
    \midrule AI3 & Ein Instrument, um die Arbeitspakete bzw. Todos eines Projektes
        verwalten und überwachen zu können. \\
    \midrule AI4 & Ein Instrument, um die Ressourcenplanung erstellen zu können. \\
    \midrule AI5 & Ein Instrument, um die Meilensteine eines Projektes verwalten 
        und überwachen zu können.\\
    \midrule AI6 & Ein Instrument, um die Geldflüsse wie Offerten, Rechnungen 
        und Teilzahlungen verwalten und überwachen zu können. \\
    \midrule AI7 & Ein Instrument, um alle Projektdaten archivieren zu können. \\
    \bottomrule
    \caption[Im Projektablauf benötigte Instrumente]{Im Projektablauf benötigte 
        Instrumente\footnotemark}
    \label{tab:projektablauf_instrumente}
\end{longtable}
\footnotetext{Eigene Darstellung}

Es können natürlich mehrere Instrumente mit einer Variante abgedeckt werden.
Zum Beispiel mit der Projektmanagementsoftware Metronom\footnote{Metronom ist die 
Projektmanagementsoftware laut Branchenvergleich im  die FEINHEIT GmbH einsetzt.
Vgl. Kapitel \ref{chap:branchenvergleich}.}. Sie wurden jedoch absichtlich getrennt 
aufgelistet, damit für jedes Instrument verschieden Varianten aufgezeigt werden können.

\subsection{Instrumentenvergleich}
\newcounter{vcounter}
In der nachfolgenden Tabelle \ref{tab:instrumenten_varianten} sind alle
Instrumente und mögliche Varianten, die zur Auswahl stehen, aufgelistet und
beschrieben.

\begin{longtable}{lllp{9cm}}
    \toprule
    \textbf{Anf.} & \textbf{Inst.} & \textbf{Beschreibung} \\
    
    \midrule AI1 
    & \addtocounter{vcounter}{1}I\arabic{vcounter} & Metronom & Mit Metronom bla Mit Metronom bla Mit Metronom bla Mit Metronom bla Mit Metronom bla \\
    & \addtocounter{vcounter}{1}I\arabic{vcounter} & Metronom & Mit Metronom bla \\
    
    \midrule AI2 
    & \addtocounter{vcounter}{1}I\arabic{vcounter} & Metronom & Mit Metronom bla \\
    & \addtocounter{vcounter}{1}I\arabic{vcounter} & Metronom & Mit Metronom bla \\
    
    \bottomrule
    \caption[Zur Auswahl stehende Varianten]{Zur Auswahl stehende Varianten\footnotemark}
    \label{tab:instrumenten_varianten}
\end{longtable}
\footnotetext{Eigene Darstellung}

\subsection{Variantenentscheid}
Welche Instrumente werden gewählt. Make or buy...