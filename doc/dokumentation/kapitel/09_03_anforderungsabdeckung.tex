Da der Lösungsansatz mit dem neuen Projektablauf und der Wahl der Instrumente
abgeschlossen ist, muss noch überprüft werden, ob die Anforderungen an den
neuen Projektablauf abgedeckt werden konnten.

Dazu werden die im Kapitel \ref{chap:akzeptanzkriterien} definierten 
Akzeptanzkriterien durchlaufen und geprüft. Dies ist in der nachfolgenden Tabelle 
\ref{tab:akzeptanzkriterien_test} ersichtlich. Zur besseren Lesbarkeit wurde zur 
Akzeptanzkriteriumsnummer das Kriterium ein weiteres mal aufgelistet. Dazu wird 
der Status der Erfüllung gezeigt und in kurzen Worten beschrieben, wodurch das
Kriterium erfüllt wird.

\clearpage

\begin{longtable}{lp{6cm}p{5cm}l}
    \toprule \textbf{Nr.} & \textbf{Kriterium} & \textbf{Beschreibung} & \textbf{Status} \\
    \midrule AK1 &
        Wurde für das Projekt ein Hauptverantwortlicher Projektleiter definiert? &
        Wird in der Projektdefinitionsphase Schritt 3.1. definiert. &
        erfüllt \\
    \midrule AK2 &
        Wurde für das Projekt ein Hauptverantwortlicher Partner definiert? &
        Wird ebenfalls in der Projektdefinitionsphase Schritt 3.1 definiert. &
        erfüllt \\
    \midrule AK3 &
        Existiert für das Projekt ein Projektbrief? &
        Wird in der Projektdefinitionsphase Schritt 3.2 definiert. &
        erfüllt \\
    \midrule AK4 &
        Können für das Projekt Meilensteine und Arbeitspakete definiert werden? &
        Wird mit der Software S8 abgedeckt. &
        erfüllt \\
    \midrule AK5 &
        Können die Mitarbeiter auf das Projekt Stunden rapportieren? &
        Wird mit der Software S9 abgedeckt. &
        erfüllt \\
    \midrule AK6 &
        Sind die definierten Projektkennzahlen in einem System ersichtlich? &
        Wird mit der Software S10 abgedeckt. &
        erfüllt \\
    \midrule AK7 &
        Sind die definierten Liquiditäts-Kennzahlen in einem System ersichtlich? &
        Wird mit der Software S10 abgedeckt. &
        erfüllt \\
    \midrule AK8 &
        Existiert eine klare Struktur zur Ablage der Projektdaten? &
        Wurde Projektübergreifend definiert, Abbildung \ref{pic:05_ablagestruktur} &
        erfüllt \\
    \midrule AK9 &
        Wurde für das Projekt ein Hauptverantwortlicher für die Qualitätssicherung definiert? &
        Wurde in der Projektabschlussphase Schritt 7.2 definiert. &
        erfüllt \\
    \bottomrule
    \caption[Überprüfung der Akzeptanzkriterien der definierten Anforderungen]{Überprüfung 
        der Akzeptanzkriterien der definierten Anforderungen\footnotemark}
    \label{tab:akzeptanzkriterien_test}
\end{longtable}
\footnotetext{Eigene Darstellung}

\clearpage

Es wurden alle Akzeptanzkriterien erfüllt. Ob nun der in der Theorie geplante
Lösungsansatz auch in der Praxis umgesetzt werden kann und die Akzeptanzkriterien
in der Realität auch tatsächlich erfüllt werden, wird der im nächsten und
letzten Kapitel durchgespielte ``Proof of Concept'' zeigen.