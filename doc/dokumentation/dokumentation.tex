%  --------------------------------------------------------------------------
%  Diplomarbeit Dokumentation
%  Created by Silvan Spross on 2011-04-02.
%  --------------------------------------------------------------------------

%  --------------------------------------------------------------------------
%  Latex Document Settings
%  --------------------------------------------------------------------------
\documentclass[
11pt, % Schriftgrösse
a4paper, % A4 Papier
BCOR25mm, % Absoluter Wert der Bindekorrektur, z.B. BCOR1cm
DIV14, % Satzspiegel festlegen siehe
       % http://www.ctex.org/documents/packages/nonstd/koma-script.pdf
footsepline = false, % Trennlinie zwischen Textkörper und Fußzeile
                     % bei normalen Seiten
headsepline, % Trennlinie zwischen Kopfzeile und Textkörper
             % bei normalen Seiten
twoside, % Zweiseitig
openright,
%halfparskip, % Europäischer Satz mit Abstand zwischen den Absätzen
abstracton, % inkl. Abstract
listof=totocnumbered, % Abb.- und Tab.verzeichnis im Inhaltsverzeichnis
bibliography=totocnumbered % Lit.zeichnis in Inhaltsverzeichnis aufnehmen
]{scrreprt}

\usepackage[automark]{scrpage2} % Gestaltung von kopf- und Fußzeile
\usepackage[ngerman]{babel}
\usepackage[ngerman]{translator}
\usepackage{tocbasic}
\usepackage[utf8]{inputenc}
\usepackage{lmodern} % Latin Modern
\usepackage[T1]{fontenc}
\usepackage{hyphenat}
\usepackage{ae} % Schöne Schriften für PDF-Dateien

% Tradmark
\def\TTra{\textsuperscript{\texttrademark}}

%1.5 Zeilenabstand
\usepackage[onehalfspacing]{setspace}

% Festlegung des Seitenstils (scrpage2)
\pagestyle{scrheadings}
\clearscrheadfoot
\automark[chapter]{section}

\lehead{\sffamily\upshape\headmark}
\cehead{}
\rehead{}
\lefoot[\pagemark]{\upshape \pagemark}
\cefoot{}
\refoot{}
\lohead{}
\cohead{}
\rohead{\sffamily\upshape\headmark}
\lofoot{}
\cofoot{}
\rofoot[\pagemark]{\scshape \pagemark}

% Surround parts of graphics with box
\usepackage{boxedminipage}

% Package for including code in the document
\usepackage{listings}

% If you want to generate a toc for each chapter (use with book)
\usepackage{minitoc}

% Abkürzungsverzeichnis erstellen.
\usepackage[printonlyused]{acronym}

% schöne Tabelle zeichnen
\usepackage{booktabs}
\renewcommand{\arraystretch}{1.4} %Die Zeilenabstände in Tabllen angepasst.

% für variable Breiten
\usepackage{tabularx}

% Durchgestrichener Text
\usepackage[normalem]{ulem} %emphasize weiterhin kursiv

% This is now the recommended way for checking for PDFLaTeX:
\usepackage{ifpdf}

\usepackage[hyperfootnotes=false]{hyperref}
\hypersetup{
  bookmarks=true,         % show bookmarks bar?
  unicode=true,           % non-Latin characters in Acrobat’s bookmarks
  pdftoolbar=true,        % show Acrobat’s toolbar?
  pdfmenubar=true,        % show Acrobat’s menu?
  pdffitwindow=true,      % window fit to page when opened
  pdfstartview={FitH},    % fits the width of the page to the window
  pdftitle={Diplomarbeit},   
  pdfauthor={Silvan Spross},
  pdfsubject={Definition und Optimierung der Projektprozesse bei allink.creative},
  pdfcreator={TeX Live 2009},
  pdfproducer={pdfTeX, Version 3.1415926-1.40.10},
  pdfnewwindow=true,      % links in new window
  colorlinks=true,       % false: boxed links; true: colored links
  linkcolor=blue,          % color of internal links
  citecolor=green,        % color of links to bibliography
  filecolor=magenta,      % color of file links
  urlcolor=cyan          % color of external links
  % linkcolor=black,          % color of internal links
  % citecolor=black,        % color of links to bibliography
  % filecolor=black,      % color of file links
  % urlcolor=black          % color of external links
}

\ifpdf
    \usepackage[pdftex]{graphicx}
\else
    \usepackage{graphicx}
\fi

\makeatletter 
\let\orgdescriptionlabel\descriptionlabel 
\renewcommand*{\descriptionlabel}[1]{% 
  \let\orglabel\label 
  \let\label\@gobble 
  \phantomsection 
  \edef\@currentlabel{#1}% 
  %\edef\@currentlabelname{#1}% 
  \let\label\orglabel 
  \orgdescriptionlabel{#1}% 
} 
\makeatother 

%  --------------------------------------------------------------------------
%  Start Document
%  --------------------------------------------------------------------------
\title{Definition und Optimierung der Projektprozesse bei allink.creative}

\author{Diplomarbeit in Informatik\\
    \\
    Studierender - Silvan Spross\\
	Auftraggeber - Michael Walder, allink.creative\\
    Projektbetreuer - Beat Seeliger\\
    Experte - tbd\\
	\\
	HSZ-T - Technische Hochschule Zürich}

\date{März 2011 bis Juni 2011}

\begin{document}

  \ifpdf
    \DeclareGraphicsExtensions{.pdf, .jpg, .tif}
  \else
    \DeclareGraphicsExtensions{.eps, .jpg}
  \fi
  
  \pagenumbering{Alph}
  
  \maketitle
  \cleardoublepage

  \begin{abstract}
tbd
\end{abstract}
  \cleardoublepage

  \pagenumbering{roman}
  
  \tableofcontents
  \cleardoublepage
  
  \pagenumbering{arabic}
  
  \chapter{Rahmenbedingungen}
  Für das Informatik Diplomstudium an der Fachhochschule Zürich für Technik
HSZ-T wird von den Studenten verlangt eine Diplomarbeit eigenständig zu
verfassen.

\section{Sprache}
Die Semesterarbeit wurde in deutscher Sprache verfasst. Englische Ausdrücke 
wurden immer dort verwendet, wo diese im Sprachgebrauch in den verwendeten 
Programmen genau so gebraucht werden.

Aus Gründen der besseren Lesbarkeit der Diplomarbeit wurde teilweise auf 
die Nennung beider Geschlechter verzichtet. In diesen Fällen ist die 
weibliche Form ausdrücklich inbegriffen.
  
\section{Richtlinien}
Folgende Dokumente mit Richtlinien der Hochschule für Technik Zürich 
wurden für die Diplomarbeit berücksichtigt:

\begin{itemize}
    \item Reglement \cite{hsz_reglement}
    \item Ablauf \cite{hsz_ablauf}
    \item Bewertungskriterien \cite{hsz_bewertungskriterien}
\end{itemize}

  \cleardoublepage
  
  \chapter{Einführung}
  \section{Motivation}
Das Schreiben einer Diplomarbeit bedeutet nebst dem baldigen Abschluss des Studiums
auch einen enormen Aufwand. Deshalb ist die Wahl des Themas und meine
Motivation in diese Arbeit die nötige Zeit und Qualität zu investieren enorm
wichtig. Aus diesem Grund möchte ich kurz erläutern, was mich zu dieser
Arbeit geführt hat.

Ein eigenes Unternehmen zu gründen und erfolgreich zu führen war schon immer
einer meiner Träume. Nach meiner Lehre im Jahre 2004 machte ich mich deshalb
selbstständig und gründete 2005 die SiSprocom GmbH\footnote{\url{http://sisprocom.ch/}}. 
Im ersten Jahr bestanden
die Tätigkeitsfelder überwiegend aus Webdesign und Schulungen. Im Bereich
Webdesign lag der Fokus hauptsächlich auf der Programmierung. Dieser Bereich 
entwickelte sich immer stärker in Richtung 
Applikationsentwicklung und dank Aufträge einer Zürcher Grossbank 
lag der Fokus bald nur noch darauf.

Durch dieses grössere Mandat wuchs die SiSprocom GmbH zwischenzeitlich
auf 3 Mitarbeiter. Dies führte zu massiv mehr Aufwand in der Administration
und schnell wurde klar, dass zwingend Stunden rapportiert und das Schreiben
von Rechnungen vereinfacht werden musste. Wir bedienten uns damals Google 
Docs\footnote{\url{http://docs.google.com/}} um die rapportierten Stunden zentral zu 
verwalten und einer selbst geschriebenen Software um vereinfacht Rechnungen 
verwalten und schreiben zu können.
So lehrreich und spannend die Arbeit in dieser Grossbank auch war, so kompliziert
waren auch ihre Abläufe und Prozesse. Zu diesem Zeitpunkt schwor ich mir, dies
in meiner Unternehmung einmal besser zu lösen.

Ende 2009 realisierte ich, dass ich zwar ein eigenes Unternehmen hatte und
selbständig war, jedoch fast ausschliesslich für einen Kunden arbeitete.
Ich kam zum Schluss, dass dies nicht mein eigentliches Ziel meiner Selbständigkeit 
war und begriff
relativ schnell, dass ich mich von dieser Abhängigkeit nur lösen konnte, wenn
ich mich vollständig aus diesem Mandat zurückziehe.
Dies tat ich dann anfangs 2010 auch und mietete ein Büro in den Räumlichkeiten
der allink GmbH\footnote{\url{http://allink.ch/}}. Deren IT hatte zu diesem Zeitpunkt 
ein paar grössere
Herausforderungen zu bewältigen und ich bot meine Hilfe an. Schnell wurde
daraus eine Partnerschaft und die SiSprocom fusionierte mit der allink.
Heute im März 2011, knapp ein Jahr danach, ist Die SiSprocom GmbH um 30\% 
gewachsen und arbeitet ausschliesslich für die erwähnte Grossbank. Alle anderen Projekte
haben wir in die allink GmbH übernommen. Die administrativen Aufgaben der
SiSprocom GmbH wurden
meinem Vater abgegeben, der zusammen mit einem Treuhänder das laufenden
Mandat und die drei Mitarbeiter gut handhaben kann. So konnte ich mich ganz
auf die neuen Herausforderungen bei der allink konzentrieren.
Die allink GmbH ist inzwischen um 70\% auf 17 Mitarbeiter gewachsen und muss sich dadurch
neuen Herausforderungen in der Organisation und Projektabläufe stellen.

Da mich die Selbständigkeit auch durch das ganze Studium begleitet hat, möchte 
ich meine Diplomarbeit nutzen um unser Unternehmen zu optimieren.

\section{Zielsetzung}
Mir und den anderen drei Partnern bei der allink GmbH ist klar, dass unser
heutiger Projektablauf nicht optimal ist. Zu oft sehen wir uns mit gleichen
Problemen konfrontiert, die in anderen Projekt schon einmal gelöst wurden.
Jedes Mal versucht man daraus zu lernen, ohne etwas konkret festzuhalten oder
wirklich zu verändern. Das liegt meist daran, dass zu viel ansteht und man
die internen Verbesserungen hinter die Aufträgen und Wünschen der Kunden
stellt.
Mein Ziel ist es mit Hilfe dieser Arbeit den aktuellen Projektablauf der
allink GmbH genauer zu untersuchen und auf dessen Vorteile und Nachteile einzugehen.
Danach versuche ich die eigentlichen Anforderungen der verschiedenen Stakeholder
unseres Projektablaufes aufzunehmen und Kennzahlen zu definieren, die in Zukunft
bei einem verbesserten Projektablauf erfüllt und gemessen werden sollen.
Daraus erarbeite ich Varianten des neuen Projektablaufes und versuche auch
diese so zu bewerten, dass die allink GmbH als mein Auftraggeber einen
Entscheid fällen kann, welchen Projektablauf man in Zukunft einsetzen und 
verfeinern möchte. Der neue Projektablauf soll abschliessend in einem ``Proof of Concept'', also
anhand eines konkreten Projektes, getestet werden. Natürlich wird ein einziger
``Proof of Concept'' nicht ausreichen um vollständig sicherzustellen, dass der
neue Projektablauf optimal ist. Dies wird sich aber dann im Laufe der Zeit zeigen.

Für mich ist das Ziel meiner Diplomarbeit erreicht, wenn ich unserer Firma anhand der Analyse und
dem Aufzeigen von Problemen und möglichen Lösungen helfen könnte, den täglichen
Ablauf unserer Projekte für alle Parteien angenehmer und effizienter zu gestalten.
Das beinhaltet natürlich auch alle erwarteten Resultate\footnote{Die erwarteten 
Resultate sind in der Aufgabenstellung im Anhang \ref{chap:aufgabenstellung} ersichtlich.}, die ich mir in meiner
Aufgabenstellung der Diplomarbeit gestellt habe.
  
  \cleardoublepage
  
  \chapter{Aufgabenstellung}
  \section{Ausgangslage}
Die Agentur allink.creative ist im letzten Jahr stark gewachsen. Von zehn
Mitarbeitern im Februar 2010 auf siebzehn Mitarbeiter im Februar 2011. Dies hat 
zur Auswirkung, dass gewisse Funktionen und Prozesse neu definiert und bestehende
überarbeitet werden müssen, um weiterhin effizient, oder wenn möglich noch 
effizienter, arbeiten zu können. Die Agentur arbeitet überwiegend mit Apple
Computern und setzt gewisse Software ein, die die Geschäftsleitung beibehalten 
möchte. Die konkreten Vorstellungen und Vorgaben müssen in dieser Arbeit erfasst 
werden.

\section{Ziel der Arbeit}
Bereiche wie die Stundenrapportierung, die Projektplanung und das Projektcontrolling 
können mit Hilfe von IT-Lösungen massgebend optimiert und vereinfacht werden. 
In dieser Arbeit sollen die Herausforderungen, die der Auftraggeber in der 
Planung und im Controlling eines Projektes zu bewältigen hat, erfasst und 
Lösungsvorschläge evaluiert werden. Der Fokus liegt dabei auf der besseren 
Messbarkeit des finanziellen Erfolges eines Projektes und des gesamten 
Unternehmens.

Die Arbeit grenzt sich ganz klar von der Finanzbuchhaltung ab, da
diese keinen direkten Einfluss auf die Projekte hat und bei allink.creative 
bei einen Treuhänder ausgelagert wurde.

\section{Aufgabenstellung}
Folgende Aufgaben soll der Studierende während dieser Arbeit bewältigen:

\begin{itemize}
    \item Ist-Situation im Bereich Projektablauf der allink.creative erfassen
    \item Kennzahlen definieren, die in Zukunft auf Projektebene gemessen 
        werden sollen
    \item Eine Recherche der Prozesse in ähnlich funktionierenden KMUs durchführen
    \item Neue Prozesse definieren und bestehende, sofern sinnvoll, überarbeiten
    \item Evaluation von IT-Lösungen, die diese Prozesse möglichst passend 
        für den Auftraggeber abbilden und die definierten Kennzahlen generieren können
\end{itemize}

\section{Erwartete Resultate}
Der Studierende soll dem Auftraggeber ein Dokument erstellen, das folgende 
Punkte beinhaltet: 

\begin{itemize}
    \item Beschreibung der Ist-Situation im Bereich Projektablauf
    \item Übersicht der bestehenden Software beim Auftraggeber
    \item Kennzahlen, die auf Projektebene gemessen werden sollen
    \item Darstellung der neuen und überarbeiteten Prozesse
    \item Übersicht der bestehenden Software in der neuen Prozesslandschaft
    \item Softwareempfehlungen für die komplette Prozessabbildung
    \item ``Make or Buy''-Entscheid mit dem Auftraggeber
\end{itemize}

Während der Arbeit sollen als ``Proof of Concept'' die selben zur Zeit verwendeten
Prozesse und Tools verwendet werden. Sobald die neuen Prozesse und Tools
definiert sind und der Auftraggeber einen Entscheid gefällt hat, sollen die
Informationen ebenfalls in die neue Prozess- und Toollandschaft übertragen werden.

\section{Abgrenzung}
Folgende Punkte werden abgegrenzt, da sie den Rahmen der Arbeit Überschreiten 
würden:

\begin{itemize}
    \item Die Analysen beschränken sich auf Recherchen im Internet und Büchern
    \item Umfragen, Erhebungen sowie Feldstudien werden nicht durchgeführt
\end{itemize}


  \cleardoublepage
  
  \chapter{Erreichte Ziele}
  In der nachfolgenden Tabelle \ref{tab:erreichte_ziele} sind alle Ziele gemäss 
den erwarteten Resultaten der Aufgabenstellung aufgelistet. Alle 
erwarteten Ziele wurden erreicht.

\begin{table}[h]
\begin{center}
    \begin{tabular}{p{11cm}ll}
        \toprule \textbf{Ziel} & \textbf{Stand} \\
        \midrule Die IST-Situation im Bereich Projektablauf der allink.creative
            ist beschrieben und abgebildet. & noch nicht erfüllt \\
        \midrule Eine Übersicht über die bestehende Software bei allink.creative
            wurde erstellt. & noch nicht erfüllt \\
        \midrule Die Kennzahlen, die auf Projektebenen gemessen werden sollen,
            wurden definiert. & noch nicht erfüllt \\
        \midrule Die neuen und überarbeiteten Prozesse sind beschrieben und
            abgebildet. & noch nicht erfüllt \\
        \midrule Eine Übersicht über die bestehende Software bei allink.creative
            in der neuen Prozesslandschaft wurde erstellt. & noch nicht erfüllt \\
        \midrule Eine Softwareempfehlung für die komplette Prozessabbildung
            wurde erstellt und begründet. & noch nicht erfüllt \\
        \midrule Ein ``Make or Buy''-Entscheid wurde mit dem Auftraggeber 
            getroffen und festgehalten. & noch nicht erfüllt \\
        \bottomrule
    \end{tabular}
    \caption{Auflistung der erwarteten Resultate mit Stand der Erfüllung}
    \label{tab:erreichte_ziele}
\end{center}
\end{table}

  \cleardoublepage
  
  \chapter{Planung}
  \section{Detailanalyse der Aufgabenstellung}
  \section{Aufwandschätzung}
  \section{Projektplan}
  \section{Termine}
  
  \chapter{Studie}
  \section{Heutige Prozesslandschaft}
  \section{Aktueller Projektablauf}
  \section{Verwendete Software}
  \section{Andere Unternehmungen}
  \subsection{Unternehmen XY}
  \subsubsection{Projektablauf}
  \subsubsection{Software}
  
  \chapter{Analyse}
  \section{Grundlagen eines Projektablaufes}
  \subsection{Vorgehen}
  \subsection{Darstellung}
  \section{Mögliche Kennzahlen}
  \subsection{Vorgehen}
  \subsection{Berechnung}
  \subsection{Abhängigkeiten}
  
  \chapter{Resultat}
  \section{Neue Prozesslandschaft}
  \subsection{Projektablauf}
  \subsection{Kennzahlen}
  \subsection{Software}
  \subsubsection{Bestehende Software}
  \subsubsection{Softwareempfehlung}
  \section{Entscheidung}
  \subsection{Prozesslandschaft}
  \subsection{Software}
  
  \chapter{Proof of Concept}
  \section{Alter Projektablauf}
  \section{Neuer Projektablauf}
  \section{Gegenüberstellung}
  \subsection{Vorteile}
  \subsection{Nachteile}
  
  \chapter{Reflektion}
  \section{Fazit}
  \section{Ausblick}
  
  \appendix
  
  \chapter{Personalienblatt}
  \begin{tabbing}
	\hspace*{4cm}   \= \kill
	Name, Vorname:  \> {\bf Spross, Silvan} \\
	Adresse:        \> {\bf Meinrad Lienert-Strasse 27} \\
	PLZ, Wohnort:   \> {\bf 8003 Zürich} \\
	\\
	Geburtsdatum:   \> {\bf 07.11.1985} \\
	Heimatort:      \> {\bf Zürich ZH} \\
\end{tabbing}

  
  \chapter{Bestätigung}
  Hiermit bestätige ich, Silvan Spross, dass die vorliegende Diplomarbeit 
``Definition und Optimierung der Projektprozesse bei allink.creative'' im
Rahmen der geltenden Reglemente und in allen Teilen selbständig erarbeitet und 
durchgeführt wurde.\\
\\
Zürich, den 31. Mai 2011\\
\\\\
Silvan Spross
  
  \listoffigures
  \listoftables
  %\lstlistoflistings
  
  \bibliographystyle{alpha}
  \bibliography{literaturverzeichnis}
  
  \chapter{Protokolle}
  \section{Kick-Off Protokoll}
  \section{Design Review Protokoll}
  \section{Arbeitsprotokoll}

\end{document}