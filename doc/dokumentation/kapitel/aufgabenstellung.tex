\section{Ausgangslage}
Die Agentur allink.creative ist im letzten Jahr stark gewachsen. Von zehn
Mitarbeitern im Februar 2010 auf siebzehn Mitarbeiter im Februar 2011. Dies hat 
zur Auswirkung, dass gewisse Funktionen und Prozesse neu definiert und bestehende
überarbeitet werden müssen, um weiterhin effizient, oder wenn möglich noch 
effizienter, arbeiten zu können. Die Agentur arbeitet überwiegend mit Apple
Computern und setzt gewisse Software ein, die die Geschäftsleitung beibehalten 
möchte. Die konkreten Vorstellungen und Vorgaben müssen von dem Studierenden
in der Arbeit erfasst werden.

\section{Ziel der Arbeit}
Bereiche wie die Stundenrapportierung, die Projektplanung und das Projektcontrolling 
können mit Hilfe von IT-Lösungen massgebend optimiert und vereinfacht werden. 
In dieser Arbeit sollen die Herausforderungen, die der Auftraggeber in der 
Planung und im Controlling eines Projektes zu bewältigen hat, erfasst und 
Lösungsvorschläge evaluiert werden. Der Fokus liegt dabei auf der besseren 
Messbarkeit des finanziellen Erfolges eines Projektes und des gesamten 
Unternehmens.

Die Arbeit grenzt sich ganz klar von der Finanzbuchhaltung ab, da
diese keinen direkten Einfluss auf die Projekte hat und bei allink.creative 
bei einen Treuhänder ausgelagert wurde.

\section{Aufgabenstellung}
Folgende Aufgaben soll der Studierende während dieser Arbeit bewältigen:

\begin{itemize}
    \item IST-Situation im Bereich Projektablauf der allink.creative erfassen
    \item Kennzahlen definieren, die in Zukunft auf Projektebene gemessen 
        werden sollen
    \item Eine Recherche der Prozesse in ähnlich funktionierenden KMUs durchführen
    \item Neue Prozesse definieren und bestehende, sofern sinnvoll, überarbeiten
    \item Evaluation von IT-Lösungen, die diese Prozesse möglichst passend 
        für den Auftraggeber abbilden und die definierten Kennzahlen generieren können
\end{itemize}

\section{Erwartete Resultate}
Der Studierende soll dem Auftraggeber ein Dokument erstellen, das folgende 
Punkte beinhaltet: 

\begin{itemize}
    \item Beschreibung der IST-Situation im Bereich Projektablauf
    \item Übersicht der bestehenden Software beim Auftraggeber
    \item Kennzahlen, die auf Projektebene gemessen werden sollen
    \item Darstellung der neuen und überarbeiteten Prozesse
    \item Übersicht der bestehenden Software in der neuen Prozesslandschaft
    \item Softwareempfehlungen für die komplette Prozessabbildung
    \item ``Make or Buy''-Entscheid mit dem Auftraggeber
\end{itemize}

Während der Arbeit sollen als ``Proof of Concept'' die selben zur Zeit verwendeten
Prozesse und Tools verwendet werden. Sobald die neuen Prozesse und Tools
definiert sind und der Auftraggeber einen Entscheid gefällt hat, sollen die
Informationen ebenfalls in die neue Prozess- und Toollandschaft übertragen werden.

\section{Abgrenzung}
tbd

