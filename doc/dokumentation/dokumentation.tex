%  --------------------------------------------------------------------------
%  Diplomarbeit Dokumentation
%  Created by Silvan Spross on 2011-04-02.
%  --------------------------------------------------------------------------

%  --------------------------------------------------------------------------
%  Latex Document Settings
%  --------------------------------------------------------------------------
\documentclass[
11pt, % Schriftgrösse
a4paper, % A4 Papier
BCOR10mm, % Absoluter Wert der Bindekorrektur, z.B. BCOR1cm
DIV14, % Satzspiegel festlegen siehe
       % http://www.ctex.org/documents/packages/nonstd/koma-script.pdf
footsepline = false, % Trennlinie zwischen Textkörper und Fußzeile
                     % bei normalen Seiten
headsepline, % Trennlinie zwischen Kopfzeile und Textkörper
             % bei normalen Seiten
oneside, % Zweiseitig
openright,
halfparskip, % Europäischer Satz mit Abstand zwischen den Absätzen
abstracton, % inkl. Abstract
listof=totocnumbered, % Abb.- und Tab.verzeichnis im Inhaltsverzeichnis
bibliography=totocnumbered % Lit.zeichnis in Inhaltsverzeichnis aufnehmen
]{scrreprt}

\usepackage[automark]{scrpage2} % Gestaltung von kopf- und Fußzeile
\usepackage[ngerman]{babel}
\usepackage[ngerman]{translator}
\usepackage{tocbasic}
\usepackage[utf8]{inputenc}
\usepackage{lmodern} % Latin Modern
\usepackage[T1]{fontenc}
\usepackage{hyphenat}
\usepackage{ae} % Schöne Schriften für PDF-Dateien

% Tradmark
\def\TTra{\textsuperscript{\texttrademark}}

%1.5 Zeilenabstand
\usepackage[onehalfspacing]{setspace}

% Festlegung des Seitenstils (scrpage2)
\pagestyle{scrheadings}
\clearscrheadfoot
\automark[chapter]{section}

% \lehead{\sffamily\upshape\headmark}
% \cehead{}
% \rehead{}
% \lefoot[\pagemark]{\upshape \pagemark}
% \cefoot{}
% \refoot{}
% \lohead{}
% \cohead{}
\lohead{\sffamily\upshape\headmark}
\lofoot{}
\cofoot{}
\rofoot[\pagemark]{\scshape \pagemark}

% Surround parts of graphics with box
\usepackage{boxedminipage}

% Package for including code in the document
\usepackage{listings}

% If you want to generate a toc for each chapter (use with book)
\usepackage{minitoc}
\usepackage{longtable}

% Abkürzungsverzeichnis erstellen.
\usepackage[printonlyused]{acronym}

% schöne Tabelle zeichnen
\usepackage{booktabs}
\renewcommand{\arraystretch}{1.4} %Die Zeilenabstände in Tabllen angepasst.

% für variable Breiten
\usepackage{tabularx}

% Durchgestrichener Text
\usepackage[normalem]{ulem} %emphasize weiterhin kursiv

% This is now the recommended way for checking for PDFLaTeX:
\usepackage{ifpdf}

\usepackage{eurosym}

\usepackage{natbib}

\usepackage{paralist}

\usepackage[hyperfootnotes=false]{hyperref}
\hypersetup{
  bookmarks=true,         % show bookmarks bar?
  unicode=true,           % non-Latin characters in Acrobat’s bookmarks
  pdftoolbar=true,        % show Acrobat’s toolbar?
  pdfmenubar=true,        % show Acrobat’s menu?
  pdffitwindow=true,      % window fit to page when opened
  pdfstartview={FitH},    % fits the width of the page to the window
  pdftitle={Diplomarbeit},   
  pdfauthor={Silvan Spross},
  pdfsubject={Definition und Optimierung der Projektprozesse bei allink.creative},
  pdfcreator={TeX Live 2009},
  pdfproducer={pdfTeX, Version 3.1415926-1.40.10},
  pdfnewwindow=true,      % links in new window
  colorlinks=true,       % false: boxed links; true: colored links
  linkcolor=blue,          % color of internal links
  citecolor=black,        % color of links to bibliography
  filecolor=magenta,      % color of file links
  urlcolor=cyan          % color of external links
  % linkcolor=black,          % color of internal links
  % citecolor=black,        % color of links to bibliography
  % filecolor=black,      % color of file links
  % urlcolor=black          % color of external links
}

\ifpdf
    \usepackage[pdftex]{graphicx}
\else
    \usepackage{graphicx}
\fi

\makeatletter 
\let\orgdescriptionlabel\descriptionlabel 
\renewcommand*{\descriptionlabel}[1]{% 
  \let\orglabel\label 
  \let\label\@gobble 
  \phantomsection 
  \edef\@currentlabel{#1}% 
  %\edef\@currentlabelname{#1}% 
  \let\label\orglabel 
  \orgdescriptionlabel{#1}% 
} 
\makeatother 

%  --------------------------------------------------------------------------
%  Start Document
%  --------------------------------------------------------------------------
\title{Definition und Optimierung der Projektprozesse bei allink.creative}

\author{Diplomarbeit in Informatik\\
    \\
    Studierender - Silvan Spross\\
	Auftraggeber - Michael Walder\\
    Projektbetreuer - Beat Seeliger\\
    Experte - Matthias Bachmann\\
	\\
	HSZ-T - Technische Hochschule Zürich}

\date{März 2011 bis Juni 2011}

\begin{document}

  \ifpdf
    \DeclareGraphicsExtensions{.pdf, .jpg, .tif}
  \else
    \DeclareGraphicsExtensions{.eps, .jpg}
  \fi

  \pagenumbering{Alph}
  
  \maketitle

  \begin{abstract}
Die Designagentur allink.creative stellt sich wegen ihres Wachstums neuen 
Herausforderungen in ihren Projektabläufen und Prozessen. Da sich die
Agentur bis anhin noch nicht mit diesem Thema explizit auseinander gesetzt hat,
wird im Rahmen dieser Arbeit der heutigen Zustand genauer analysiert und 
zusammen mit den Bedürfnissen und Anforderungen erfasst.
Anhand den gewonnenen Informationen und den vorhandenen Theorien wird ein Konzept
mit praxisnahen Lösungsvarianten aufgezeigt. Die gewählte
Variante wird in einem ``Proof of Concept'' getestet und aus den daraus
entstehenden Resultaten werden Schlüsse für die zukünftige Vorgehensweise 
gezogen. Über die ganze Arbeit hinweg stehen auch die zur Zeit und in Zukunft zu verwendenden
Tools und Software zur Unterstützung im Vordergrund.
\end{abstract}

  \pagenumbering{Roman}
  
  \tableofcontents
  
  \chapter{Personalienblatt}
  \input{./kapitel/01_00_personalienblatt.tex}
  
  \chapter{Bestätigung}
  Hiermit bestätige ich, Silvan Spross, dass die vorliegende Diplomarbeit 
``Definition und Optimierung der Projektprozesse bei allink.creative'' im
Rahmen der geltenden Reglemente und in allen Teilen selbständig erarbeitet und 
durchgeführt wurde.\\
\\
Zürich, den 31. Mai 2011\\
\\\\
Silvan Spross
  
  \chapter{Vorwort des Authors}
  Das Schreiben einer Diplomarbeit bedeutet nebst dem baldigen Abschluss des Studiums
auch einen grossen Aufwand. Deshalb ist die Wahl des Themas und meine
Motivation in diese Arbeit die nötige Zeit und Qualität zu investieren sehr
wichtig. Aus diesem Grund möchte ich kurz erläutern, was mich zu dieser
Arbeit geführt hat.

Ein eigenes Unternehmen zu gründen und erfolgreich zu führen war schon immer
einer meiner Träume. Nach meiner Lehre im Jahre 2004 machte ich mich deshalb
selbstständig und gründete 2005 die SiSprocom GmbH\footnote{Website der SiSprocom GmbH, \url{http://sisprocom.ch/}}. 
Im ersten Jahr bestanden
die Tätigkeitsfelder überwiegend aus Webdesign und Schulungen. Im Bereich
Webdesign lag der Fokus hauptsächlich auf der Programmierung. Dieser Bereich 
entwickelte sich immer stärker in Richtung 
Applikationsentwicklung und dank Aufträgen einer Zürcher Grossbank 
lag der Fokus bald nur noch darauf.

Durch dieses grössere Mandat wuchs die SiSprocom GmbH zwischenzeitlich
auf 3 Mitarbeiter. Dies führte zu massiv mehr Aufwand in der Administration.
Schnell wurde klar, dass zwingend Stunden rapportiert und das Schreiben
von Rechnungen vereinfacht werden musste. Wir bedienten uns damals Google 
Docs\footnote{Googles Online Office Suite, \url{http://docs.google.com/}} um die rapportierten Stunden zentral zu 
verwalten und einer selbst geschriebenen Software um vereinfacht Rechnungen 
verwalten und schreiben zu können.
So lehrreich und spannend die Arbeit in dieser Grossbank auch war, so kompliziert
waren auch ihre Abläufe und Prozesse. Zu diesem Zeitpunkt schwor ich mir, dies
in meiner Unternehmung einmal besser zu lösen.

Ende 2009 realisierte ich, dass ich zwar ein eigenes Unternehmen hatte und
selbständig war, jedoch fast ausschliesslich nur für einen Kunden arbeitete.
Ich kam zum Schluss, dass dies nicht das eigentliches Ziel meiner Selbständigkeit 
war und begriff
relativ schnell, dass ich mich von dieser Abhängigkeit nur lösen konnte, wenn
ich mich als Person vollständig aus diesem Mandat zurückziehe.
Dies tat ich dann anfangs 2010 auch und mietete ein Büro in den Räumlichkeiten
der allink GmbH\footnote{Website der allink GmbH, \url{http://allink.ch/}}. Deren IT hatte zu diesem Zeitpunkt 
ein paar grössere
Herausforderungen zu bewältigen und ich bot meine Hilfe an. Schnell wurde
daraus eine Partnerschaft und die SiSprocom fusionierte mit der allink, indem zwischen
den Partnern Anteile der jeweiligen Firma ausgetauscht wurden.

Heute, im März 2011, knapp ein Jahr danach, ist Die SiSprocom GmbH um 30\% 
gewachsen und arbeitet ausschliesslich für die erwähnte Grossbank. Alle anderen Projekte
haben wir in die allink GmbH übernommen. Die administrativen Aufgaben der
SiSprocom GmbH wurden an
meinem Vater abgetreten, der zusammen mit einem Treuhänder das laufende
Mandat und die drei Mitarbeiter gut handhaben kann. So konnte ich mich ganz
auf die neuen Herausforderungen bei der allink konzentrieren.

Die allink GmbH ist inzwischen um 70\% auf 17 Mitarbeiter gewachsen und muss sich dadurch
neuen Herausforderungen in der Organisation und bei den Projektabläufen stellen.

Da mich die Selbständigkeit auch durch das ganze Studium begleitet hat, möchte 
ich meine Diplomarbeit nutzen um unser Unternehmen zu optimieren.
  
  \chapter{Einleitung}
  \pagenumbering{arabic}
  \section{Ausgangslage}
Die Agentur allink.creative ist im letzten Jahr stark gewachsen. Von zehn
Mitarbeitern im Februar 2010 auf siebzehn Mitarbeiter im Februar 2011. Dies hat 
zur Auswirkung, dass gewisse Abläufe und Prozesse neu definiert und bestehende
überarbeitet werden müssen, um weiterhin effizient, oder wenn möglich noch 
effizienter, arbeiten zu können. Die Agentur arbeitet zurzeit überwiegend mit Apple
Computern und setzt gewisse Software ein, die die Geschäftsleitung beibehalten 
möchte. Es soll jedoch innerhalb dieser Arbeit geprüft werden, welche Software
weiterhin Sinn macht und welche man möglicherweise ersetzen oder neu anschaffen
bzw. sogar selbst entwickeln möchte.

\section{Problemstellung}
Durch den schnellen Wachstum der Agentur stösst sie bei der Abwicklung ihrer
Projekte an Grenzen. Die Partner, die bisher den Überblick über alle
Projekte und deren Abläufe im Auge behalten konnten, sind bei dieser Grösse
nicht mehr in der Lage dies beizubehalten. Deshalb muss mehr Struktur geschaffen und
den Mitarbeitern mehr Verantwortung und Kompetenzen abgegeben werden. Und trotzdem
soll es der Geschäftsleitung mit Hilfe von Controling Tools möglich bleiben,
einen Überblick über die Lage der Agentur zu behalten.

Die Partner erhoffen sich dadurch ein gesundes Wachstum der Agentur ermöglichen
zu können. Zusätzlich wird vermutet, dass durch Optimierungen im Projektablauf
auch Kosten gespart werden können und die ganze Agentur im Allgemeinen 
professionalisiert werden kann.

\section{Zielsetzung}
Das Hauptziel dieser Arbeit besteht darin, das aktuelle Projektmanagement der 
allink zu analysieren, den zur Zeit gegebenen Umständen anzupassen und zu
optimieren, damit ein weiteres Wachstum der Agentur vereinfacht wird.

Allen Partnern bei der allink ist klar, dass der heutiger Projektablauf 
nicht optimal ist. Zu oft sehen sie sich mit gleichen Problemen konfrontiert, 
die in anderen Projekt schon einmal angetroffen und gelöst wurden.
Jedes Mal wird versucht daraus zu lernen, ohne etwas konkret festzuhalten oder
wirklich zu verändern. Das liegt meist daran, dass zu viel ansteht und
die internen Verbesserungen hinter die Aufträge und Wünsche der Kunden
gestellt werden.

Der aktuellen Projektablauf der allink wird genauer untersucht und auf 
dessen Stärken und Schwächen eingegangen. Danach werden die Bedürfnisse und 
Anforderungen der verschiedenen Stakeholdern des Projektablaufes aufgenommen und
Kennzahlen definiert, die in Zukunft bei einem verbesserten Projektablauf gemessen 
werden sollen. Daraus wird ein neues Konzept des überarbeiteten Projektablaufes 
mit verschiedenen Varianten in der Umsetzung erstellt und zusammen mit der 
Geschäftsleitung der allink entschieden, welchen Projektablauf 
man in Zukunft einsetzen und verfeinern möchte. Der neue Projektablauf soll 
abschliessend in einem ``Proof of Concept'', also anhand eines konkreten Projektes, 
getestet werden.

\section{Aufbau der Arbeit}
Im Kapitel \ref{chap:theorie_teil} werden dem Leser die Grundlagen zu Projektmanagement 
näher gebracht. Neben den jeweiligen Begriffserklärungen werden wichtige Modelle 
und Tools aus der Theorie und Praxis aufgezeigt.

Im Kapitel \ref{chap:analyse} wird der aktuelle Projektablauf bei der allink
GmbH genauer analysiert und aufgezeigt. Mit Hilfe der Prozessdarstellung wird
das aktuelle Vorgehen dargestellt und mit Beispielen untermalt. In einem
Branchenvergleich werden ähnlich grosse Agentur interviewt und deren Vorgehen
als Vergleich herangezogen.

Im Kapitel \ref{chap:anforderungen} werden die Bedürfnisse und Anforderungen 
aller Stakeholders die in den Projektablauf involviert sind aufgenommen und daraus
Anforderungen gebildet. Zusätzlich werden Kennzahlen definiert, die in Zukunft
während und nach den Projekten gemessen werden sollen.

Im Kapitel \ref{chap:konzept} wird ein Konzept des neuen Projektablaufes
erarbeitet und verschieden Varianten zur Umsetzung aufgezeigt. Auf dieser Grundlage
entscheidet sich der Auftraggeber für einen neuen Projektablauf. Dieser wird
im Kapitel \ref{chap:proof_of_concept} der Arbeit in einem ``Proof of Concept'' 
getestet.

Im letzten Kapitel \ref{chap:reflektion} wird die Arbeit in Form eines Fazits 
zusammengefasst. Zudem wird ein Ausblick auf die zukünftige Verwendung und den 
Einsatz des neuen Projektablaufes abgegeben.

Die folgende Abbildung \ref{pic:01_gliederung_arbeit} beschreibt die Gliederung der 
Arbeit in graphischer Form und zeigt auf, wie die erarbeiteten Resultate in
das Konzept des neuen Projektablaufes einfliessen.

\begin{figure}[htbp]
\begin{center}
\includegraphics[width=0.6\textwidth,angle=0]{./bilder/einleitung/01_gliederung_arbeit.pdf}
\caption[]{Aufbau der Diplomarbeit\footnotemark}
\label{pic:01_gliederung_arbeit}
\end{center}
\end{figure}
\footnotetext{Eigene Darstellung}

% \section{Methodische Vorgehensweise}

\section{Inhaltliche Schwerpunkte}
Die inhaltlichen Schwerpunkte dieser Arbeit liegen in den folgenden Bereichen:

\begin{itemize}
    \item Einarbeitung Theorie Projektmanagement und Business Reengineering
    \item Analyse des aktuellen Projektablaufes der allink GmbH
    \begin{itemize}
        \item Darstellung des Projektablaufes
        \item Aufzeigen der Stärken und Schwächen
        \item Zurzeit verwendete Software und Hilfsmittel
    \end{itemize}
    \item Variantenbildung neuer Projektabläufe
    \begin{itemize}
        \item Erarbeitung von praxisnahen Varianten
        \item Bewertung und Beurteilung der möglichen Varianten
    \end{itemize}
    \item Überprüfung und Test des neu gewählten Projektablaufes
    \item Fazit und Reflektion der Arbeit
\end{itemize}

Die Arbeit soll dem Leser und anderen Agenturen in einer ähnlichen Situation
aufzeigen, was für Herausforderungen bei einem Wachstum entstehen und wie
sie mit Hilfe von Veränderungen und Anpassungen bewältigt werden können.

\section{Inhaltliche Ein- und Abgrenzung}
Die Arbeit fokussiert sich auf eine Agentur. Es wird vertieft auf deren Probleme
und Herausforderungen eingegangen. Deshalb sind die Schlüsse die darin gezogen
werden nicht grundsätzlich für jede Agentur anwendbar. Es wird jedoch versucht, das
ganze so global wie möglich zu betrachten. Da zum Schluss aber eine für die
Agentur in der Praxis anwendbare neue Lösung gesucht wird, passt diese wohl
kaum in jede Firmenkultur.

Zusätzlich grenzt sich die Arbeit von folgenden Punkten klar ab:

\begin{itemize}
    \item Die Analysen beschränken sich auf Recherchen im Internet und Büchern.
    \item Umfragen, Erhebungen sowie Feldstudien werden nur begrenzt im Rahmen
        von Interviews durchgeführt.
    \item Die definierten zu messenden Kennzahlen können in den Bereich der Betriebswirtschaft
        und Buchhaltung fallen. Es wird aber nicht näher auf deren Theorien eingegangen.
\end{itemize}

  
  
  \chapter{Theoretische Grundlagen Projektmanagement}\label{chap:theorie_teil}
  \section{Projektmanagement - Begriff}
Für das gemeinsame Verständnis des Begriffes ``Projektmanagement'' wird zuerst
eine Definition des Wortes gemacht. Diese Arbeit lehnt sich an diese 
Begriffsdefinition an. Danach wird näher auf den Aufbau eines
Projektablaufes eingegangen.

Im Rahmen eines Projektmanagement werden die diversen Aufgaben ganzheitlich in
einem Projekt eingebettet und unter Berücksichtigung der Parameter Kosten, Termine
und Qualität geplant und durchgeführt.\footnote{\citealp*[Vgl.][S. 9]{burghardt2007einfuehrung}}
Die Stiftung für Forschung und Beratung am Betriebswissenschaftlichen Institut 
(BWI) der ETH Zürich definiert den Begriff Projektmanagement wie folgt:

\begin{quote}
``Projektmanagement wird als Überbegriff aller planenden, überwachenden,
koordinierenden und steuernden Massnahmen verstanden, die für die Um- oder
Neugestaltung von Systemen (resp. Problemlösungen) erforderlich sind.''\footnote{\citealp*[S. 1.1]{stiftung1998projekt}}
\end{quote}

\section{Projektablauf}
Der Projektablauf gehört zu den Methoden und Techniken der Organisation. Als
Projekt bezeichnet man ein Vorhaben, das einmalig ist und einen definierten
Start- und Endtermin hat. Im Projektablauf regelt man die Ablauforganisation
eines Projektes, also welche Aufgaben wann zu erledigen sind.\footnote{\citealp*[Vgl.][S. 136]{schmidt2002einfuehrung}}
Das Projektmanagement mit dem Projektablauf als Methode umfasst alle Aktivitäten,
die für eine vollständige Abwicklung eines Projektes erforderlich sind.\footnote{\citealp*[Vgl.][S. 11]{burghardt2007einfuehrung}}

Ein Projektablauf unterteilt man in vier Hauptabschnitte, die in der Abbildung \ref{pic:01_hauptabschnitte}
dargestellten sind.

\clearpage

\begin{figure}[htbp]
\begin{center}
\includegraphics[width=0.85\textwidth,angle=0]{./bilder/theorie/01_hauptabschnitte.pdf}
\caption[]{Vier Hauptabschnitte eines Projektablaufes\footnotemark}
\label{pic:01_hauptabschnitte}
\end{center}
\end{figure}
\footnotetext{Eigene Darstellung in Anlehnung an \citealp*[Bild 1.2]{burghardt2007einfuehrung}}

\subsection{Projektdefinition}
Die Projektdefinition besteht aus der eigentlichen Gründung eines Projektes,
der Definition des Projektziels und die Organisation des Projektes.

Zu Beginn eines Projektes steht der Projektantrag. Er beinhaltetet alle relevanten
Informationen wie eine Aufgabenbeschreibung und Termine. Wird der Projektantrag
angenommen, wandelt sich der Antrag in einen Projektauftrag um.\footnote{\citealp*[Vgl.][S. 13]{burghardt2007einfuehrung}}
Als nächstes muss ein eindeutiges und vollständiges Projektziel definiert werden.
Dies geschieht meist zusammen mit dem Auftraggeber anhand eines Anforderungskatalogs
oder Pflichtenhefts.

Es empfiehlt sich eine Problemfeldanalyse und eine Wirtschaftlichkeitsbetrachtung
zu machen. Ohne genauere Kenntnisse zur Wirtschaftlichkeit eines Projektes sollte
keines begonnen werden.\footnote{\citealp*[Vgl.][S. 45]{burghardt2007einfuehrung}}
Unter einer Problemfeldanalyse versteht man eine klare Definition des eigentlichen
Problems. Wobei das Problem eher als Herausforderung zu betrachten ist. Man
stellt sich dabei folgende Fragen:

\begin{itemize}
    \item Was ist das Problem bzw. die Herausforderung?
    \item Wer sind die Betroffenen und die Beteiligten?
    \item Was sind deren wichtigste Ziele?
\end{itemize}

Bei der Überprüfung der Wirtschaftlichkeit eines Projektes kann dies im Sinne
des Endproduktes oder des Auftrages geschehen. Im Ersteren versucht man zu
errechnen, ob das Endprodukt einen wirtschaftlichen Nutzen für den Auftraggeber
darstellt. Dies sollte der Auftraggeber normalerweise schon selbst vorgenommen
haben. Für die Projektleitung steht schlussendlich der wirtschaftliche Nutzen
des Auftrages für die eigene Unternehmung im Vordergrund. Kann der Auftrag so
offeriert und durchgeführt werden, dass das Projekt am Ende profitabel ist.\footnote{\citealp*[Vgl.][S. 4]{prueter2007multi}}

Ist dies zutreffend, müssen die organisatorischen Voraussetzungen für das Projekt geschaffen werden,
indem der Projektleiter ernannt und eine passende Projektorganisation gewählt wird.
Die Definition einer Projektorganisation lautet nach DIN 69901:

\begin{quote}
``Gesamtheit der Organisationseinheiten und der aufbau- und ablauforganisatorischen
Regelungen zur Abwicklung eines bestimmten Projektes.''\footnote{Vgl. DIN 69901}
\end{quote}

Je nach Bereichsüberschreitung innerhalb des Unternehmens und der einzubindenden
Projektmitarbeiter sowie der Bedeutung und der Grösse des Projektes kann zwischen
fünf Formen von Projektorganisationen unterschieden werden:\footnote{\citealp*[Vgl.][S. 56]{burghardt2007einfuehrung}}

\begin{itemize}
    \item Reine Projektorganisation
    \item Einfluss-Projektorganisation
    \item Matrix-Projektorganisation
    \item Auftrags-Projektorganisation
    \item Projektmanagement in der Linie
\end{itemize}

In der reinen Projektorganisation wir in der bestehenden Unternehmensstruktur
eine zusätzliche Stelle für das Projekt geschaffen und alle beteiligten 
Projektmitarbeiter unter einem Projektleiter, der die Linienautorität hat, 
zusammengefasst.\footnote{\citealp*[Vgl.][S. 56]{burghardt2007einfuehrung}}
In der Grafik \ref{pic:05_projektorganisationen_reine} ist 
ein Beispiel einer reinen Projektorganisation abgebildet.

\begin{figure}[htbp]
\begin{center}
\includegraphics[width=0.5\textwidth,angle=0]{./bilder/theorie/05_projektorganisationen_reine.pdf}
\caption[]{Reine Projektorganisation\footnotemark}
\label{pic:05_projektorganisationen_reine}
\end{center}
\end{figure}
\footnotetext{Eigene Darstellung in Anlehnung an \citealp*[Bild 2.9]{burghardt2007einfuehrung}}

In einer Einlfuss-Projektorganisation gibt es anstelle einem Projektleiter
einen Projektkoordinator, der als Stabsstelle eingefügt. Er hat jedoch kaum
Kompetenzen und kann nur koordinierend und lenkend wirken.\footnote{\citealp*[Vgl.][S. 57]{burghardt2007einfuehrung}}
In der Grafik \ref{pic:05_projektorganisationen_einfluss}
ist ein Beispiel einer Einfluss-Projektorganisation abgebildet.

\clearpage

\begin{figure}[htbp]
\begin{center}
\includegraphics[width=0.45\textwidth,angle=0]{./bilder/theorie/05_projektorganisationen_einfluss.pdf}
\caption[]{Einfluss-Projektorganisation\footnotemark}
\label{pic:05_projektorganisationen_einfluss}
\end{center}
\end{figure}
\footnotetext{Eigene Darstellung in Anlehnung an \citealp*[Bild 2.10]{burghardt2007einfuehrung}}

In der Matrix-Projektorganisation trägt der Projektleiter zwar die ganze Verantwortung
des Projektes, hat aber nicht die ganze Weisungsbefugnis für alle beteiligten Mitarbeiter.
Die fachliche Weisungsbefugnis unterliegt zwar dem Projektleiter, die disziplinarische
bleibt jedoch weiterhin beim direkten Vorgesetzten in der Linienorganisation.\footnote{\citealp*[Vgl.][S. 58]{burghardt2007einfuehrung}}
In der Grafik \ref{pic:05_projektorganisationen_matrix} ist ein Beispiel einer 
Marix-Projektorganisation abgebildet.

\begin{figure}[htbp]
\begin{center}
\includegraphics[width=0.5\textwidth,angle=0]{./bilder/theorie/05_projektorganisationen_matrix.pdf}
\caption[]{Matrix-Projektorganisation\footnotemark}
\label{pic:05_projektorganisationen_matrix}
\end{center}
\end{figure}
\footnotetext{Eigene Darstellung in Anlehnung an \citealp*[Bild 2.11]{burghardt2007einfuehrung}}

Die Auftrags-Projektorganisation ist ebenfalls eine matrixorientierte 
Organisationsform. Bei dieser gibt es aber keine Doppelunterstellung der Mitarbeiter.
Somit ist hier der Projektleiter auch für die fachtechnische Durchführung des
Projektes verantwortlich.\footnote{\citealp*[Vgl.][S. 58]{burghardt2007einfuehrung}}
In der Grafik \ref{pic:05_projektorganisationen_auftrags}
ist ein Beispiel einer Marix-Projektorganisation abgebildet.

\clearpage

\begin{figure}[htbp]
\begin{center}
\includegraphics[width=0.55\textwidth,angle=0]{./bilder/theorie/05_projektorganisationen_auftrags.pdf}
\caption[]{Auftrags-Projektorganisation\footnotemark}
\label{pic:05_projektorganisationen_auftrags}
\end{center}
\end{figure}
\footnotetext{Eigene Darstellung in Anlehnung an \citealp*[Bild 2.12]{burghardt2007einfuehrung}}

Die Durchführung eines Projektes erfordert nicht immer ein Einrichten einer eigenen
Projektorganisation. Beim Projektmanagement über die Linie wird ebenfalls ein
Projektleiter ernannt, der dann meist eine Art Gruppenführer-Funktion einnimmt,
ähnlich wie bei der Einfluss-Projektorganisation.

\subsection{Projektplanung}
In der Projektplanung definiert man die Strukturplanung, eine Aufwandschätzung,
die Arbeits- und Kostenplanung sowie das Risikomanagement.

In der Strukturplanung wird das Vorhaben technisch, aufgabengemäss und kaufmännisch
anhand den Anforderungen strukturiert. Auf den sich ergebenden Strukturen bauen
alle weiteren Planungsschritte auf. Danach werden daraus die einzelnen Aufgabenpakete
abgeleitet, für die dann eine Aufwandsschätzung durchzuführen ist.\footnote{\citealp*[Vgl.][S. 14]{burghardt2007einfuehrung}}
Wenn möglich sollten zur Schätzung nebst dem eigenen Erfahrungspotential auch
Erfahrungen von Experten zum Thema herangezogen werden.

Ein allgemeines Schema zur Vorgehensweise zum Sammeln von Erfahrungswerten zur 
Aufwandsschätzung ist in der Grafik \ref{pic:02_schema_aufwandsschaetzung}
abgebildet.

\clearpage

\begin{figure}[htbp]
\begin{center}
\includegraphics[width=0.75\textwidth,angle=0]{./bilder/theorie/02_schema_aufwandsschaetzung.pdf}
\caption[]{Sammeln von Erfahrungswerten zur Aufwandsschätzung\footnotemark}
\label{pic:02_schema_aufwandsschaetzung}
\end{center}
\end{figure}
\footnotetext{Eigene Darstellung des Schemas in Anlehnung an \citealp*[S. 112]{litke2007projektmanagement}}

Es gibt diverse Aufwandsschätzungsverfahren die genutzt werden können. Die Methoden unterscheiden
sich beim Schätzverfahren und können zum Beispiel nach folgender Klassifikation 
erfolgen:\footnote{Vgl. \citealp*{noth1986aufwandschaetzung} und \citealp*{knoell1991aufwandsschaetzung}}

\begin{itemize}
    \item Anzahl der Phasen, die abgedeckt werden, zum Beispiel:
    \begin{itemize}
        \item Einzelaktivitäten
        \item Programmierung
        \item Detailentwurf und Programmierung
        \item Gesamter Entwicklungsprozess
    \end{itemize}
    \item Theoretische oder praktische Absicherung, zum Beispiel:
    \begin{itemize}
        \item Unternehmensspezifische Verfahren
        \item Unternehmensunabhängige Praxisverfahren
        \item Wissenschaftlich fundierte Verfahren
    \end{itemize}
    \item Verwendungszweck, zum Beispiel:
    \begin{itemize}
        \item Kosten/Nutzenanalyse zur Kalkulation der Kosten
        \item Kapazitäts- und Terminplanung zur Ermittlung von Plangrössen
    \end{itemize}
\end{itemize}

Anhand den Ergebnissen aus der Aufwandsschätzung werden nun die einzelnen
Arbeitspakete bzw. Teilaufgaben in eine Arbeitsplanung übernommen. Oft empfiehlt
es sich hier einen Netzplan zur Erstellung der Aufgaben- und Terminplanung
anzufertigen. ``Die Netzplantechnik ist trotz aller Kritik eines der 
leistungsfähigsten Projektmanagement-Hilfsmittel, wenn sie richtig eingesetzt wird.''
\footnote{\citealp*[S. 14]{burghardt2007einfuehrung}}

\begin{figure}[htbp]
\begin{center}
\includegraphics[width=0.5\textwidth,angle=0]{./bilder/theorie/03_darstellung_netzplan.pdf}
\caption[]{Konzeptionelle Darstellung eines determinisitischen Netzplanes\footnotemark}
\label{pic:03_darstellung_netzplan}
\end{center}
\end{figure}
\footnotetext{Eigene Darstellung des Schemas in Anlehnung an \citealp*[Bild 3.17]{burghardt2007einfuehrung}}

% http://books.google.de/books?id=m54LKbnCIoYC&pg=PA155&dq=Netzplantechnik&hl=de&ei=Qai-Tf2dNsmDOqL12dsF&sa=X&oi=book_result&ct=result&resnum=9&ved=0CHYQ6AEwCA#v=onepage&q=Netzplantechnik&f=false

\subsection{Projektkontrolle}
An erster Stelle der Projektkontrolle steht der Plan/Ist-Vergleich der vorgegebenen
Projektparameter. Durch einen laufenden Vergleich im Rahmen der Projektkontrolle
erreicht man, dass Abweichungen frühzeitig erkannt werden. Diese Abweichungen
führen zu ``geeigneten'' Massnahmen, die rechtzeitig ergriffen werden können.
Die Projektkontrolle umfasst im ganzen die Termin-, Aufwands-, Kosten-, 
und Sachfortschrittskontrolle, Qualitätssicherung, Projektdokumentation und
das Personalmanagement.\footnote{\citealp*[Vgl.][S. 15]{burghardt2007einfuehrung}}

Die Terminkontrolle wird in der Praxis so umgesetzt, dass die Einhaltung
und Erreichung der gesetzten Meilensteine kontrolliert wird. Wenn ein 
Meilenstein nicht eingehalten werden kann, muss kontrolliert werden, ob die
weiteren davon betroffen sind. Häufig ist das der Fall und die Termine
müssen angepasst und neu gewählt werden. Hierbei ist es wichtig, den Auftraggeber
über die Veränderungen zu informieren.

Die Aufwands- und Kostenkontrolle wird in der Praxis meist durch die Kontrolle
der rapportierten Stunden durchgeführt. Man sollte in beiden Fällen, also der
Termin- und Aufwandskontrolle, eine Trendanalyse erstellen, um eine mögliche
Entwicklung daraus abzuleiten.

\clearpage

In der Praxis werden immer wieder folgende, in der Grafik \ref{pic:06_trendanalyse}
abgebildeten, sechs typische Kurvenverläufe beobachtet.\footnote{\citealp*[Vgl.][S. 177 und S. 194]{burghardt2007einfuehrung}}

\begin{figure}[htbp]
\begin{center}
\includegraphics[width=0.8\textwidth,angle=0]{./bilder/theorie/06_trendanalyse.pdf}
\caption[]{Typische Kurvenverläufe bei Trendanalysen\footnotemark}
\label{pic:06_trendanalyse}
\end{center}
\end{figure}
\footnotetext{Eigene Darstellung des Schemas in Anlehnung an \citealp*[Bild 4.4 und 4.13]{burghardt2007einfuehrung}}

\begin{description}
    \item [Normaler Verlauf] Hier sind geringe Termin- und Kostenverschiebungen 
    ablesbar. Mit hoher Wahrscheinlichkeit kann der Gesamttermin und das Budget 
    eingehalten werden.

    \item [Extrem ansteigender Verlauf] Hier wurde anscheinend laufend zu optimistisch geplant 
    und budgetiert. Der Endtermin wird sich mit hoher Wahrscheinlichkeit erheblich 
    verschieben und das Budget wird nicht eingehalten werden können.

    \item [Trendewende-Verlauf] Hier wurden möglicherweise die geplanten Termine 
    und Budgets künstlich eingehalten und die ``Probleme'' nach hinten verschoben. 
    Gegen Ende zeichnet sich eine erhebliche Veränderung ab. Ein rechtzeitiger 
    korrigierender Eingriff wurde dadurch verhindert.

    \item [Divergierender Verlauf] Gehen die Kurven auseinander, kann man ebenfalls 
    von einer falschen Rapportierung ausgehen. Man sollte die Trendanalyse im 
    Nachhinein ganz überarbeiten und korrigieren.
    
    \item [Gleichmässig fallender Verlauf] Hier wurde anscheinend mit einem sehr 
    grossen Sicherheitspuffer geplant. In so einem Fall ist es wichtig, seine
    Aufwandsschätzung am Ende des Projektes korrigiert in die Erfahrung einfliessen
    zu lassen, damit man in Zukunft genauer plant.
    
    \item [Zickzack-Verlauf] Ein solcher Verlauf deutet auf eine grosse
    Unsicherheit bezüglich des Fortschrittes hin. Mit grosser Wahrscheinlichkeit
    wird der Endtermin auch hier nicht eingehalten werden können.
\end{description}

Die Sachfortschrittskontrolle ist eine der wichtigsten Kontrollaufgaben für
den Projektleiter. Es ist zuggleich aber auch die schwierigste, da oft keine
unmittelbare Messgrössen vorhanden sind.
Die Kernfragen, die man sich in der Sachfortschrittskontrolle stellt, sind folgenden:\footnote{\citealp*[Vgl.][S. 134]{jenny2009projektmanagement}}

\begin{itemize}
    \item Wie verhält sich der Projektaufwand zu den erbrachten Leistungen?
    \item Wie hoch ist der Zielerreichungsgrad der definierten Projektziele?
\end{itemize}

Es wird empfohlen, während der Projektdurchführung mehrmals eine Restaufwands-
und Restzeitschätzung vorzunehmen und die Trendanalysen zu aktualisieren.\footnote{\citealp*[Vgl.][S. 16]{burghardt2007einfuehrung}}

Bei der Projektdokumentation handelt es sich um die vollständigen Informationen
über das zu entwickelnde Produkt. Es empfiehlt sich eine Projektakte mit einer
vorgegebenen Ordnung aufzubauen und eine Art Projekttagebuch zu führen, dessen
Inhalt an keine Ordnungssystematik gebunden ist. Die daraus abzulesenden
Informationen werden zur Projektberichterstattung verwendet.

Das Personalmanagement in einem Projekt beinhaltet eine projektkonforme
Personalführung und das Fördern einer positiven Zusammenarbeit in einem
Projektteam. Wichtig ist stets die Akzeptanz der Projektleitung und die 
Motivation der Projektmitarbeiter. Ein gutes Konfliktmanagement spielt
dabei ebenfalls eine wichtige Rolle, damit man auf Konflikte innerhalb des
Projektes frühzeitig reagieren und sie bewältigen kann. Projektmanagement
wird oft auch als ``permanentes Konfliktmanagement'' bezeichnet. Dabei ist es
wichtig, dass Konflikte als etwas Normales und als Aufgaben oder zu lösende 
Probleme angesehen werden.\footnote{\citealp*[S. 119]{kessler2004projektmanagement}}

\subsection{Projektabschluss}
Der Projektabschluss umfasst die Schritte Produktabnahme, Projektabschluss,
Erfahrungssicherung und Projektauflösung.

Der Projektabschluss wird durch die Produktabnahme eingeleitet. Im besten Fall
wird der Abnahmetest durch eine entwicklungsunabhängige Stelle durchgeführt.
Die Übergabe an den Auftraggeber ist in einem Produktabnahmebericht festzuhalten.
Er sollte aus einem Übergabeprotokoll, dass der Auftragnehmer dem Auftraggeber
überreicht und einem Übernahmeprotokoll, dass der Auftraggeber dem Auftragnehmer
überreicht bestehen. Der Ablauf der Produktübergabe ist in der Grafik
\ref{pic:07_produkteuebergabe} visuell dargestellt.

\begin{figure}[htbp]
\begin{center}
\includegraphics[width=0.6\textwidth,angle=0]{./bilder/theorie/07_produkteuebergabe.pdf}
\caption[]{Produkteabnahmebericht in der Produktübergabe\footnotemark}
\label{pic:07_produkteuebergabe}
\end{center}
\end{figure}
\footnotetext{Eigene Darstellung des Schemas in Anlehnung an \citealp*[Bild 5.2]{burghardt2007einfuehrung}}

Der Produktabnahmebericht sollte eine Beschreibung der Ergebnisse und eventuelle
Nachforderungen beinhalten. In der Regel beginnt nach der Abnahme die
Gewährleistungsfrist. Das bedeutet, dass der Zeitpunkt der Produktabnahme von
rechtlicher Relevanz sein kann und der Produktabnahmebericht vom Auftraggeber
unterschrieben werden sollte.\footnote{\citealp*[Vgl.][S. 86]{cronenbroeck2004handbuch}}
Zu diesem Zeitpunkt sollte man sich auch Gedanken über eine zukünftige Betreuung
machen.

In der Projektabschlussanalyse wird die ehemals erstellte Wirtschaftlichkeitsrechnung
auf ihre Einhaltung durchleuchtet. Zusätzlich werden die Einhaltung der Termine sowie
Leistungs- und Qualitätsmerkmale betrachtet. Wichtige Kennzahlen sind zudem
die Änderungshäufigkeit und Fehlerquote.\footnote{\citealp*[Vgl.][S. 265]{schelle2007projekte}}
Grundsätzlich sollten alle gesammelten Daten in eine Art Erfahrungsdatenbank 
des Unternehmens einfliessen. Diese stellen eine wichtige Voraussetzung für das 
Optimieren von Aufwandsschätzungen dar und können somit auf zukünftige Projekte 
eine positive Wirkung haben.\footnote{\citealp*[Vgl.][S. 275]{burghardt2007einfuehrung}}

\clearpage

\begin{figure}[htbp]
\begin{center}
\includegraphics[width=0.6\textwidth,angle=0]{./bilder/theorie/04_unterteilung_erfahrungsdaten.pdf}
\caption[]{Unterteilung der Erfahrungsdaten\footnotemark}
\label{pic:04_unterteilung_erfahrungsdaten}
\end{center}
\end{figure}
\footnotetext{Eigene Darstellung des Schemas in Anlehnung an \citealp*[Bild 5.7]{burghardt2007einfuehrung}}

Die Projektauflösung ist der letzte Schritt in der Projektabschlussphase. Wie 
zu Beginn erwähnt soll jedes Projekt auch ein eindeutiges Ende haben. Dies ist
für die Projektmitarbeiter besonders wichtig, da sie dann offiziell von dem
Projekt entbunden sind und in neue Projekte und Aufgaben übergeleitet werden
können. Möglicherweise ist es je nach Projekt auch sinnvoll, einen Abschluss
gebührend zu feiern.

\section{Zwischenfazit}
Was nimmt man mit, auf was achten...
  
  \chapter{Analyse IST-Zustand allink}\label{chap:analyse}
  \section{Unternehmen}
Die allink GmbH wurde 2005 von den drei Partnern David Zangger, Christoph Schlatter
und Michael Walder gegründet. David Zangger und Christoph Schlatter kümmern sich
seit jeher als Creative Directors um die grafischen Umsetzungen und die Designqualität.
Michael Walder kümmerte sich bis 2010 nebst den geschäftsführerischen Tätigkeiten
auch um die ganze Informatik. Nach der Fusionierung mit der SiSprocom übergab
er die Leitung der Informatik an Silvan Spross ab und widmet sich nun dem Aufbau der
Beratung und Projektleitung. Seit 2010 bezeichnet sich allink als interdisziplinäre 
und marktnahe Designagentur, wie man ihrer Website entnehmen kann:

\begin{quote}
``allink.creative ist eine Designagentur, die Produkte, Marken und Unternehmen 
attraktiv und einzigartig in Szene setzt. Unser Design soll Menschen erreichen 
und begeistern. Weil wir marktnah arbeiten, stellen wir den wirtschaftlichen 
Nutzen von Design stets in den Vordergrund.
Wir arbeiten interdisziplinär, damit wir die breiten Möglichkeiten an 
Kommunikationskanälen optimal für unsere Kunden einsetzen können. Unser 
Team von Spezialisten aus den vielfältigsten Bereichen findet innovative 
und gesamtheitliche Lösungen, um Marken und Firmen wirkungsvoll zu positionieren. 
Die Bereiche Branding, Online Communication, Packaging Design und Product 
Design zählen dabei zu unseren Kernkompetenzen.''\footnote{Vgl. Rubrik ``About'' 
auf der Website von allink, \url{http://allink.ch/}}
\end{quote}

Das Unternehmen lässt sich zur Zeit in drei Hauptbereiche unterteilen: Beratung, Grafik
und Informatik. Diese Aufteilung ist in der Abbildung \ref{pic:04_funktionales_organigramm}
dargestellt.

\begin{figure}[htbp]
\begin{center}
\includegraphics[width=0.46\textwidth,angle=0]{./bilder/analyse/04_funktionales_organigramm.pdf}
\caption[Funktionales Organigramm von allink]{Funktionales Organigramm von allink\footnotemark}
\label{pic:04_funktionales_organigramm}
\end{center}
\end{figure}
\footnotetext{Eigene Darstellung}

Es kümmern sich jeweils ein bis zwei Partner um einen
Bereich. Dies beinhaltet nebst den Alltagsarbeiten auch organisatorische 
Aufgaben wie das Personalmanagement und die Verteilung der Arbeitspakete an den 
geeignetsten Mitarbeiter. In der Grafik \ref{pic:mitarbeiter_pro_bereich} 
sind die Mitarbeiter pro Bereich abgebildet.

\begin{figure}[htbp]
\begin{center}
\includegraphics[width=0.55\textwidth,angle=0]{./bilder/analyse/mitarbeiter_pro_bereich.pdf}
\caption[Übersicht der Mitarbeitern pro Bereich]{Übersicht der Mitarbeitern pro Bereich\footnotemark}
\label{pic:mitarbeiter_pro_bereich}
\end{center}
\end{figure}
\footnotetext{Eigene Darstellung}

Im Bereich Grafik sind nebst drei Vollzeitangestellen noch ein Praktikant und eine
Lehrtochter angestellt. In der Beratung beschäftigt allink zur Zeit zwei Mitarbeiter.
Die Informatik setzt sich aus zwei Vollzeitangestellten sowie einem Praktikanten
zusammen. Die drei weiteren Informatiker arbeiten als externe Berater und
werden höchst selten in einem internen Projekt beschäftigt.

Trotz dieser Aufteilung in mehrere Bereiche geht der Kommunikationsweg nicht
immer über die für den Mitarbeiter zuständigen Partner. 

\begin{figure}[htbp]
\begin{center}
\includegraphics[width=0.40\textwidth,angle=0]{./bilder/analyse/kommunikationswege.pdf}
\caption[Beispiel der agilen Kommunikationswege]{Beispiel der agilen Kommunikationswege\footnotemark}
\label{pic:kommunikationswege}
\end{center}
\end{figure}
\footnotetext{Eigene Darstellung}

Das ist auch grundsätzlich nicht zu vermeiden, bzw. das muss zu einem gewissen
Grad auch der Fall sein. Ansonsten wären die Abläufe von wenigen Anlaufstellen abhängig
und das ganze würde ins Stocken geraten. Trotzdem birgt es gewisse Schwachstellen.
Zum Beispiel, dass nicht alle Beteiligten eines Projektes den gleichen Informationsstand 
haben. Dies kann sich beim Kontakt mit einem Kunden negativ auswirken, wenn der
Kunde schon über etwas in Kenntnis gesetzt wurde, und der Mitarbeiter am
Telefon selbst noch nicht über die selben Informationen verfügt. Die Schwächen 
werden in der Analyse des Projektablaufes noch genauer beleuchtet.

Der Aufbau einer Beratung, die zur Zeit im Gange ist, soll genau solche Probleme
lösen. Jedoch steckt allink hier noch in den Kinderschuhen und erhofft sich
unter anderem aus dieser Arbeit Erkenntnisse über einen möglichst optimalen 
Projektablauf.


\section{Kunden}
Im Gegensatz zur SiSprocom hat die allink einen relativ grossen Kundenstamm.
Das bedeutet, dass kein grosses Klumpenrisiko existiert. Jedoch ist auch
die Auftragskontinuität der einzelnen Kunden viel kleiner. Für manche arbeitet
allink jeden Monat an einem neuen Projekt, für anderen einmal im Jahr.
In der Grafik \ref{pic:kundenauszug} ist ein Teil der Kunden abgebildet.

\begin{figure}[htbp]
\begin{center}
\includegraphics[width=0.73\textwidth,angle=0]{./bilder/analyse/kundenauszug.jpg}
\caption[Kundenauszug von allink]{Kundenauszug von allink\footnotemark}
\label{pic:kundenauszug}
\end{center}
\end{figure}
\footnotetext{Eigene Darstellung}

Unterteilt man die Kunden von allink in die gängigen Firmengrössen Kleinstunternehmen,
kleine Unternehmen und mittlere Unternehmen, so gelangt man zu einer interessanten 
Verteilung. Die Definition der Unternehmensgrössen basiert auf einer Empfehlung
der Kommission der Europäischen Union\footnote{\citealp*[Vgl.][Anhang Art. 2]{eu_komission_unternehmen}}.
Diese unterteilt die Unternehmen nach dem in der Tabelle \ref{tab:eu_unterteilung} 
abgebildetem Schlüssel.

\begin{table}[h]
\begin{center}
    \begin{tabular}{clccccc}
        \toprule \textbf{Typ} & \textbf{Beschäftigte} & & \textbf{Umsatzerlös} & & \textbf{Bilanzsumme} \\
        \midrule Kleinstunternehmen & $<$ 10 & und & $\leq$ 2 Mio \euro & oder & $\leq$ 2 Mio \euro \\
        \midrule Kleine Unternehmen & $<$ 50 & und & $\leq$ 10 Mio \euro & oder & $\leq$ 10 Mio \euro \\
        \midrule Mittlere Unternehmen & $<$ 250 & und & $\leq$ 50 Mio \euro & oder & $\leq$ 43 Mio \euro \\
        \bottomrule
    \end{tabular}
    \caption[Empfehlung Unternehmensunterteilung, Kommission der Europäischen Union]{Empfehlung 
        Unternehmensunterteilung, Kommission der Europäischen Union\footnotemark}
    \label{tab:eu_unterteilung}
\end{center}
\end{table}
\footnotetext{Eigene Darstellung mit Anlehnung an \citealp*[Anhang Art. 2]{eu_komission_unternehmen}}

Die Kategorisierung der Kunden, abgebildet in der nachstehenden Grafik \ref{pic:kundenkategorisierung},
basiert überwiegend auf den Informationen von moneyhouse\footnote{moneyhouse bietet Handelsregister- und Firmendaten, \url{http://www.moneyhouse.ch}}, 
da man nicht alle Kunden um aktuelle Mitarbeiter- und Umsatzzahlen bitten konnte.
Zusätzlich zu den in der Tabelle \ref{tab:eu_unterteilung} aufgelisteten Kategorien 
wurde noch die Kategorie ``Grosse Unternehmen'' hinzugefügt, um Firmen mit genau 
oder mehr als 250 Mitarbeiter kategorisieren zu können.

\begin{figure}[htbp]
\begin{center}
\includegraphics[width=0.75\textwidth,angle=0]{./bilder/analyse/kundenkategorisierung.pdf}
\caption[Anzahl Kunden der allink in Unternehmensgrössen kategorisiert]{Anzahl 
    Kunden der allink in Unternehmensgrössen kategorisiert\footnotemark}
\label{pic:kundenkategorisierung}
\end{center}
\end{figure}
\footnotetext{Eigene Darstellung}

Es wird absichtlich nicht die vollständige Liste der Kunden von allink und deren 
Kategorisierung beigelegt, da allink nicht alle Kunden in dieser Arbeit namentlich 
aufgelistet haben möchte. Bei Bedarf kann die Liste aber bei allink eingesehen werden.

Wie man erkennen kann, arbeitet allink überwiegend mit Kleinst- und kleinen 
Unternehmungen zusammen. Interessanterweise auch mit mehr grossen als mittleren
Unternehmen. Die verschiedenen Firmenkulturen der Kunden haben auch einen 
Einfluss auf die Projektabläufe. Grundsätzlich stellen sich wohl alle Kunden
der Herausforderung, ihre Projekte gut zu organisieren. Je nach Auflagen oder
Abläufe der Kunden, hat dies auch Auswirkungen auf den Projektablauf der allink.
Deshalb wäre es falsch anzunehmen, dass sie zur Zeit bei jedem Projekt und 
Kunden das selbe Vorgehen anwenden können.

Möglicherweise genau deshalb existiert zur Zeit kein definierter Projektablauf. 
Doch gibt es klar erkennbare Gemeinsamkeiten über alle Projekte, bzw. Projektabläufe.
Im nächsten Kapitel \ref{chap:projektablauf} wird versucht, den heutigen Projektablauf so 
abzubilden, dass er für die meisten vergangenen Projekte Gültigkeit hat.


\section{Projektablauf}\label{chap:projektablauf}
\input{./kapitel/06_03_projektablauf.tex}

\section{Verwendete Software}
Die Untersuchung der zur Zeit eingesetzten Sofware beschränkt sich auf jene,
die einen direkten Einfluss auf den aktuellen Projektablauf bei allink haben.
Es wird keine Software aufgelistet, die zur Erbringung der eigentlichen 
Dienstleistungen im Unternehmen eingesetzt wird.

In der Tabelle \ref{tab:verwendete_software} sind die verwendeten Tools
aufgelistet. Zu jeder Software wird der Hersteller, die Kategorisierung und der 
Verwendungszweck innerhalb von allink und dem Projektablauf angegeben. Danach
wird der Verwendungszweck für jede Software noch genauer umschrieben und wo
möglich mit einem Beispiel untermalt.

\begin{longtable}{lllp{6cm}}
    \toprule \textbf{Bezeichnung} & \textbf{Hersteller} & \textbf{Kategorie} & \textbf{Verwendungszweck} \\
    \midrule MacOS X & Apple & Betriebssystem & 
        \begin{minipage}[t]{6cm}
            \begin{compactitem}
                \item Über den ganzen Projektablauf
            \end{compactitem}
        \end{minipage}
        \\\\
    \midrule Microsoft Windows & Microsof & Betriebssystem & 
        \begin{minipage}[t]{6cm}
            \begin{compactitem}
                \item Tests während der Entwicklung
                \item Präsentation in der Abnahme
            \end{compactitem}
        \end{minipage}
        \\\\
    \midrule MacOS X Server & Apple & Betriebssystem &
        \begin{minipage}[t]{6cm}
            \begin{compactitem}
                \item Zentrale Dateiablage
                \item Server des Stundenwidgets
                \item Internes Wiki der Informatik
            \end{compactitem}
        \end{minipage}
        \\\\
    \midrule Apple iWork & Apple & Office Suite &
        \begin{minipage}[t]{6cm}
            \begin{compactitem}
                \item Kalkulation von Offerten
                \item Erstellung von Offerten
                \item Erstellung von Rechnungen und Briefschaften
            \end{compactitem}
        \end{minipage}
        \\\\
    \midrule Microsoft Office & Microsoft & Office Suite &
        \begin{minipage}[t]{6cm}
            \begin{compactitem}
                \item Informationsaustausch mit Kunden
            \end{compactitem}
        \end{minipage}
        \\\\
    \midrule Apple Mail & Apple & E-Mail Software &
        \begin{minipage}[t]{6cm}
            \begin{compactitem}
                \item Korrespondenz mit Kunden
                \item Interne Kommunikation
            \end{compactitem}
        \end{minipage}
        \\\\
    \midrule Stundenwidget & allink & Dashboardwidget &
        \begin{minipage}[t]{6cm}
            \begin{compactitem}
                \item Stundenrapportierung
            \end{compactitem}
        \end{minipage}
        \\\\
    \bottomrule
    \caption[Verwendete Software bei allink]{Verwendete Software bei allink\footnotemark}
    \label{tab:verwendete_software}
\end{longtable}
\footnotetext{Eigene Darstellung}

\subsubsection{MacOS X}
Als Hauptbetriebssystem verwendet allink MacOS X\footnote{Betriebssystem von Apple, \url{http://www.apple.com/macosx/}}.
Es läuft auf allen Arbeitsstationen der Mitarbeiter und Partner. Dies hat
viele Gründe. Grafiker arbeiten überwiegend mit MacOS X, da sich Apple
früh mit Adobe zusammenschloss um Werkzeuge für Grafiker zur Verfügung zu stellen.
Seit mehreren Jahren ist MacOS X auch für Entwickler wieder interessant. Die 
Mitarbeiter sind gemäss dem Auftraggeber sehr zufrieden damit und die Geschäftsleitung 
möchte die Verwendung von MacOS X zu diesem Zeitpunkt nicht in Frage stellen.

\subsubsection{Microsoft Windows}
Zusätzlich haben einige der Entwickler noch Microsoft Windows\footnote{Betriebssystem von Microsoft, \url{http://www.microsoft.com/windows/}}
über eine Virtualisierungslösung installiert. Dies wird überwiegend zu Testzwecken 
benötigt. Es existiert zu diesem Zeitpunkt aber kein Rechner, der ausschliesslich
mit Microsoft Windows arbeitet.

\subsubsection{MacOS X Server}
Auf dem Server der allink liegt das zentrale Dateiablagesystem. Die Zugriffsrechte können
auf dem Server für jede Freigabe definiert werden. So haben zum Beispiel
die Mitarbeiter keinen Zugriff auf den Administrationsordner der Geschäftsleitung.
Über diesen Server laufen auch die einzelnen Firmenkalender der Partner. Sie
können so gemeinsam Termine buchen und haben stets Einblick in die anderen
Kalender. Auch hier ist sich die Geschäftsleitung einig, zur Zeit keine 
Veränderung vorzunehmen.

\subsubsection{Apple iWork}
Für interne Zwecke setzt allink zur Zeit auf die Apple eigene Office Suite iWork\footnote{Office Suite von Apple, \url{http://www.apple.com/de/iwork/}}.
Pages wird überwiegend zur Erstellung von Offerten und Rechnungen verwendet. Mit Numbers
werden Tabellenkalkulationen wie die Liquiditätsplanung oder Lohnblätter erstellt.
Da dies aber von der Geschäftsleitung nie klar kommuniziert wurde, existieren
auch einige Excel Files, das Pendant der Microsoft Office Suite.

\subsubsection{Microsoft Office}
Da viele Kunden mit Microsoft Produkten arbeiten benötigt auch allink die
Office Suite\footnote{Microsoft Office für Mac, \url{http://www.microsoft.com/germany/mac}}, 
um Dateien mit Kunden ohne Interoperabilitätsprobleme austauschen zu können. 
Nicht jede Arbeitsstation verfügt zur Zeit über eine Installation. Da diese
Suite aber im Vergleich mit Apples iWork einiges teurer in den Anschaffung ist,
möchte die Geschäftsleitung diese über kurz oder lang nur noch auf wenigen
Arbeitsstationen installieren.

\subsubsection{Apple Mail}
Apple Mail ist in das Betriebssytem von MacOS X integriert und wird gratis
mitgeliefert. Es verfügt über alle nötigen Funktionen, die eine E-Mail-Client-Software
heutzutage erfüllen muss. Auch die Verwendung von Apple Mail möchte die Geschäftsleitung 
zur Zeit nicht in Frage stellen.

\subsubsection{Stundenwidget}
Das Stundenwidget ist eine Eigenentwicklung von allink und läuft
im Dashboard\footnote{Widget Lösung von Apple, \url{http://www.apple.com/downloads/dashboard/}}
des Betriebssystems. Es ermöglicht Stunden auf ein Projekt zu buchen.
Die Daten werden auf dem eigenen Server gespeichert und pro Projekt abgelegt. Dieses Widget ist auf allen
Arbeitsstationen installiert und wird von den Mitarbeitern gelegentlich verwendet
um ihre Stunden auf ein Projekt zu rapportieren.


\section{Stärken und Schwächen}
In diesem Kapitel werden die Stärken und Schwächen des Unternehmens mit Fokus
auf den aktuellen Projektablauf genauer untersucht. Als Methode
wird die SWOT-Analyse\footnote{Die SWOT-Analyse ist ein Instrument um
Stärken, Schwächen, Chancen und Risiken darzustellen. Es eignet sich sowohl zur Situationsanalyse
wie auch als Instrument der Strategieformulierung. \citealp*[Vgl.][S. 134]{homburg2000quantitative}} 
verwendet, welche in der nachstehenden Grafik \ref{pic:swot_analyse} abgebildet ist.
Die einzelnen Punkte sind in Zusammenarbeit mit der Geschäftsleitung in einem Workshop
erarbeitet worden.

% Der Studierende ist sich bewusst, dass sich die Chancen und Risiken 
% nicht nur auf den Projektablauf sondern auf das ganze Unternehmen beziehen. 
% Da der Projektablauf jedoch ein Kernelement des Unternehmens ist, können sie 
% auch bei dieser Betrachtung verwendet werden.

\begin{figure}[htbp]
\begin{center}
\includegraphics[width=0.8\textwidth,angle=0]{./bilder/analyse/swot_analyse.pdf}
\caption[SWOT-Analyse von allink]{SWOT-Analyse von allink\footnotemark}
\label{pic:swot_analyse}
\end{center}
\end{figure}
\footnotetext{Eigene Darstellung}

Es wird nun im Detail auf die einzelnen Punkte aus der SWOT-Analyse eingegangen
und wo möglich mit einem Beispiel aus der Praxis untermalt.

\subsection{Stärken}
Bei den Stärken handelt es sich um Eigenschaften des Unternehmens,
die es zu diesem Zeitpunkt mitbringt und teilweise noch nicht vollständig
ausnutzt. 

\subsubsection{Stabiles Team}
Das Unternehmen verfügt über ein starkes, ausgewogenes und interdisziplinäres Team. 
Sowohl im Bereich der Grafik wie auch der Informatik sind überaus fähige Mitarbeiter 
angestellt, deren Potenzial noch nicht vollständig ausgeschöpft wird. Die Beratung
befindet sich erst im Aufbau und das Unternehmen hat somit noch nicht ausreichend
Erfahrungen sammeln können. Die Mitarbeiter der Beratung sind ebenfalls
top motiviert, es ist jedoch noch zu früh um über die Fähigkeiten treffende
Aussagen machen zu können. Allgemein kann man das Team als extrem leistungswillig
bezeichnen, da sich alle zusammen mit der Agentur entwickeln möchten.

\subsubsection{Gute Infrastruktur}
Das Unternehmen verfügt über eine gute und moderne Infrastruktur. Kein Arbeitsgerät
ist älter als eineinhalb Jahre und überall sind die neusten Softwarepakete installiert.
Die allink legt grossen Wert auf gute Werkzeuge, da so die neusten Technologien
angewendet werden können und es daran nicht scheitern soll.

\subsubsection{Fundierte Erfahrung}
Durch die in den letzten fünf Jahren umgesetzten Projekte verfügt das Unternehmen
über einen grossen Erfahrungsschatz auf den es zurückgreifen kann. Das hilft
überwiegend bei der Beratung von Kunden.
Auch verfügt das Unternehmen, wie man der Analyse der Kunden schon entnehmen
konnte, über einen relativ grossen Kundenstamm. Dies ist vor allem bei der
Akquisition sehr hilfreich, da man auf eine grosse Anzahl Referenzprojekte 
verweisen kann.

\subsection{Chancen}
Bei den Chancen handelt es sich um positive Auswirkungen, die zu erwarten sind,
wenn das Unternehmen die Schwächen in den Griff kriegt.

\subsubsection{Mehr Aufträge}
Der grosse Kundenstamm und die zur Zeit hohe Reputation des Unternehmens
birgt ein grosses Potenzial an neuen Aufträgen. Dabei ist das Potenzial der
Akquisition erst gering ausgeschöpft. Nur ein kleiner Prozentsatz der durchgeführten
Projekte in den letzten 5 Jahren wurde akquiriert. In den meisten Fällen entstanden
die Projekte durch direkte Anfragen von Kunden oder Partneragenturen.

\subsubsection{Gesundes Wachstum}
Die allink kann zur Zeit nicht alle Projektanfragen annehmen, da sie über zu
wenig Mitarbeiter verfügt. Es könnten mehr Mitarbeiter angestellt werden. Dies
wurde aber bis anhin, durch die Anzeichen der Risiken die man schon mit dem bisherigen
Wachstum bewältigen muss, von der Geschäftsleitung vermieden.

\subsection{Schwächen}
Bei den Schwächen handelt es sich um Dinge die zur Zeit im Unternehmen nicht
optimal gelöst sind. Wenn diese nicht frühzeitig erkannt werden und nichts
dagegen unternommen wird, kann das die Entwicklung der Chancen verhindern und
das Eintreten der Risiken fördern.

\subsubsection{Fehlende Qualitätssicherung}
Durch die Überbelastung der Mitarbeiter können mit der Zeit die versprochenen Timings
nicht mehr eingehalten werden. Da die Ablagestrukturen nicht einheitlich geregelt
sind, kann es vorkommen, dass ein Mitarbeiter einem Kunden ein veraltetes oder
noch nicht freigegebenes Dokument sendet. Dies zieht einen zusätzlichen 
Mehraufwand mit sich, da man sich beim Kunden entschuldigen und rechtfertigen
muss. Zusätzlich strapaziert es auch die Beziehung zum Kunden.
Oft bleibt gegen Ende eines Projektes auch zu wenig Zeit die nötige 
Qualitätskontrollen durchzuführen, da man sich möglichst schnell um ein anderes,
möglicherweise auch schon überfälliges, Projekt kümmern muss. Der Kunde entdeckt
dann offensichtliche Fehler selbst und zweifelt zwangsläufig an der ganzen Arbeit.

\subsubsection{Mangelnde Effizienz}
Einfache Abläufe werden unnötig verkompliziert. Daten von vergangenen Projekten 
sind nicht mehr auffindbar, da sie nicht sauber archiviert wurden. Die ganze
Struktur wird dadurch langsam und ineffizient. Was sich wiederum negativ auf die zur
Verfügung stehende Zeit auswirkt.
Man hält zudem vor, während und nach einem Projekt nur an wenigen Standards fest. 
Das ganze Vorgehen ist nicht einheitlich, da in jedem Projekt wieder von
neuem entschieden wird, wie man vorgehen will. Es werden nur wenige einheitliche
Dokumente verwendet, zum Beispiel für die Erstellung von Offerten und Rechnungen.
Aber auch da entstehen schnell Fehler, zum Beispiel während der Umstellungen des 
Mehrwert Steuersatzes von 7.6\% auf 8\%. Da kein einheitliches Basistemplate
existiert, muss bei jeder Rechnung noch einmal sichergestellt werden, dass
der korrekte Steuersatz hinterlegt ist.

\subsubsection{Kein Controlling}
Auch bietet die fehlende bzw. chaotische Struktur nur wenige Punkte um Kennzahlen
zu messen. Den Umsatz den man mit dem Projekt erzielt hat ist zwar bekannt,
jedoch kann nur aus dem Gefühl heraus erahnt werden, ob mit dem Projekt einen
Gewinn für die Firma erzielt werden konnte. Die Mitarbeiter sind zwar angehalten
ihre Stunden in ein gemeinsames Stundenwidget einzutragen, jedoch werden
die Informationen nicht ausgewertet und können nicht mehr einzelnen Mitarbeitern
zugeordnet werden.

\subsection{Risiken}
Bei den Risiken handelt es sich um negative Auswirkungen, die zu erwarten sind,
wenn das Unternehmen die Schwächen nicht in den Griff bekommt.

\subsubsection{Hohe Kosten}
Durch die fehlende Kontrolle während eines Projektes, verliert man die
Übersicht über die Aufwände und schlussendlich die Kosten. Dadurch entsteht
ein Kostenrisiko, welches Konsequenzen für die Liquidität von allink haben kann.

\subsubsection{Ressourcenüberlastung}
Die Belastung für den Mitarbeiter wie auch für die Partner ist so über
längere Zeit nicht tragbar. Durch eine Überarbeitung kann es zu Ausfällen kommen, die
die Situation zusätzlich verschlimmern könnten. Das ganze endet in einer 
schlechten Firmenkultur und das Unternehmung beginnt von Innen zu zerfallen.

\subsubsection{Reputationsverlust}
Da man Timings nicht mehr einhalten kann und man sich gegenüber dem Kunden
oft rechtfertigen muss entsteht ein schlechtes Bild der Unternehmung und sie
verliert an Vertrauen. Da die Konkurrenz im Tätigkeitsfeld der allink relativ
gross ist, ist ein möglicher Absprung und Angenturwechsel seitens des Kunden nicht 
auszuschliessen.


% \subsection{Methode zur Analyse}
% \subsubsection{Vorgehensmodell nach Grochla}
% \input{./kapitel/vorgehensmodell.tex}

  
  \chapter{Branchenvergleich}\label{chap:branchenvergleich}
  \section{Interviews mit anderen Agenturen}
Da allink nicht die erste Agentur ist, die sich den Herausforderungen der
Organisation und des Projektmanagement stellt, macht es Sinn sich in der Branche
umzusehen und ähnliche Agenturen zu befragen. Der Studierende erhofft sich aus
den Resultaten Hinweise und Tipps bezüglich Projektmanagement und möglichen
Tools die unterstützend eingesetzt werden können.

Als Interviewpartner werden möglichst ähnliche Agenturen wie die allink gesucht.
Optimal sind Agenturen, die die Herausforderung des Wachstums schon meistern
konnten. Der Studierende fragt in Zürich ein paar vergleichbare Agenturen an.

\subsection{Fragekatalog}
Der in der Tabelle \ref{tab:fragekatalog} dargestellte Fragekatalog wurde für 
das Interview mit einer anderen Unternehmung aufgebaut. Die einzelnen Themenblöcke 
und die Zusammenstellung der Fragen wurden gezielt auf die Struktur der bestehenden 
IST-Analyse der allink zusammengestellt um einen möglichst guten Vergleich darstellen 
zu können.

\newcounter{qcounter}
\begin{longtable}{lp{14cm}}
    \toprule \textbf{Nr.} & \textbf{Frage} \\
    \midrule & \textbf{Allgemeine Fragen} \\
    \midrule \addtocounter{qcounter}{1}\arabic{qcounter} & Wer ist mein Interviewpartner? \\
    \midrule \addtocounter{qcounter}{1}\arabic{qcounter} & Was ist die Funktion meines Interviewpartners? \\
    \midrule \addtocounter{qcounter}{1}\arabic{qcounter} & Was sind die Aufgaben meines Interviewpartners? \\
    \midrule & \textbf{Unternehmen} \\
    \midrule \addtocounter{qcounter}{1}\arabic{qcounter} & Wann wurde das Unternehmen gegründet? \\
    \midrule \addtocounter{qcounter}{1}\arabic{qcounter} & Wie viele Partner mit Mitspracherecht existieren? \\
    \midrule \addtocounter{qcounter}{1}\arabic{qcounter} & Wie ist das Organigramm des Unternehmens aufgebaut? \\
    \midrule \addtocounter{qcounter}{1}\arabic{qcounter} & Wie viele Vollzeitangstellte werden beschäftigt? \\
    \midrule \addtocounter{qcounter}{1}\arabic{qcounter} & Sind Praktikanten angestellt? \\
    \midrule \addtocounter{qcounter}{1}\arabic{qcounter} & Bietet das Unternehmen Lehrstellen an? \\
    \midrule & \textbf{Kunden} \\
    \midrule \addtocounter{qcounter}{1}\arabic{qcounter} & Wie viele Kunden sind Kleinstunternehmen? \\
    \midrule \addtocounter{qcounter}{1}\arabic{qcounter} & Wie viele Kunden sind kleine Unternehmen? \\
    \midrule \addtocounter{qcounter}{1}\arabic{qcounter} & Wie viele Kunden sind mittlere Unternehmen? \\
    \midrule \addtocounter{qcounter}{1}\arabic{qcounter} & Wie viele Kunden sind grosse Unternehmen? \\
    \midrule & \textbf{Projektablauf} \\
    \midrule \addtocounter{qcounter}{1}\arabic{qcounter} & Wie viele Projekte werden akquiriert? \\
    \midrule \addtocounter{qcounter}{1}\arabic{qcounter} & Wie viele Projekte entstehen durch direkte Anfragen? \\
    \midrule \addtocounter{qcounter}{1}\arabic{qcounter} & Wie werden die Aufwände eines potenziellen Projektes geschätzt? \\
    \midrule \addtocounter{qcounter}{1}\arabic{qcounter} & Wie wird auf Änderungen während des Projektes reagiert? \\
    \midrule \addtocounter{qcounter}{1}\arabic{qcounter} & Kommt es vor, dass Projekte während der Durchführung abgebrochen werden? \\
    \midrule \addtocounter{qcounter}{1}\arabic{qcounter} & Von wem werden die Arbeitspakete zusammengestellt? \\
    \midrule \addtocounter{qcounter}{1}\arabic{qcounter} & Wie werden die Arbeitspakete verteilt? \\
    \midrule \addtocounter{qcounter}{1}\arabic{qcounter} & Wie werden die Aufwände rapportiert? \\
    \midrule \addtocounter{qcounter}{1}\arabic{qcounter} & Wer kommuniziert direkt mit einem Kunden? \\
    \midrule \addtocounter{qcounter}{1}\arabic{qcounter} & Wie wird das Feedback eines Kunden verarbeitet? \\
    \midrule \addtocounter{qcounter}{1}\arabic{qcounter} & Wie wird mit zusätzlich zu verrechneten Anforderungen verfahren? \\
    \midrule \addtocounter{qcounter}{1}\arabic{qcounter} & Wie werden die Projektdaten archiviert? \\
    \midrule & \textbf{Verwendete Software} \\
    \midrule \addtocounter{qcounter}{1}\arabic{qcounter} & Auf welches Betriebssystem setzt das Unternehmen? \\
    \midrule \addtocounter{qcounter}{1}\arabic{qcounter} & Welche Office Suite setzt das Unternehmen ein? \\
    \midrule \addtocounter{qcounter}{1}\arabic{qcounter} & Was für Projektmanagement-Software wird verwendet? \\
    \midrule & \textbf{Stärken und Schwächen} \\
    \midrule \addtocounter{qcounter}{1}\arabic{qcounter} & Wo liegen die Stärken des Unternehmens? \\
    \midrule \addtocounter{qcounter}{1}\arabic{qcounter} & Wo sieht das Unternehmen ihre Chancen? \\
    \midrule \addtocounter{qcounter}{1}\arabic{qcounter} & Wo liegen die Schwächen des Unternehmens? \\
    \midrule \addtocounter{qcounter}{1}\arabic{qcounter} & Mit was für Risiken sieht sich das Unternehmen konfrontiert? \\
    \bottomrule
    \caption[Fragekatalog zur Marktanalyse]{Fragekatalog zur Marktanalyse\footnotemark}
    \label{tab:fragekatalog}
\end{longtable}
\footnotetext{Eigene Darstellung}

Der erstellte Fragenkatalog dient als Leitfaden im Interview. Die Antworten
werden entgegengenommen, diskutiert und anschliessend in beschreibender Form
dokumentiert. Die verweisenden Fragen werden in Klammern an der dazugehörigen
Textstelle angegeben. Falls gewissen Fragen nicht beantwortet wurden, wird dies 
erwähnt und begründet.

\subsection{Panter IIc}
Das Unternehmen Panter IIc mit Sitz in Zürich hat sich freundlicherweise dazu
bereiterklärt mit dem Studierenden ein Interview durchzuführen. Der Interview
Partner ist Syrus Mozafar. Er ist Teilhaber der Panter IIc und Mitglied der
Geschäftsleitung (\textbf{1}).

Er ist laut eigenen Angaben auch für das Projektmanagement zuständig (\textbf{2})
und oft selbst Projektleiter. Sein Aufgabenbereich liegt speziell in der 
Projektplanung, Konsolidierung der Ressourcenplanung und das Auszahlen von
Löhnen in der Buchhaltung (\textbf{3}).

\subsubsection{Unternehmen}
Das Unternehmen ist im Jahre 2005 als GmbH gegründet worden. Die Rechtsform
ist bis heute erhalten geblieben (\textbf{4}). Es existieren fünf Teilhaber,
jedoch kann in ihrer Firmenkultur jeder Mitarbeiter bei grösseren Entscheidungen
des Unternehmens mitentscheiden (\textbf{5}).

Es existiert zur Zeit kein richtiges Organigramm, die Struktur ist flach
und jeder Mitarbeiter hat gewisse Kompetenzen und Aufgaben (\textbf{6}). 
Insgesamt hat Panter zwölf Mitarbeiter und zusätzliche sechs Mitarbeiter im
Personalverleih. Die Anstellungspensum variiert zwischen 20\% bis 80\% (\textbf{7}).
Zur Zeit wird ein Praktikant jedoch noch keine Lehrlinge beschäftigt und die Geschäftsleitung
wird dies zu diesem Zeitpunkt nicht ändern, da sie darin eher einen Mehraufwand
als Nutzen sehen (\textbf{8} und \textbf{9}).

\subsubsection{Kunden}
Panter verfügt über einen relativ grossen Kundenstamm. Die Verteilung
der Unternehmensgrössen der Kunden unterteilen sich in ca. 40\% Kleinstunternehmen,
20\% kleine Unternehmen, 10\% mittlere Unternehmen und 30\% grosse
Unternehmen (\textbf{10} bis \textbf{13}).

\subsubsection{Projektablauf}
Bei Panter werden ca. 20\% der Projekte selbst akquiriert (\textbf{14}) und ca.
80\% entstehen durch direkte Anfragen oder Folgeaufträge (\textbf{15}). Kleinere
Projekte mit einem Umsatzvolumen bis 20'000 CHF werden nur grob im Alleingang
des Projektleiters geschätzt. Bei grösseren Projekten werden auf Grund der
Technologiewahl Experten innerhalb von Panter oder ausserhalb herbeigezogen
um eine möglichst genaue Schätzung abgeben zu können (\textbf{16}). Sie verwenden
dazu keine speziellen Verfahrenstechniken und setzen überwiegend die Software
Microsoft Excel ein. Bei grösseren Änderungen während eines Projektes wird
erneut eine Schätzung vorgenommen und dem Kunden die zusätzlichen Aufwände
offeriert (\textbf{17}). Es wird von Fall zu Fall entschieden ob eine Änderung 
zusätzlich verrechnet oder dem Kunden ``geschenkt'' wird (\textbf{24}). Bis zu 
diesem Zeitpunkt kam es erst einmal zu einem vollständigen Abbruch eines Projektes 
während dessen Durchführung (\textbf{18}).

Die Arbeitspakete werden überwiegend während der Erstellung der Schätzung und 
der Offerte vom Projektleiter und den Experten zusammengestellt (\textbf{19}).
Dabei ist im Normalfall immer auch eine Person der Geschäftsleitung vertreten.
Die Arbeitspakete werden dann auch gleich von dieser Gruppe an die für das 
Projekt eingeplanten Ressourcen verteilt (\textbf{20}).

Die Aufwände werden von den Mitarbeitern sehr exakt rapportiert (\textbf{21}),
da die Lohnsummen der einzelnen Mitarbeiter von den Anzahl geleisteten Stunden
abhängt. Es existieren somit keine Fixlöhne bei Panter.

Die Kommunikation mit dem Kunden erfolgt im Normalfall über den Projektleiter.
Es kommt aber auch vor, dass ein Mitarbeiter direkt bei einem Kunden zusätzliche
Informationen einfordert (\textbf{22}). Es existieren keine fixen Regeln dazu.
Das Feedback des Kunden wird im Unternehmens Wiki eingetragen. Da die zerstreute
Verteilung der Projektdaten in der Vergangenheit schon öfters bemängelt wurde, 
baut Panter zu diesem Zeitpunkt eine Struktur für ein Projektarchiv auf. Sie
verwenden zudem ein allgemeines E-Mail Konto um projektspezifische E-Mails
zentral und für alle Mitarbeiter zugänglich abzulegen (\textbf{23} und \textbf{25}).

\subsubsection{Verwendete Software}
Das Unternehmen setzt auf kein spezifisches Betriebssystem. Jeder Mitarbeiter
hat sein eigenes Gerät mit seinem präferierten System installiert. Einzig für
die unternehmenseigenen Server wird das Betriebssystem Debian\footnote{Debian 
ist ein frei verfügbares Betriebssystem, \url{http://www.debian.org/}} 
vorausgesetzt (\textbf{26}).

Also Office Suite setzt Panter auf die Open-Source
Lösung OpenOffice\footnote{OpenOffice ist eine frei verfügbare Office Suite, 
\url{http://de.openoffice.org/}}. Zusätzlich, um Interoperabilitätsprobleme
mit Kunden zu vermeiden, hat Panter eine Arbeitsstation mit Microsoft Windows
und Microsoft Office ausgerüstet (\textbf{27}).

Panter setzt überwiegend auf die Projektmanagement Software RedMine\footnote{RedMine
ist eine Webapplikation für Projektmanagement, \url{http://www.redmine.org/}},
wobei sie nicht vollständig sondern nur Teile davon nutzen (\textbf{28}). 
Das unternehmenseigene Wiki und diverse Exceldokumente unterstützen die in RedMine
verwalteten Projekte. 

\subsubsection{Stärken und Schwächen}
Die Stärken sieht Panter zur Zeit überwiegend in ihrer Grösse und der flachen
Struktur. Mit dieser können ihre Projekte zur Zeit sehr erfolgreich durchgeführt 
werden. Auch sehen sie das Mitspracherecht aller Mitarbeiter als grossen Vorteil.
Zusätzlich sei auch die verwendete und moderne Technologie bis jetzt immer
ein grosser positiver Faktor gewesen (\textbf{29}).

Chancen sehen sie ebenfalls im Wachstum der Agentur und allgemein könnten
sie effizienter werden. Die Schätzverfahren können noch verbessert werden und
in Zukunft möchten sie mehr auf Leistung und nicht pauschal verrechnen können (\textbf{30}).

Die Schwächen liegen hingegen bei den nicht vollständig geregelten 
Verantwortungsbereichen und dass sie keine Cashcow\footnote{Als Cashcow bezeichnet
man in der Betriebswirtschaftslehre ein Produkt oder einen Kunden, mit dem man
laufend hohe Gewinne erwirtschaften kann.} besitzen (\textbf{31}).

Als Risiko sehen sie die schwankende Liquiditätsplanung, die nur auf zwei Monate 
hinaus garantiert werden kann. Zusätzlich fürchten sie eine negative Veränderung 
der Firmenkultur bei einem weiteren Wachstum (\textbf{32}).

\subsection{FEINHEIT GmbH}
Das Unternehmen FEINHEIT GmbH mit Sitz in Zürich hat sich freundlicherweise
ebenfalls dazu bereiterklärt mit dem Studierenden ein Interview zum Branchenvergleich
durchzuführen. Der Interview Partner ist Matthias Kestenholz. Er ist einer der
fünf Gründer der FEINHEIT (\textbf{1}). Seine Funktion lässt sich als CTO\footnote{Der Chief Technical Officer (CTO) ist
für die technische Entwicklung und Forschung innerhalb eines Unternehmens
verantwortlich.} am Besten umschreiben (\textbf{2}). Er leitet zu 20\% die
technische Entwicklung und entwickelt selber noch 80\% seiner Arbeitszeit. Er
ist zudem zuständig für den Aufbau der Qualitätskontrolle und hat das letzte
Wort bei einer Neuanstellung in der Informatik (\textbf{3}).

\subsubsection{Unternehmen}
Das Unternehmen ist im Jahre 2005 als GmbH gegründet worden. Die Rechtsform
ist bis heute erhalten geblieben (\textbf{4}). Die FEINHEIT hat sich vor ca. 
einem halben Jahr viele Gedanken zur Organisation ihres Unternehmens gemacht 
und eine klare Struktur erstellt. Das aktuelle Organigramm ist in der Grafik 
\ref{pic:organigramm_feinheit} abgebildet.

\begin{figure}[htbp]
\begin{center}
\includegraphics[width=0.4\textwidth,angle=0]{./bilder/analyse/organigramm_feinheit.pdf}
\caption[Organigramm der FEINHEIT GmbH]{Organigramm der FEINHEIT GmbH\footnotemark}
\label{pic:organigramm_feinheit}
\end{center}
\end{figure}
\footnotetext{Eigene Darstellung nach der Beschreibung von Matthias Kestenholz.}

Es existieren fünf Partner bzw. Gründer (\textbf{5}) und ein Geschäftsführer,
der gleichzeitig auch ein Partner ist. Die Stabsstellen Qualitätssicherung setzten
sich aus zwei und das Sekretariat aus einer Person zusammen. In der Geschäfts-
bzw. Teamleitung existieren pro Team ein bis zwei Teamleiter und die Teams
setzen sich aus vier bis sieben Mitarbeitern zusammen (\textbf{6}).

Insgesamt beschäftigt die FEINHEIT zwanzig Mitarbeiter. Das Anstellungspensum
variiert zwischen 60\% und 100\% (\textbf{7}). Darunter sind zwei Praktikanten (\textbf{8})
jedoch noch keine Lehrlinge (\textbf{9}) beschäftigt. Es wird jedoch angestrebt
bis gleichzeitig drei Praktikanten und in Zukunft auch Lehrlinge zu beschäftigen.

\subsubsection{Kunden}
Die FEINHEIT verfügt über einen realtiv grossen Kundenstamm. Die Verteilung der
Unternehmensgrössen der Kunden unterteilen sich in ca. 30\% Kleinstunternehmen,
30\% kleine Unternehmen, 10\% mittlere Unternehmen und 30\% grosse Unternehmen 
(\textbf{10} bis \textbf{13}).

\subsubsection{Projektablauf}
Bei der FEINHEIT werden ca. 20\% bis 30\% der Projekte selbst akquiriert (\textbf{14})
und ca. 70\% bis 80\% entstehen durch direkte Anfragen oder Folgeaufträge (\textbf{15}).
Die Aufwände eines potenziellen Projektes werden von einem Teamleiter, der auch
gleich die Aufgabe des Projektleiters einnimmt, zusammen mit dessen Team geschätzt.
Die Schätzungen basieren überwiegend auf Referenzprojekten und Erfahrungswerten (\textbf{16}).
Bei Veränderungen während eines Projektes wird in den meisten Fällen eine Nachofferte
erstellt. Es kommt auch vor, dass keine Nachofferte erstellt werden kann, wenn 
sich die erste Offerte von den neuen Aufwänden nicht klar abgegrenzt hat (\textbf{17} und \textbf{24}).
Selten wurde ein Projekt während der Durchführung abgebrochen, wenn aber immer
von Seiten der Kunden (\textbf{18}).

Die Arbeitspakete werden vom Teamleiter zusammen mit dem Team erarbeitet (\textbf{19})
und auch gleich auf die einzelnen Mitarbeiter im selben Team verteilt (\textbf{20}).
Wenn das Team nicht über genügend Ressourcen für das Projekt verfügt, kann der
Teamleiter sich mit den anderen Teamleitern absprechen und temporär eine freie
Ressource eines anderen Teams in das Projekt miteinbeziehen.

Die Aufwände werden von den Mitarbeitern sehr exakt auf 0.1 Stunden genau 
rapportiert (\textbf{21}). Diese dienen zusätzlich zur Kontrolle der Arbeitszeiten
der einzelnen Mitarbeitern. Es müssen pro Tag abzüglich der gesetzlich 
vorgeschriebenen Pausen im Schnitt 7.9 Stunden rapportiert werden. Die Teams
innerhalb des Unternehmens funktionieren als Profitcenters\footnote{Als Profitcenter
wird ein organisatorischer Teil eines Unternehmens, für den separat dessen Erfolg
gemessen wird, bezeichnet.} und der Teamleiter ist für die Kontrolle und Führung 
seines Teams zuständig.

Die Kommunikation mit dem Kunden erfolgt im Normalfall über den Teamleiter. Es
kann aber auch vorkommen, dass ein Mitarbeiter direkt bei einem Kunden zusätzliche
Informationen einfordert (\textbf{22}). Der Teamleiter hat die Aufgabe den 
Überblick über das Projekt zu behalten (\textbf{23}). Die Daten aller laufenden
und archivierten Projekte werden auf einem zentralen Server gespeichert und alle
Mitarbeiter greifen direkt auf die selben Daten zu (\textbf{25}). Um dies zu 
ermöglichen musste ab einer gewissen Grösse ein stärkerer Server und ein 
leistungsstarkes Netzwerk installiert werden.

\subsubsection{Verwendete Software}
Das Unternehmen setzt auf die Betriebssysteme MacOS X und Linux. Wobei ca.
95\% der Arbeitsstationen MacOS X installiert haben. Auf einer Handvoll 
Arbeitsstationen ist zudem VirtualBox\footnote{VirtualBox ist eine frei verfügbare
Virtualisierungssoftware von Oracle.} mit einer Windows Installation eingerichtet (\textbf{26}).
Dies überwiegend zu Testzwecken und zur Qualitätskontrolle.

Die FEINHEIT setzt überwiegend auf die Office Suite von Apple. Verwendet wird
jedoch auch OpenOffice. Ganz selten Microsoft Office über VirtualBox (\textbf{27}).
Zur Erstellung von Präsentation wird seit neustem auch Landslide\footnote{Landslide
ist ein OpenSource Lösung um Präsentationen mit der HTML5 Technik zu erstellen.} 
eingesetzt.

Das am häufigsten eingesetzte Hilfsmittel zur Unterstützung des Projektmanagement
ist das Whiteboard\footnote{Whiteboard ist keine Software sondern eine einfache
weisse Tafel auf der mit Kreide oder Stiften geschrieben werden kann.}. Die 
wichtigste Software ist jedoch die Eigenentwicklung Metronom\footnote{Metronom 
ist eine kostenpflichtige und webbasierte Firmensoftware. \url{http://fineware.ch/}}, worüber die ganze
Projekt-, Ressourcen-, Lohn- und Arbeitszeitverwaltung läuft (\textbf{28}). Teilweise wird
noch Trac\footnote{Trac ist eine frei verfügbare und webbasierte 
Projektmanamgement-Software speziell für Softwareprojekte.} eingesetzt, jedoch 
werden diese Projekte vor zu von Metronom abgelöst.

\subsubsection{Stärken und Schwächen}
Die Stärken sieht die FEINHEIT zur Zeit in ihrem ``kreativen Chaos'' und ihrem
Team. Sie arbeiten nach dem Motto ``We can do it!'' und fürchten sich nicht 
vor neuen Herausforderungen (\textbf{29}).

Ihre Chancen sehen sie in der Erschaffung der Möglichkeit sogenannte ``Robin Hood''-Projekte,
also Non-Profit Projekte, durchführen zu können. Zudem möchten sie stets ein
gutes Lernumfeld für alle Mitarbeitenden bleiben (\textbf{30}).

Gleichzeitig sehen Sie die Stärke des Chaos auch als ihre grösste Schwäche und 
zwar überwiegend in der fehlenden Konsolidierung innerhalb des Unternehmens. 
Dem wurde aber seit der Umstrukturierung vor einem halben Jahr stark entgegen
gewirkt und hat sich bereits verbessert (\textbf{31}).

Das grösste Risiko sehen sie in einem weiteren Wachstum und dem möglichen 
Verlust des Familiengefühls innerhalb des Unternehmens (\textbf{32}).

\section{Zwischenfazit}
Der Branchenvergleich mit den anderen Unternehmen mit ähnlicher Grösse und
Dienstleistungen war sehr interessant und aufschlussreich. Bei der Panter IIc
ist vor allem der Ansatz, dass alle Mitarbeiter bei Entscheidungen des 
Unternehmens mitreden dürfen sehr interessant. Sie befinden sich jedoch noch
in einer Unternehmensgrösse, bei der sich viele Herausforderungen, denen die allink 
gegenüber steht, noch nicht ergeben haben. Auch hat allink gewisse Herausforderungen,
wie zum Beispiel die Beschäftigung von Praktikanten und Lehrlingen, schon gemeistert.

Die FEINHEIT GmbH bietet da mehr interessante Ansätze und Lösungen. Vor allem
die Struktur scheint gut zu skalieren und ermöglicht anscheinend ein gesundes Wachstum.
Auch ist die Erschaffung von Stellen wie die Qualitätssicherung und das Sekretariat,
die keinen direkten Profit für das Unternehmen generieren, ein Indiz für ein
gutes Vorgehen und besseres Projektmanagement. Laut eigenen Aussagen machen sie
überwiegend den konsequenten Einsatz des Projektmanagement Tools Metronom dafür
verantwortlich. Dies ist jedoch nur ein Mittel zum Zweck und kann auf
diverse Arten umgesetzt werden.

Bei der Erarbeitung des Lösungsansatzes des neuen Projektmanagements für allink 
im Kapitel \ref{chap:konzept} werden diese Erkenntnisse einfliessen.

  
  \chapter{Anforderungsanalyse}\label{chap:anforderungen}
  In der Anforderungsanalyse werden die Anforderungen an den Lösungsansatz des
neuen Projektmanagements und des Projektablaufes ermittelt. Das Ziel sind messbare
Anforderungen, die in Form einer Anforderungsabdeckung auf ihre Umsetzung
geprüft werden können.

Als Anforderungsquellen dienen die Bedürfnisse der einzelnen Stakeholders, die 
zusammen mit der Geschäftsleitung ermittelt werden. Es wird explizit zwischen
Bedürfnissen und Anforderungen unterschieden, da die Bedürfnisse nicht zwingend
in einer Anforderung münden müssen. Auch müssen Bedürfnisse nicht messbar sein.
Bei der Auflistung der Anforderungen wird jedoch auf die erarbeiteten Bedürfnisse 
verwiesen, die mit dieser Anforderung abgedeckt werden sollen.

Zusätzlich zu den Bedürfnissen der Stakeholders werden zusammen mit der Geschäftsleitung
der allink Kennzahlen definiert, die in Zukunft auf Projektebene und projektübergreifend
gemessen werden sollen. Diese werden ebenfalls von den messbaren Anforderungen
abgedeckt und darauf verwiesen.

\section{Bedürfnisse der Stakeholders}
\newcounter{bcounter}

Bei der Aufnahme der Bedürfnisse der verschiedenen Stakeholders wird darauf geachtet, dass sie so einfach und 
eindeutig wie möglich formuliert werden. Zusätzlich wird jedes Bedürfnis beschrieben
und fortlaufend nummeriert, damit in den Anforderungen wieder darauf verwiesen
werden kann.

\subsection{Kunde}
Die in der Tabelle \ref{tab:beduerfnisse_stakeholder_kunde} aufgelisteten 
Bedürfnisse beziehen sich auf die Sicht des Kunden, wurden aber von der
Geschäftsleitung der allink formuliert.

\clearpage

\begin{longtable}{l>{\raggedright}p{3cm}p{10cm}}
    \toprule \textbf{Nr.} & \textbf{Bedürfnis} & \textbf{Beschreibung} \\
    \midrule \addtocounter{bcounter}{1}B\arabic{bcounter} & Klare Timings & 
        Der Kunde möchte klare Timings haben, die von allink auch eingehalten 
        werden können.\\
    \midrule \addtocounter{bcounter}{1}B\arabic{bcounter} & Gute Beratung & 
        Der Kunde möchte geführt werden. Die Beratung muss dem Kunden das 
        Gefühl vermitteln, dass er genau das bekommt, was er benötigt.
        Dabei können auch kleine Zwischenmenschliche Tipps oder Gefälligkeiten
        sehr förderlich sein.\\
    \midrule \addtocounter{bcounter}{1}B\arabic{bcounter} & Gute Qualität & 
        Qualität ist dem Kunden sehr wichtig. Fehler sollte er nicht selbst 
        entdecken.\\
    \midrule \addtocounter{bcounter}{1}B\arabic{bcounter} & Vertraulichkeit & 
        Je nach Daten die in einem Projekt verwendet werden müssen, muss
        auch eine gewisse Vertraulichkeit an den Tag gelegt werden, so dass
        der Kunde darauf Vertrauen kann, dass die Daten bei der Agentur 
        sicher aufbewahrt und für keine anderen Dinge verwendet werden.\\
    \midrule \addtocounter{bcounter}{1}B\arabic{bcounter} & Prestige der Agentur & 
        Nicht zu vernachlässigen ist auch der Bekanntheitsgrad von allink
        selbst. Viele Kunden arbeiten gerne mit einer in der Szene bekannteren
        Agentur zusammen.\\
    \bottomrule
    \caption[Bedürfnisse an das neue Projektmanagement seitens des Kunden]{Bedürfnisse 
        an das neue Projektmanagement seitens des Kunden\footnotemark}
    \label{tab:beduerfnisse_stakeholder_kunde}
\end{longtable}
\footnotetext{Eigene Darstellung}

\subsection{Geschäftsleitung}
Die in der Tabelle \ref{tab:beduerfnisse_stakeholder_partner} aufgelisteten 
Bedürfnisse beziehen sich auf die Sicht der Geschäftsleitung.

\begin{longtable}{l>{\raggedright}p{3cm}p{10cm}}
    \toprule \textbf{Nr.} & \textbf{Bedürfnis} & \textbf{Beschreibung} \\
    \midrule \addtocounter{bcounter}{1}B\arabic{bcounter} & Überblick Projektverlauf & 
        Die Geschäftsleitung wünscht sich einen aussagekräftigen Überblick 
        über die jeweiligen Projekte und deren Verläufe.\\
    \midrule \addtocounter{bcounter}{1}B\arabic{bcounter} & Finanzieller Überblick & 
        Auch wünscht sich die Geschäftsleitung einen finanziellen Überblick 
        über alle Projekte und die Liquidität der Unternehmung.\\
    \midrule \addtocounter{bcounter}{1}B\arabic{bcounter} & Entlastung von Wiederkehrendem & 
        Bei wiederkehrenden Projekten und Abläufen soll die Geschäftsleitung
        soweit wie möglich entlastet werden, da die wichtigsten Informationen
        schon existieren.\\
    \midrule \addtocounter{bcounter}{1}B\arabic{bcounter} & Verantwortungs- träger & 
        Die Geschäftsleitung möchte mehr Verantwortung für weniger kritische
        Projekte vollständig an Berater oder Projektleiter abgeben können,
        die die Kompetenz haben, eigene Entscheidungen darin zu fällen.\\
    \bottomrule
    \caption[Bedürfnisse an das neue Projektmanagement seitens der Geschäftsleitung]{Bedürfnisse 
        an das neue Projektmanagement seitens der Geschäftsleitung\footnotemark}
    \label{tab:beduerfnisse_stakeholder_partner}
\end{longtable}
\footnotetext{Eigene Darstellung}

\subsection{Mitarbeiter}
Die in der Tabelle \ref{tab:beduerfnisse_stakeholder_mitarbeiter} aufgelisteten 
Bedürfnisse beziehen sich auf die Sicht der Mitarbeiter von allink, wurden aber 
von der Geschäftsleitung formuliert.

\begin{longtable}{l>{\raggedright}p{3cm}p{10cm}}
    \toprule \textbf{Nr.} & \textbf{Bedürfnis} & \textbf{Beschreibung} \\
    \midrule \addtocounter{bcounter}{1}B\arabic{bcounter} & Klare Briefings & 
        Der Mitarbeiter wünscht sich klare Briefings, woraus hervorgeht, was
        das Ziel des Projektes und die Aufgabe des einzelnen Mitarbeiters ist.\\
    \midrule \addtocounter{bcounter}{1}B\arabic{bcounter} & Geregelte Arbeitszeiten  & 
        Ausserordentliche Einsätze zu Randzeiten sollten durch eine bessere
        Planung möglichst vermieden werden. Die Mitarbeiter sind sich zwar
        bewusst, dass dies vorkommen kann und sie sind auch bereit, ihren Teil beizutragen.
        Trotzdem sollte es auf ein Minimum reduziert werden.\\
    \midrule \addtocounter{bcounter}{1}B\arabic{bcounter} & Definierte Ablagestruktur & 
        Zum Wohle aller wünscht sich der Mitarbeiter eine klarere Ablagestruktur
        von den aktuellen und archivierten Projekten. Dies kann viele Abläufe
        vereinfachen und unnötige Kommunikation vermeiden.\\
    \midrule \addtocounter{bcounter}{1}B\arabic{bcounter} & Klare Ansprechpartner & 
        Innerhalb eines Projektes, wie auch im Unternehmen, wünscht sich der
        Mitarbeiter klare Ansprechpartner.\\
    \midrule \addtocounter{bcounter}{1}B\arabic{bcounter} & Minimum an administrativen Arbeiten & 
        Damit sich der Mitarbeiter mit seinen Talenten möglichst gut auf 
        seine Aufgaben konzentrieren kann, sollten administrative Arbeiten
        wie das Rapportieren der Arbeitszeiten auf ein Minimum reduziert
        werden.\\
    \bottomrule
    \caption[Bedürfnisse an das neue Projektmanagement seitens der Mitarbeiter]{Bedürfnisse 
        an das neue Projektmanagement seitens der Mitarbeiter\footnotemark}
    \label{tab:beduerfnisse_stakeholder_mitarbeiter}
\end{longtable}
\footnotetext{Eigene Darstellung}

\subsection{Drittanbieter}
Die in der Tabelle \ref{tab:beduerfnisse_stakeholder_drittanbieter} aufgelisteten 
Bedürfnisse beziehen sich auf die Sicht eines Drittanbieters, wie zum Beispiel
einer Druckerei. Sie wurden ebenfalls von der Geschäftsleitung der allink 
formuliert.

\begin{longtable}{l>{\raggedright}p{3cm}p{10cm}}
    \toprule \textbf{Nr.} & \textbf{Bedürfnis} & \textbf{Beschreibung} \\
    \midrule \addtocounter{bcounter}{1}B\arabic{bcounter} & Klare Aufträge & 
        Die Aufträge an den Drittanbieter sollten sauber definiert sein und
        alle nötigen Unterlagen zur Ausführung beinhalten.\\
    \midrule \addtocounter{bcounter}{1}B\arabic{bcounter} & Klare Ansprechpartner & 
        Wie auch der Mitarbeiter wünscht sich ein Drittanbieter für einen 
        Auftrag einen klaren Verantwortlichen innerhalb von allink.\\
    \midrule \addtocounter{bcounter}{1}B\arabic{bcounter} & Rechnungen bezahlen & 
        Die Rechnungen sollten rechtzeitig bezahlt werden. Wo möglich sollte
        man auch vorhandene Skonto\footnote{Als Skonto bezeichnet man einen 
        Preisnachlass auf einen Rechnungsbetrag bei einer frühen Bezahlung.}
        Optionen nutzen, wovon beide Parteien profitieren.\\
    \bottomrule
    \caption[Bedürfnisse an das neue Projektmanagement seitens der Drittanbieter]{Bedürfnisse 
        an das neue Projektmanagement seitens der Drittanbieter\footnotemark}
    \label{tab:beduerfnisse_stakeholder_drittanbieter}
\end{longtable}
\footnotetext{Eigene Darstellung}

\clearpage

\section{Kennzahlen}
\newcounter{kcounter}
Die nachfolgend definierten Kennzahlen sind in Zusammenarbeit mit der Geschäftsleitung
entstanden. Die Benennung der einzelnen Kennzahlen entspringen keinem Lehrbuch,
sondern wurden mit Hinblick auf die Bedürfnisse der allink formuliert. Gewisse
Begriffe sind der Geschäftsleitung bereits gängig und wurden deshalb der Einfachheitshalber 
so belassen. Durch die jeweilige Beschreibung sollte es dem Leser jedoch klar
werden, wie die Kennzahl zustande kommt.

\subsection{Projekt}
Die in der Tabelle \ref{tab:proj_kennzahlen_anforderungen_projektmanagement} abgebildeten
Projekt-Kennzahlen basieren einerseits auf den Anforderungen und auf zusätzlichen
Diskussionen mit der Geschäftsleitung. Die Kennzahlen sollen in Zukunft auf
Projektebene gemessen werden können.

\begin{longtable}{lp{2cm}p{3cm}p{8cm}}
    \toprule \textbf{Nr.} & \textbf{Stufe} & \textbf{Kennzahl} & \textbf{Beschreibung} \\
    \midrule \addtocounter{kcounter}{1}K\arabic{kcounter} & Projekt & Bruttokosten &
        Die Bruttokosten eines Projektes setzen sich aus allen dem Kunden 
        gegenüber offerierten Beträgen zusammen.\\
    \midrule \addtocounter{kcounter}{1}K\arabic{kcounter} & Projekt & Nettokosten &
        Die Nettokosten eines Projektes werden berechnet, indem man die Bruttokosten
        abzüglich den externen Kosten von Drittanbietern rechnet.\\
    \midrule \addtocounter{kcounter}{1}K\arabic{kcounter} & Projekt & Zielstundensatz &
        Der Zielstundensatz ist der durchschnittlich zu erreichende Stundensatz,
        den man in diesem Projekt erreichen will.\\
    \midrule \addtocounter{kcounter}{1}K\arabic{kcounter} & Projekt & Stundenmaximum &
        Das Stundenmaximum eines Projektes wird berechnet, indem man das Nettobudget
        durch den Zielstundensatz teilt.\\
    \midrule \addtocounter{kcounter}{1}K\arabic{kcounter} & Projekt & Total Stunden &
        Die Total Stunden setzten sich aus allen rapportierten Stunden
        von allen Mitarbeitern, die am Projekt mitgearbeitet haben, zusammen.\\
    \midrule \addtocounter{kcounter}{1}K\arabic{kcounter} & Projekt & Eigentlicher Stundensatz &
        Der eigentliche Stundensatz ist der durchschnittlich erreichte Stundensatz im Projekt
        und berechnet sich, indem man das Nettobudget durch die tatsächlich
        rapportierten Stunden teilt.\\
    \bottomrule
    \caption[Projekt-Kennzahlen-Anforderungen an das neue Projektmanagement]{Projekt-Kennzahlen-Anforderungen 
        an das neue Projektmanagement\footnotemark}
    \label{tab:proj_kennzahlen_anforderungen_projektmanagement}
\end{longtable}
\footnotetext{Eigene Darstellung}

Ein Rechenbeispiel der in der Tabelle \ref{tab:proj_kennzahlen_anforderungen_projektmanagement}
definierten Projekt-Kennzahlen ist in der nachfolgenden Tabelle \ref{tab:proj_kennzahlen_anforderungen_projektmanagement_bsp}
abgebildet.

\begin{longtable}{lp{3cm}p{3cm}p{7cm}}
    \toprule \textbf{Nr.} & \textbf{Kennzahl} & \textbf{Betrag} & \textbf{Beschreibung} \\
    \midrule K1 & Bruttokosten & 20'000 CHF & 
        Gegenüber dem Kunden offerierter Betrag. \\
    \midrule K2 & Nettokosten & 16'400 CHF &
        Bruttokosten abzüglich geplanter Kosten von Drittanbietern, zum
        Beispiel einer Druckerei.\\
    \midrule K3 & Zielstundensatz & 150 CHF & 
        Am Ende des Projektes ist das Ziel dem Kunden 150 CHF pro rapportierte
        Stunde verrechnet zu haben. \\
    \midrule K4 & Stundenmaximum & 109 Stunden & 
        Das bedeutet es stehen theoretisch 109 Stunden zur Verfügung. \\
    \midrule K5 & Total Stunden & 135 Stunden & 
        Rapportiert wurden schlussendlich insgesamt 135 Stunden. \\
    \midrule K6 & Eigentlicher Stun- densatz & 121 CHF & 
        Das bedeutet man hat am Ende die Stunden für 121 CHF pro Stunde
        verrechnen können. \\
    \bottomrule
    \caption[Rechenbeispiel der Projekt-Kennzahlen]{Rechenbeispiel der 
        Projekt-Kennzahlen\footnotemark}
    \label{tab:proj_kennzahlen_anforderungen_projektmanagement_bsp}
\end{longtable}
\footnotetext{Eigene Darstellung}

In diesem praxisnahen Beispiel sieht man, dass der Zielstundensatz nicht erreicht
wurde. Dank diesen Kennzahlen kann man dies nun besser erkennen und nach den 
dafür verantwortlichen Gründen suchen. Auch kann man bei einer laufenden 
Betrachtung während eines Projektes besser abschätzen, ob man die Planung
einhalten kann oder nicht.

\subsection{Liquiditätsplanung}
Die in der Tabelle \ref{tab:liq_kennzahlen_anforderungen_projektmanagement} abgebildeten
Liquiditäts-Kennzahlen basieren ebenfalls auf den Anforderungen und auf zusätzlichen
Diskussionen mit der Geschäftsleitung. Diese Kennzahlen sollen in Zukunft
projektübergreifend gemessen und zur Liquiditätsplanung des Unternehmens
verwendet werden können.

\begin{longtable}{lp{2cm}p{3cm}p{8cm}}
    \toprule \textbf{Nr.} & \textbf{Stufe} & \textbf{Kennzahl} & \textbf{Beschreibung} \\
    \midrule \addtocounter{kcounter}{1}K\arabic{kcounter} & Kalender- monat & Bruttokosten &
        Die Bruttokosten pro Monat lassen sich durch die Summe aller Bruttokosten
        aller Projekte berechnen.\\
    \midrule \addtocounter{kcounter}{1}K\arabic{kcounter} & Kalender- monat & Nettokosten &
        Die Nettokosten pro Monat lassen sich durch die Summe aller Nettokosten
        aller Projekte berechnen.\\
    \midrule \addtocounter{kcounter}{1}K\arabic{kcounter} & Kalender- monat & Fremdkosten &
        Die Fremdkosten pro Monat werden durch die Summe der Differenz der Brutton-
        und Nettokosten aller Projekte berechnet.\\
    \midrule \addtocounter{kcounter}{1}K\arabic{kcounter} & Kalender- monat & Offen &
        Die noch offenen Beträge pro Monat werden durch die Summe in diesem
        Monat geplanter, aber noch nicht in Rechnung gestellter Kosten berechnet.\\
    \midrule \addtocounter{kcounter}{1}K\arabic{kcounter} & Planung & Nächste 30 Tage &
        Die noch offenen Beträge in den nächsten 30 Tagen werden durch die Summe,
        der in den nächsten 30 Tagen geplanten und noch nicht in Rechnung
        gestellten Kosten berechnet.\\
    \midrule \addtocounter{kcounter}{1}K\arabic{kcounter} & Planung & Weiter 30 Tage &
        Die noch offenen Beträge weiter weg als 30 Tage werden durch die Summe
        der in den Tagen weiter weg als 30 Tagen geplanten und noch nicht in
        Rechnung gestellten Kosten berechnet.\\
    \midrule \addtocounter{kcounter}{1}K\arabic{kcounter} & Planung & Auf sicher ungeplant &
        Die auf sicher geplanten Beträge lassen sich durch die Summe aller
        noch nicht geplanten und noch nicht in Rechnung gestellten Kosten
        berechnen.\\
    \midrule \addtocounter{kcounter}{1}K\arabic{kcounter} & Planung & Offeriert &
        Die offerierten Beträge werden durch die Summe aller offerierten 
        Beträge über alle noch nicht gestarteten Projekte berechnet.\\
    \bottomrule
    \caption[Liquiditäts-Kennzahlen-Anforderungen an das neue Projektmanagement]{Liquiditäts-Kennzahlen-Anforderungen 
        an das neue Projektmanagement\footnotemark}
    \label{tab:liq_kennzahlen_anforderungen_projektmanagement}
\end{longtable}
\footnotetext{Eigene Darstellung}

Mit diesen Liquiditäts-Kennzahlen sollte es der Geschäftsführung möglich sein,
eine saubere und relativ genaue Liquiditätsplanung auf zwei Monate hinaus
erstellen zu können. Voraussetzung für die Genauigkeit der Daten ist eine
saubere Erfassung und Pflege in einem passendem System.

\clearpage

\section{Anforderungen}\label{chap:sec_anforderungen}
\newcounter{acounter}
Die aufgenommenen und beschriebenen Bedürfnisse und Kennzahlen werden nun in der Formulierung
der Anforderungen berücksichtigt und in der Tabelle \ref{tab:anforderungen_projektmanagement} 
dargestellt. Es liegt in der Natur der Sache, dass Bedürfnisse von mehreren 
Anforderungen abgedeckt und nicht alle Bedürfnisse eins zu eins berücksichtigt 
werden können.

Jede Anforderung wird beschrieben und die darin abgedeckten Bedürfnisse aufgelistet. 
Die Anforderungen werden zudem priorisiert, indem sie in Muss- und Kann-Anforderungen 
kategorisiert werden.\footnote{\citealp*[Vgl.][S. 32]{hobel2006gabler}}

\begin{longtable}{llp{6cm}p{1cm}p{1cm}l}
    \toprule \textbf{Nr.} & \textbf{Anforderung} & \textbf{Beschreibung} & \textbf{Bed.} & \textbf{Ken.} & \textbf{Prio.} \\
    \midrule \addtocounter{acounter}{1}A\arabic{acounter} & Projektorganisation &
        Für jedes Projekt muss ein Hauptverantwortlicher Projektleiter definiert
        werden. Falls dies nicht ein Partner der Geschäftsleitung ist, muss
        zusätzlich noch ein Verantwortlicher Partner definiert werden. & 
        B2, B8, B9, B13, B16 & 
        - & 
        Muss \\
    \midrule \addtocounter{acounter}{1}A\arabic{acounter} & Projektbrief &
        Zu jedem Projekt muss ein klar formulierter Projektbrief erstellt
        werden, der die Ziele des Projektes klar aufzeigt. & 
        B10 & 
        - & 
        Muss \\
    \midrule \addtocounter{acounter}{1}A\arabic{acounter} & Projektplanung &
        Es muss eine zentrale Verwaltung der Projekte existieren, wo Meilensteine
        und deren Arbeitspakete (Todos) verwaltet werden können. & 
        B1, B10 & 
        - & 
        Muss \\
    \midrule \addtocounter{acounter}{1}A\arabic{acounter} & Zeiterfassung &
        Es muss eine möglichst einfache Zeiterfassungs-Lösung
        existieren, wo Mitarbeiter die aufgewendeten Stunden auf Stufe 
        Projekt rapportieren können. & 
        B6, B14 & 
        K5 & 
        Muss \\
    \midrule \addtocounter{acounter}{1}A\arabic{acounter} & Projektcontrolling &
        Es muss eine zentrale Projektverwaltung existieren, wo die wichtigsten
        Kennzahlen eines Projektes betrachtet werden können. & 
        B6, B7 & 
        K1 bis K6 & 
        Muss \\
    \midrule \addtocounter{acounter}{1}A\arabic{acounter} & Kostenüberblick &
        Es muss eine zentrale und Projektübergreifende Kostenverwaltung existieren, wo
        Offerten, externe Kosten und Rechnungen geplant und gepflegt werden können. & 
        B6, B7, B17 & 
        K1 bis K14 & 
        Muss \\
    \midrule \addtocounter{acounter}{1}A\arabic{acounter} & Projektablage &
        Es muss eine klare und möglichst selbsterklärende Struktur für die
        Ablage von laufenden und archivierten Projekten definiert werden. & 
        B8, B12 & 
        - & 
        Muss \\
    \midrule \addtocounter{acounter}{1}A\arabic{acounter} & Qualitätskontrolle &
        Die Qualitätskontrolle muss im Projektablauf klar verankert sein und ein
        Verantwortlicher muss pro Projekt definiert werden. 
        & 
        B3, B4 & 
        - & 
        Muss \\
    \bottomrule
    \caption[Anforderungen an das neue Projektmanagement]{Anforderungen an das 
        neue Projektmanagement\footnotemark}
    \label{tab:anforderungen_projektmanagement}
\end{longtable}
\footnotetext{Eigene Darstellung}

Die Bedürfnisse B5, B11 und B15 konnten nicht direkt den erarbeiteten Anforderungen
zugeordnet werden. Sie können jedoch durch das Erfüllen aller anderen Anforderungen
besser befriedigt werden. Die gewünschten Kennzahlen konnten alle abgedeckt werden.
Zu diesem Zeitpunkt wurden alle Anforderungen als Muss-Anforderungen priorisiert.

\clearpage

\section{Akzeptanzkriterien}\label{chap:akzeptanzkriterien}
\newcounter{akcounter}
Die in der nachfolgenden Tablle \ref{tab:akzeptanzkriterien} definierten Akzeptanzkriterien 
dienen zur Kontrolle der im Kapitel \ref{chap:sec_anforderungen} definierten Anforderungen.
Nach der Erarbeitung des Lösungsansatzes kann anhand dieser Kriterien geprüft
werden, ob die Anforderungen umgesetzt wurden. Bei allen Kriterien wird angegeben
welche Anforderung damit geprüft wird.

\begin{longtable}{lp{12cm}p{1cm}}
    \toprule \textbf{Nr.} & \textbf{Kriterium} & \textbf{Anf.} \\
    \midrule \addtocounter{akcounter}{1}AK\arabic{akcounter} &
        Wurde für das Projekt ein Hauptverantwortlicher Projektleiter definiert? &
        A1 \\
    \midrule \addtocounter{akcounter}{1}AK\arabic{akcounter} &
        Wurde für das Projekt ein Hauptverantwortlicher Partner definiert? &
        A1 \\
    \midrule \addtocounter{akcounter}{1}AK\arabic{akcounter} &
        Existiert für das Projekt ein Projektbrief? &
        A2 \\
    \midrule \addtocounter{akcounter}{1}AK\arabic{akcounter} &
        Können für das Projekt Meilensteine und Arbeitspakete definiert werden? &
        A3 \\
    \midrule \addtocounter{akcounter}{1}AK\arabic{akcounter} &
        Können die Mitarbeiter auf das Projekt Stunden rapportieren? &
        A4 \\
    \midrule \addtocounter{akcounter}{1}AK\arabic{akcounter} &
        Sind die definierten Projektkennzahlen in einem System ersichtlich? &
        A5 \\
    \midrule \addtocounter{akcounter}{1}AK\arabic{akcounter} &
        Sind die definierten Liquiditäts-Kennzahlen in einem System ersichtlich? &
        A6 \\
    \midrule \addtocounter{akcounter}{1}AK\arabic{akcounter} &
        Existiert eine klare Struktur zur Ablage der Projektdaten? &
        A7 \\
    \midrule \addtocounter{akcounter}{1}AK\arabic{akcounter} &
        Wurde für das Projekt ein Hauptverantwortlicher für die Qualitätssicherung definiert? &
        A8 \\
    \bottomrule
    \caption[Akzeptanzkriterien der definierten Anforderungen]{Akzeptanzkriterien 
        der definierten Anforderungen\footnotemark}
    \label{tab:akzeptanzkriterien}
\end{longtable}
\footnotetext{Eigene Darstellung}

Nun muss das neue Projektmanagement und der neue Projektablauf im Kapitel \ref{chap:loesungsansatz}
so erarbeitet werden, dass alle Anforderungen abgedeckt und anhand der Akzeptanzkriterien
geprüft werden können. 
  
  \chapter{Lösungsansatz}\label{chap:konzept}
  
  \chapter{Proof of Concept}\label{chap:proof_of_concept}
  
  \chapter{Reflektion}\label{chap:reflektion}
  \section{Fazit}
  \section{Ausblick}
  
  \appendix
  
  \chapter{Rahmenbedingungen}
  Für das Informatik Diplomstudium an der Fachhochschule Zürich für Technik
HSZ-T wird von den Studenten verlangt eine Diplomarbeit eigenständig zu
verfassen.

\section{Sprache}
Die Semesterarbeit wurde in deutscher Sprache verfasst. Englische Ausdrücke 
wurden immer dort verwendet, wo diese im Sprachgebrauch in den verwendeten 
Programmen genau so gebraucht werden.

Aus Gründen der besseren Lesbarkeit der Diplomarbeit wurde teilweise auf 
die Nennung beider Geschlechter verzichtet. In diesen Fällen ist die 
weibliche Form ausdrücklich inbegriffen.
  
\section{Richtlinien}
Folgende Dokumente mit Richtlinien der Hochschule für Technik Zürich 
wurden für die Diplomarbeit berücksichtigt:

\begin{itemize}
    \item Reglement \footnote{\citealp*[Vgl.][ganzes Dokument]{hsz_reglement}}
    \item Ablauf \footnote{\citealp*[Vgl.][ganzes Dokument]{hsz_ablauf}}
    \item Bewertungskriterien \footnote{\citealp*[Vgl.][ganzes Dokument]{hsz_bewertungskriterien}}
\end{itemize}

  
  \chapter{Aufgabenstellung}\label{chap:aufgabenstellung}
  \section{Ausgangslage}
Die Agentur allink.creative ist im letzten Jahr stark gewachsen. Von zehn
Mitarbeitern im Februar 2010 auf siebzehn Mitarbeiter im Februar 2011. Dies hat 
zur Auswirkung, dass gewisse Funktionen und Prozesse neu definiert und bestehende
überarbeitet werden müssen, um weiterhin effizient, oder wenn möglich noch 
effizienter, arbeiten zu können. Die Agentur arbeitet überwiegend mit Apple
Computern und setzt gewisse Software ein, die die Geschäftsleitung beibehalten 
möchte. Die konkreten Vorstellungen und Vorgaben müssen von dem Studierenden
in der Arbeit erfasst werden.

\section{Ziel der Arbeit}
Bereiche wie die Stundenrapportierung, die Projektplanung und das Projektcontrolling 
können mit Hilfe von IT-Lösungen massgebend optimiert und vereinfacht werden. 
In dieser Arbeit sollen die Herausforderungen, die der Auftraggeber in der 
Planung und im Controlling eines Projektes zu bewältigen hat, erfasst und 
Lösungsvorschläge evaluiert werden. Der Fokus liegt dabei auf der besseren 
Messbarkeit des finanziellen Erfolges eines Projektes und des gesamten 
Unternehmens.

Die Arbeit grenzt sich ganz klar von der Finanzbuchhaltung ab, da
diese keinen direkten Einfluss auf die Projekte hat und bei allink.creative 
bei einen Treuhänder ausgelagert wurde.

\section{Aufgabenstellung}
Folgende Aufgaben soll der Studierende während dieser Arbeit bewältigen:

\begin{itemize}
    \item Ist-Situation im Bereich Projektablauf der allink.creative erfassen
    \item Anforderungen an den neuen Prozess festlegen. Unteranderem Kennzahlen 
        definieren, die in Zukunft auf Projektebene gemessen werden sollen
    \item Eine Recherche der Prozesse in ähnlich funktionierenden KMUs durchführen
    \item Neue Prozesse definieren und bestehende überdenken
    \item Evaluation von IT-Lösungen, die diese Prozesse möglichst passend 
        für den Auftraggeber abbilden und die definierten Kennzahlen generieren können
\end{itemize}

\section{Erwartete Resultate}
Der Studierende soll dem Auftraggeber ein Dokument erstellen, das folgende 
Punkte beinhaltet: 

\begin{itemize}
    \item Beschreibung der Ist-Situation im Bereich Projektablauf
    \item Übersicht der bestehenden Software beim Auftraggeber
    \item Anforderungen an den neuen Prozess inkl. Kennzahlen, die auf 
        Projektebene gemessen werden sollen
    \item Konzept des neuen und überarbeiteten Prozesses
    \item Übersicht der bestehenden Software in der neuen Prozesslandschaft
    \item Softwareempfehlungen für die komplette Prozessabbildung
    \item ``Make or Buy''-Entscheid mit dem Auftraggeber
\end{itemize}

Sobald der neue Prozess und die Tools definiert sind und der Auftraggeber 
einen Entscheid gefällt hat, soll anhand
eines realen Projektes ein ``Proof of Concept'' erstellt werden.

\section{Abgrenzung}
Folgende Punkte werden abgegrenzt, da sie den Rahmen der Arbeit Überschreiten 
würden:

\begin{itemize}
    \item Die Analysen beschränken sich auf Recherchen im Internet und Büchern
    \item Umfragen, Erhebungen sowie Feldstudien werden nicht durchgeführt
\end{itemize}


  
  \chapter{Detailanalyse der Aufgabenstellung}
In der Detailanalyse der Aufgabenstellungen werden die zu bewältigenden
Aufgaben und deren erwartete Resultate ausformuliert.

\section{Aufgaben und Resultate}
\subsubsection{Ist-Situation im Bereich Projektablauf der allink.creative erfassen}
Es soll eine Beschreibung des aktuellen Projektablaufes der allink erarbeitet
und möglichst vollständig alle heutigen Prozessschritte dargestellt und die Vor- 
und Nachteile des Projektablaufes aufgezeigt werden. Hinzu kommt die darin verwendete Software und
deren Einsatzzweck.
\\\\
\underline{Erwartete Resultate}

\begin{description}
    \item[R1] Beschreibung der Ist-Situation im Bereich Projektablauf
    \item[R2] Übersicht der bestehenden Software beim Auftraggeber
\end{description}

\subsubsection{Kennzahlen definieren, die in Zukunft auf Projektebene gemessen werden sollen}
Zusammen mit dem Auftraggeber sollen Voraussetzungen, unteranderem Kennzahlen,
definiert werden, die bei der Erarbeitung des neuen Projektablaufes berücksichtig
werden müssen. 
\\\\
\underline{Erwartete Resultate}

\begin{description}
    \item[R3] Anforderungen an den neuen Prozess inkl. Kennzahlen, die auf 
        Projektebene gemessen werden sollen
\end{description}
  
\subsubsection{Eine Recherche der Prozesse in ähnlich funktionierenden KMUs durchführen}
Der Studierende soll in Gesprächen mit anderen KMUs recherchieren, was Agenturen
mit ähnlichen Voraussetzungen und Herausforderungen wie allink für Projektabläufe
und Hilfsmittel verwenden.
\\\\
\underline{Erwartete Resultate}

Die Informationen aus den Recherchen sollen in den Lösungsvorschlag eingearbeitet
und berücksichtigt werden.

\subsubsection{Neue Prozesse definieren und bestehende, sofern sinnvoll, überarbeiten}
Anhand den bis dahin gewonnen Erkenntnissen soll der bestehende Projektablauf
überarbeitet und neu definiert werden. Dazu arbeitet der Studierende eng mit
dem Auftraggeber zusammen, damit sichergestellt ist, dass eine in der Realität
umsetzbare Lösung definiert wird.
\\\\
\underline{Erwartete Resultate}

\begin{description}
    \item[R4] Konzept des neuen und überarbeiteten Prozesses
    \item[R5] Übersicht der bestehenden Software in der neuen Prozesslandschaft
    \item[R6] Softwareempfehlungen für die komplette Prozessabbildung
\end{description}

\subsubsection{Evaluation von IT-Lösungen, die diese Prozesse möglichst passend 
    für den Auftraggeber abbilden und die definierten Kennzahlen generieren können}
Sofern die bestehende Software des Auftraggebers nicht ausreicht um den neuen
Projektablauf im Unternehmen umzusetzen, soll der Studierende alternative 
Lösungen evaluieren. Darin können auch selbst umzusetzende Tools empfohlen 
werden, sofern der Aufwand für dessen Entwicklung gerechtfertigt ist.
\\\\
\underline{Erwartete Resultate}

\begin{description}
    \item[R7] ``Make or Buy''-Entscheid mit dem Auftraggeber
\end{description}

\section{Aufwandschätzung}
Aufgrund den Anforderungen an die Diplomarbeit und aus der Aufgabenstellung
ergeben sich Arbeitspakete, die in eine Planungs- und Umsetzungsphase
der Diplomarbeit unterteilt werden. Auch versuche ich den Aufwand der einzelnen
Arbeitspaketen in Stunden zu schätzen und diese dann in eine realistische 
Projektplanung einfliessen zu lassen.

\subsection{Planungsphase}
Die Arbeitspakete, die vor der Freigabe der Diplomarbeit erbracht werden müssen,
sind in Stunden geschätzt und in der Tabellen \ref{tab:auwand_planungsphase} 
aufgelistet.

\begin{table}[htbp]
\begin{center}
    \begin{tabular}{llc}
        \toprule & \textbf{Arbeitspaket} & \textbf{Aufwand in Stunden} \\
        \midrule \textbf{P1} & Thema evaluieren & 16 \\
        \midrule \textbf{P2} & Aufgabenstellung ausarbeiten & 8 \\
        \midrule \textbf{P3} & Betreuer finden & 8 \\
        \midrule \textbf{P4} & Kick-Off vorbereiten & 12 \\
        \bottomrule & \textbf{Total Stunden} & \textbf{44} \\
        \bottomrule
    \end{tabular}
    \caption{Aufwandschätzung der Arbeitspakete der Planungsphase}
    \label{tab:auwand_planungsphase}
\end{center}
\end{table}

\subsection{Umsetzungsphase}
Die Arbeitspakete, die während der Diplomarbeit erbracht werden müssen, sind
in Stunden geschätzt und in der Tabelle \ref{tab:aufwand_umsetzungsphase}
aufgelistet.

\begin{table}[htbp]
\begin{center}
    \begin{tabular}{llc}
        \toprule & \textbf{Arbeitspaket} & \textbf{Aufwand in Stunden} \\
        \midrule \textbf{P5} & Ist-Situation erfassen & 32 \\
        \midrule \textbf{P6} & Kennzahlen definieren & 20 \\
        \midrule \textbf{P7} & Review vorbereiten & 8 \\
        \midrule \textbf{P8} & Prozesse definieren und überarbeiten & 44 \\
        \midrule \textbf{P9} & Evaluation von IT Lösungen & 28 \\
        \midrule \textbf{P10} & Abschluss Arbeit & 16 \\
        \midrule \textbf{P11} & Druckauftrag & 12 \\
        \midrule \textbf{P12} & Vorbereitung Präsentation & 16 \\
        \bottomrule & \textbf{Total Stunden} & \textbf{176} \\
        \bottomrule
    \end{tabular}
    \caption{Aufwandschätzung der Arbeitspakete der Umsetzungsphase}
    \label{tab:aufwand_umsetzungsphase}
\end{center}
\end{table}

\chapter{Projektadministration}
\section{Projektplanung}
Den provisorischen Projektplan erstelle ich aufgrund den geschätzten Aufwände
und den vorgegebenen Terminen der HSZ-T. Die gewählten Termine der Meilensteine
``Abgabe Diplomarbeit'' und ``Präsentation Diplomarbeit'' sind zu diesem Zeitpunkt
noch nicht offiziell bestätigt.

Das Total der zu bewältigenden und geschätzten Stunden beläuft sich 
auf 220 Stunden. Diese versuche ich möglichst realistisch über den mir noch
zur Verfügung stehenden Zeitraum zu verteilen. Dies habe ich mit Hilfe der 
Projektplanungs-Software Merlin\footnote{Website von Merlin, \url{http://www.projectwizards.net/de/merlin/}} 
erstellt und eine Übersicht ist in der unten stehenden Grafik \ref{pic:projektplan} 
dargestellt.

\begin{figure}[htbp]
\begin{center}
\includegraphics[width=1\textwidth,angle=0]{./bilder/anhang/projektplanung.pdf}
\caption{Projektplan der Diplomarbeit aus Merlin}
\label{pic:projektplan}
\end{center}
\end{figure}

Ich bin mir bewusst, dass es eine eher optimistische Projektplanung ist. Jedoch
steht mir, aus administrativen Gründen der HSZ-T, nur noch dieser Zeitraum für
die Durchführung der Diplomarbeit übrig um mein Studium erfolgreich abschliessen
zu können. Trotz der optimistischen Planung bin ich zuversichtlich, diese
im gegebenen Zeitrahmen umsetzen und abschliessen zu können.

\section{Termine}
In der Tablle \ref{tab:termine_diplomarbeit} sind alle Termine, die sich aus 
dem Projektplan, den bekannten Terminen und Wunschterminen ergeben, in chronologischer
Reihenfolge ihrer Durchführung aufgelistet.

Zusätzlich habe ich sie mit den abhängenden Arbeitspaketen
und Resultaten ergänzt, die zum Zeitpunkt des Termins vorhanden und erfüllt
sein sollten.

\begin{table}[htbp]
\begin{center}
    \begin{tabular}{llll}
        \toprule & & \multicolumn{2}{c}{\textbf{Abhängende}} \\
        \textbf{Datum} & \textbf{Termin} & \textbf{Arbeitspakete} & \textbf{Resultate} \\
        \midrule 28.02.2011 & Aufgabenstellung einreichen & P1, P2, P3 & - \\
        \midrule 11.03.2011 & Kick-Off Meeting & P4 & -\\
        \midrule 13.04.2011 & Review Termin & P5, P6, P7 & R1, R2, R3 \\
        \midrule 17.05.2011 & ``Make or Buy''-Entscheid & P8, P9 & R4, R5, R6 \\
        \midrule 01.06.2011 & Abgabe Diplomarbeit & P10, P11 & R7 \\
        \midrule 15.06.2011 & Präsentation Diplomarbeit & P12 & - \\
        \bottomrule
    \end{tabular}
    \caption{Auflistung der Termine der Diplomarbeit}
    \label{tab:termine_diplomarbeit}
\end{center}
\end{table}

\section{Versionsverwaltung}
Zur besseren Nachvollziehbarkeit meiner Diplomarbeit führe ich seit Beginn an
ein Repository mit git\footnote{Freie Software zur Versionsverwaltung von Dateien,
\url{http://git-scm.com/}} und ein ausführliches Arbeitsprotokoll als Wikipage.
Meinen aktuellen Stand inklusive aller Unterlagen veröffentliche ich auf der 
Plattform github\footnote{Hosting Dienst, \url{https://github.com/}}. Diese 
Informationen sind öffentlich verfügbar und für jedermann unter folgenden 
Adressen zugänglich:

\subsection{git Repository}
\url{https://github.com/sspross/diplomarbeit}

\subsection{Wiki}
\url{https://github.com/sspross/diplomarbeit/wiki}

\subsection{Arbeitsprotokoll}
\url{https://github.com/sspross/diplomarbeit/wiki/arbeitsprotokoll}

\section{Erreichte Ziele}
In der nachfolgenden Tabelle \ref{tab:erreichte_ziele} sind alle Ziele gemäss 
den erwarteten Resultaten der Aufgabenstellung aufgelistet. Alle 
erwarteten Ziele wurden erreicht.

\begin{table}[htbp]
\begin{center}
    \begin{tabular}{p{10cm}lcl}
        \toprule \textbf{Ziel} & \textbf{Resultat} & \textbf{Stand} \\
        \midrule Die Ist-Situation im Bereich Projektablauf der allink.creative
            ist beschrieben und abgebildet. & R1 & erfüllt \\
        \midrule Eine Übersicht über die bestehende Software bei allink.creative
            wurde erstellt. & R2 & erfüllt \\
        \midrule Die Anforderungen an den neuen Prozess inkl. Kennzahlen, die auf 
            Projektebene gemessen werden sollen, wurden definiert. & R3 
            & erfüllt \\
        \midrule Das Konzept des neuen und überarbeiteten Prozesses ist 
            vorhanden. & R4 & erfüllt \\
        \midrule Eine Übersicht über die bestehende Software bei allink.creative
            in der neuen Prozesslandschaft wurde erstellt. & R5 & erfüllt \\
        \midrule Eine Softwareempfehlung für die komplette Prozessabbildung
            wurde erstellt und begründet. & R6 & erfüllt \\
        \midrule Ein ``Make or Buy''-Entscheid wurde mit dem Auftraggeber 
            getroffen und festgehalten. & R7 & erfüllt \\
        \bottomrule
    \end{tabular}
    \caption{Auflistung der erwarteten Resultate mit Stand der Erfüllung}
    \label{tab:erreichte_ziele}
\end{center}
\end{table}
  
  \chapter{Protokolle}
  \section{Kick-Off Protokoll}
  \section{Design Review Protokoll}
  \section{Arbeitsprotokoll}
  
  \listoffigures
  \listoftables
  % \lstlistoflistings
  
  \bibliographystyle{unsrtnat}
  \bibliography{literaturverzeichnis}

\end{document}