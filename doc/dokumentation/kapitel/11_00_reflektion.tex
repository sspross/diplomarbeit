\section{Fazit}
Der konzipierte Lösungsansatz wurde im ``Proof of Concept'' positiv getestet
und scheint zu funktionieren und alle erarbeiteten Anforderungen wurden erreicht. 
Dank der stetig engen Zusammenarbeit mit der Geschäftsleitung wurde eine
Lösung gefunden, die gut auf die Bedürfnisse der allink passt. Mit den neu
gewählten Instrumente bin ich und ist die allink bis zum jetzigen Zeitpunkt ebenfalls
sehr zufrieden und die Geschäftsleitung sieht darin sehr viel Potenzial.
Wenn der neue Projektablauf nun konsequent umgesetzt wird, hat allink die 
Möglichkeit weiter zu wachsen. Auch gefällt der mir die Mischung
zwischen eingekaufter und selbst entwickelter Software. So muss sich allink
nicht den Tools anpassen, sondern kann die Tools in ihre eigene Richtung 
lenken. Wobei der Kern des Projektmanagements Tools weiterhin von einer
anderen Firma gepflegt und weiterentwickelt wird, da allink dies nicht zu den
ihren Kernkompetenzen hinzufügen möchte.

Rückblickend hätte ich die Instrumentenwahl objektiver durchführen müssen.
Ich stützte mich dabei sehr an die bereits bekannten und aus dem
Branchenvergleich kennengelernten Tools. Dies habe ich überwiegend aus dem Grund
gemacht, da wir uns in der Geschäftsleitung schnell von Basecamp und deren
API begeistert haben und natürlich aus Zeitgründen. Trotzdem scheint mir die Wahl 
sehr passend für allink ausgefallen zu sein.
Der Branchenvergleich war einer der interessantesten Teile für mich.
Würde ich die Arbeit ein weiteres Mal schreiben, gäbe es wohl mehr als
zwei interviewte Agenturen. Denn es gibt wohl noch viele weitere kreative 
Lösungsansätze zu sehen.

Alles in allem bin ich mit meiner Arbeit und den Resultaten sehr
zufrieden und hoffe damit unserer Agentur einen Schritt weiter geholfen zu haben.

\section{Ausblick}
Die Tools sind nun schon zwei Wochen im Einsatz und dank der Unterstützung
von Michael Walder befinden sich schon fast alle Projekte im allink.planer.
Für die allink ist klar, dass dieser nun noch weiter ausgebaut werden muss.
In Zukunft sollen Offerten und Rechnungen direkt aus dem Tool generiert werden
können um die Administration weiter zu vereinfachen. Auch wird man die 
Projektkontrolle noch weiter ausbauen, indem die Meilensteine aus Basecamp
periodisch bezogen und gespeichert werden. So kann dann eine, in der Theorie 
empfohlene, Trendanalyse der Projekte erstellt werden. Zudem will man die
Stundensätze pro Mitarbeiter möglichst genau hinterlegen, um die wirkliche
Rentabilität eines Projektes errechnen zu können.

Es gibt also noch viel zu tun und ich werde weiter an den Projekten arbeiten. 
Diese sind alle unter dem Github Account\footnote{Alle Open-Source
Projekte von allink findet man unter \url{http://github.com/allink}.} von allink
einsehbar und können von jedermann benutzt und mitentwickelt werden. So kann
die erarbeitet Lösung dieser Diplomarbeit von allen eingesetzt werden, die
sich dafür interessieren. 

\section{Danksagung}
Ich möchte mich bei allen, die mich bei der Umsetzung dieser Diplomarbeit
unterstützt haben, bedanken. Insbesondere bedanke ich mich bei Syrus Mozafar
und Matthias Kestenholz, die mir ausführlich über die Abläufe in ihrer Agentur
berichtet haben. Und natürlich auch bei Michael Walder, der von Seiten der allink
ebenfalls viel Zeit investiert hat.

Auch möchte ich meiner Familie und meiner Freundin danken, die mir stets Rückhalt 
gegeben hat und insbesondere meiner Mutter, die mich während den anstrengenden Wochen 
Mental und kulinarisch aufgemuntert hat.

Natürlich danke ich auch meinem Betreuer, Beat Seeliger, der mich ebenfalls
sehr unterstützt hat und ich hoffe, dass wir auch weiterhin nach meinem
Studium auf privater und beruflicher Ebene in Kontakt bleiben.

Zum Schluss möchte ich auch noch meinen Studienkollegen Stefan Laubenberger
und Roman Würsch danken, die stets für bereichernden Diskussionen zur 
Verfügung standen. Und natürlich meiner Kollegin Marisa Meyer für das freundliche 
Korrekturlesen meiner Arbeit.

Vielen Dank!
