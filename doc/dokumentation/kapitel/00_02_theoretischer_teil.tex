\section{Projektmanagement}
Im Rahmen des Projektmanagements werden die vielfältigen Aufgaben in einem
Entwicklungsprozess nicht gemäss ihrem funktionalen Inhalt den einzelnen 
Entwicklungsstellen zugeordnet und dort in einer zeitlichen Reihenfolge 
abgearbeitet, sondern ganzheitlich in einem Projekt eingebettet und unter
Berücksichtigung entsprechender Kosten-, Termin- und Qualitätsparameter 
zielorientiert beplant und durchgeführt.\footnote{\citealp*[Vgl.][S. 9]{burghardt2007einfuehrung}}

\subsection{Begriffserklärung}
\begin{quote}
Projektmanagement wird als Überbegriff aller planenden, überwachenden,
koordinierenden und steuernden Massnahmen verstanden, die für die Um- oder
Neugestaltung von Systemen (resp. Problemlösungen) erforderlich sind.\footnote{\citealp*[Vgl.][S. 1.1]{stiftung1998projekt}}
\end{quote}

\subsection{Projektablauf}
Projektmanagement als Methode einer effizienten Projekteinführung umfasst alle
Aktivitäten, die für eine sachgerechte, termingerechte und kostengerechte
Abwicklung von Projekten erforderlich sind. Um dies zu erreichen, muss das
Projektmanagement in vielfältiger Weise auf den Projektablauf ``regelnd''
wirken. Einerseits werden für die Entwicklung Planvorgaben gemacht, auf deren 
Basis steuernde Massnahmen auf den Ablauf einwirken; andererseits müssen an
definierten Stellen des Entwicklungsprozesses projektbewertende Messgrössen zur
Projektbeurteilung ermittelt und ausgewertet werden.\footnote{\citealp*[Vgl.][S. 11]{burghardt2007einfuehrung}}
Ein Projektablauf unterteilt man laut Definition in die in der Abbildung \ref{pic:01_hauptabschnitte}
dargestellten vier Hauptabschnitte.

\begin{figure}[htbp]
\begin{center}
\includegraphics[width=0.85\textwidth,angle=0]{./bilder/theorie/01_hauptabschnitte.pdf}
\caption{Vier Hauptabschnitte eines Projektablaufes}
\label{pic:01_hauptabschnitte}
\end{center}
\end{figure}

\subsubsection{Projektdefinition}
Die Projektdefinition besteht aus der eigentlichen Gründung eines Projektes,
der Definition des Projektziels, die Organisation des Projektes und des Prozesses.

Am Anfang eines Projektes steht der Projektantrag, der alle relevanten Angaben,
wie Aufgabenbeschreibung, Kosten- und Terminziele sowie Verantwortlichkeiten 
aufnimmt. Mit seiner Verabschiedung wandelt sich der Antrag zum offiziellen
Projektauftrag.\footnote{\citealp*[Vgl.][S. 13]{burghardt2007einfuehrung}}
Als nächstes muss ein eindeutiges und vollständiges Projektziel definiert werden.
Dies geschieht meist zusammen mit dem Auftraggeber anhand eines Anforderungskatalogs
bzw. Pflichtenhefts.

Zur fachlichen, organisatorischen und wirtschaftlichen Absicherung des Projektantrags
empfehlen sich eine Problemfeldanalyse und eine Wirtschaftlichkeitsbetrachtung.
Ohne genaue Kenntnis des Problemfeldes des Projektes sowie ohne Ermittlung der zu erwartenden
Wirtschaftlichkeit des zu entwickelnden Produktes sollte kein Projekt begonnen werden.\footnote{\citealp*[Vgl.][S. 13]{burghardt2007einfuehrung}}
Danach müssen die organisatorischen Voraussetzungen für das Projekt geschaffen werden,
indem der Projektleiter ernannt und eine passende Projektorganisation gewählt wird.

\subsubsection{Projektplanung}
In der Projektplanung definiert man die Strukturplanung, eine Aufwandschätzung,
die Arbeits- und Kostenplanung sowie das Risikomanagement.

Die Projektplanung beginnt mit der Strukturplanung. Aufbauend auf dem
Anforderungskatalog wird das Entwicklungsvorhaben technisch, aufgabengemäss
und kaufmännisch strukturiert. Die sich hierbei ergebenden Strukturen stellen
die Grundpfeiler einer zielorientierten Entwicklung dar; auf sie setzen alle
weiteren Planungsschritte auf. Aus dem Projektstrukturplan werden die 
Aufgabenpakete abgeleitet, für die dann eine Aufwandschätzung durchzuführen ist.
Ausser dem eigenen Erfahrungspotential sollten die Erfahrungen aussenstehender 
Experten sowie die Möglichkeiten von Aufwandsschätzungsverfahren genutzt werden.
Aufwandsschätzungsverfahren und Expertenbefragungen sind hierbei sich gegenseitig
befruchtende Vorgehensweisen. Mit den Ergebnissen der Aufwandsschätzung wird
nun für die einzelnen Arbeitspakete bzw. Teilaufgaben eine Arbeitsplanung vorgenommen.
Häufig empfiehlt sich hier zur Aufgaben- und Terminplanung der Einsatz eines
Netzplans, entweder rechnerunterstützt oder manuell. Die Netzplantechnik ist
trotz aller Kritik eines der leistungsfähigsten Projektmanagement-Hilfsmittel,
wenn sie richtig eingesetzt wird.\footnote{\citealp*[Vgl.][S. 14]{burghardt2007einfuehrung}}

\subsubsection{Projektkontrolle}

\subsubsection{Projektabschluss}

\section{Business Reengineering}
Zweihundert Jahre lang folgten die Menschen bei der Gründung und beim Aufbau
von Unternehmen der brillanten Entdeckung von Adam Smith, dass industrielle
Arbeit in ihre einfachsten und grundlegendsten Aufgaben zerlegt werden sollte.
Im postindustriellen Zeitalter, an dessen Schwelle wir uns heute befinden,
wird hinter der Gründung und Gestaltung von Unternehmen der Gedanke stehen,
diese Aufgaben wieder zu kohärenten Unternehmensprozessen zusammenzuführen.\footnote{\citealp*[Vgl.][S. 12]{hammer2003business}}

\subsection{Begriffserklärung}

