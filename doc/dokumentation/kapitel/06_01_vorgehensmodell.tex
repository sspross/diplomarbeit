Zur Erstellung meiner Ist-Analyse verwende ich die ersten drei Phasen des 
Vorgehensmodell von Erwin Grochla\cite[S. 44-74]{grochla1982grundlagen}.
Die weiteren vier Phasen führen über eine Analyse hinaus und kommen
nicht zur Anwendung.

\subsection{Voruntersuchung}
In der Voruntersuchung, auch Pilotstudie genannt, stellt man sich Fragen wie
``Was soll geändert werden?'' und ``Welches sind die Ziele?''. Man verschafft
sich einen Grobüberblick über das eigentliche Problem und die Rahmenbedingungen
möglicher Lösungen. Auch entscheidet man in der Voruntersuchung, ob der Prozess
fortgesetzt oder abgebrochen werden soll. In der Voruntersuchung können die 
meisten Analyse- und Bewertungstechniken verwendet werden.

Nach Absprache mit dem Auftraggeber werde ich mir zur Voruntersuchung mit  
Hilfe von MindMaps einen besseren Überblick über die aktuellen Probleme bei
allink verschaffen. Auf die einzelnen Punkte soll genauer eingegangen
und wenn möglich Fallbeispiele aus der Praxis genannt werden.

Ich verwende die Technik der MindMaps schon seit einigen Jahren und verwende
sie sehr oft in Projekten um mir einen Überblick über Themen zu verschaffen.

\begin{quotation}
Eine Mind-Map beschreibt eine besonders von Tony Buzan geprägte kognitive 
Technik, die z.B. zur Erschliessung und visuellen Darstellung eines Themengebietes, 
zur Planung oder für Mitschriften genutzt werden kann. Hierbei soll das Prinzip 
der Assoziation helfen, Gedanken frei zu entfalten und die Fähigkeiten des Gehirns 
zu nutzen. Die Mind-Map wird nach bestimmten Regeln erstellt und gelesen. Den 
Prozess bzw. das Themengebiet bzw. die Technik bezeichnet man als Mind-Mapping.
\cite{wikipedia_mindmap}
\end{quotation}

\subsection{Ist-Aufnahme}
In der zweiten Phase erfasst man den aktuellen Zustand indem man das zu 
Untersuchende aus möglichst vielen Betrachtungswinkeln analysieren. Zu den
Techniken der Ist-Aufnahme zählen die Selbstaufschreibung, die Befragung und 
die Beobachtung.

Ich werde mit mir selbst eine Selbstaufschreibung durchführen, da ich in meinem 
Arbeitsalltag unseren Problemen genau so ausgesetzt bin. Aber da mir nur begrenzt
Zeit zur Verfügung steht und damit die Mitarbeiter möglichst ungestört arbeiten
können, begrenze ich Befragungen auf die Partner. Die Meinungen und Beobachtungen
aller Partner fliesst dann in die Ist-Aufnahme ein.

\subsection{Ist-Kritik}
In der dritten Phase widmet man sich den erhobenen Informationen aus der zweiten Phase. 
Die genauen Ursachen der Probleme sollen herausgeschält werden, damit nicht nur 
Symptome behandelt werden. 

Um die Probleme auf ihre tatsächlichen Ursachen zurückführen zu können,
werde ich eine Form der Prüfmatrix, eine Problem-Ursachen Matrix, erstellen.

\begin{quotation}
    Die Prüfmatrix ist ein vereinfachtes Verfahren um Mängel und deren Ursachen 
    zu ermitteln. Mängel werden dabei möglichen Ursachenkategorien matrixförmig 
    gegenübergestellt. Im Schnittpunkt von Mangel und Ursachenkategorie werden 
    die tatsächlichen Ursachen gesucht.\cite{schmidt2000methode}
\end{quotation}