In der Anforderungsanalyse werden die Anforderungen an den Lösungsansatz des
neuen Projektmanagements und des Projektablaufes ermittelt. Das Ziel sind messbare
Anforderungen, die in Form einer Anforderungsabdeckung auf ihre Umsetzung
geprüft werden können.

Als Anforderungsquellen dienen die Bedürfnisse der einzelnen Stakeholders, die 
zusammen mit der Geschäftsleitung ermittelt werden. Es wird explizit zwischen
Bedürfnisse und Anforderungen unterschieden, da die Bedürfnisse nicht zwingend
in einer Anforderung münden müssen. Auch müssen Bedürfnisse nicht messbar sein.
Bei der Auflistung der Anforderungen wird jedoch auf die erarbeiteten Bedürfnisse 
verwiesen, die mit dieser Anforderung abgedeckt werden sollen.

Zusätzlich zu den Bedürfnissen der Stakeholder werden zusammen mit der Geschäftsleitung
der allink Kennzahlen definiert, die in Zukunft auf Projektebene und projektübergreifend
Gemessen werden sollen. Diese werden ebenfalls von den messbaren Anforderungen
abgedeckt und darauf verwiesen.

\section{Bedürfnisse der Stakeholder}
\newcounter{bcounter}

Bei der Aufnahme der Bedürfnisse der verschiedenen Stakeholdern wird darauf geachtet, dass sie so einfach und 
eindeutig wie möglich formuliert werden. Zusätzlich wird jedes Bedürfnis beschrieben
und fortlaufend nummeriert, damit in den Anforderungen wieder darauf verwiesen
werden kann.

\subsection{Kunde}
Die in der Tabelle \ref{tab:beduerfnisse_stakeholder_kunde} aufgelisteten 
Bedürfnisse beziehen sich auf die Sicht des Kunden, wurden aber von der
Geschäftsleitung der allink formuliert.

\begin{longtable}{lp{3cm}p{10cm}}
    \toprule \textbf{Nr.} & \textbf{Bedürfnis} & \textbf{Beschreibung} \\
    \midrule \addtocounter{bcounter}{1}B\arabic{bcounter} & Klare Timings & 
        Der Kunde möchte klare Timings haben, die von allink auch eingehalten 
        werden können.\\
    \midrule \addtocounter{bcounter}{1}B\arabic{bcounter} & Gute Beratung & 
        Der Kunde möchte geführt werden. Die Beratung muss dem Kunden das 
        Gefühl vermitteln, dass er genau das bekommt, was er benötigt.
        Dabei können auch kleine Zwischenmenschliche Tipps oder Gefälligkeiten
        sehr förderlich sein.\\
    \midrule \addtocounter{bcounter}{1}B\arabic{bcounter} & Gute Qualität & 
        Qualität ist dem Kunden sehr wichtig. Fehler sollte er nicht selbst 
        entdecken.\\
    \midrule \addtocounter{bcounter}{1}B\arabic{bcounter} & Vertraulichkeit & 
        Je nach Daten die in einem Projekt verwendet werden müssen, muss
        auch eine gewisse Vertraulichkeit an den Tag gelegt werden, so dass
        der Kunden darauf Vertrauen kann, dass die Daten bei der Agentur 
        sicher aufbewahrt und für keine anderen Dinge verwendet werden.\\
    \midrule \addtocounter{bcounter}{1}B\arabic{bcounter} & Prestige der Agentur & 
        Nicht zu vernachlässigen ist auch der Bekanntheitsgrad von allink
        selbst. Viele Kunden arbeiten gerne mit einer in der Szene bekannteren
        Agentur zusammen.\\
    \bottomrule
    \caption[Bedürfnisse an das neue Projektmanagement seitens des Kunden]{Bedürfnisse 
        an das neue Projektmanagement seitens des Kunden\footnotemark}
    \label{tab:beduerfnisse_stakeholder_kunde}
\end{longtable}
\footnotetext{Eigene Darstellung}

\subsection{Geschäftsleitung}
Die in der Tabelle \ref{tab:beduerfnisse_stakeholder_partner} aufgelisteten 
Bedürfnisse beziehen sich auf die Sicht der Geschäftsleitung.

\begin{longtable}{lp{3cm}p{10cm}}
    \toprule \textbf{Nr.} & \textbf{Bedürfnis} & \textbf{Beschreibung} \\
    \midrule \addtocounter{bcounter}{1}B\arabic{bcounter} & Überblick Projektverlauf & 
        Die Geschäftsleitung wünscht sich einen aussagekräftigen Überblick 
        über die jeweiligen Projekte und deren Verläufe.\\
    \midrule \addtocounter{bcounter}{1}B\arabic{bcounter} & Finanzieller Überblick & 
        Auch wünscht sich die Geschäftsleitung einen finanziellen Überblick 
        über alle Projekte und die Liquidität der Unternehmung.\\
    \midrule \addtocounter{bcounter}{1}B\arabic{bcounter} & Entlastung von Wiederkehrendem & 
        Bei wiederkehrenden Projekten und Abläufe soll die Geschäftsleitung
        soweit wie möglich entlastet werden, da die wichtigsten Informationen
        schon existieren.\\
    \midrule \addtocounter{bcounter}{1}B\arabic{bcounter} & Verantwortungs- träger & 
        Die Geschäftsleitung möchte mehr Verantwortung für weniger kritische
        Projekte vollständig an Berater oder Projektleiter abgeben können,
        die die Kompetenz haben, eigene Entscheidungen darin zu fällen.\\
    \bottomrule
    \caption[Bedürfnisse an das neue Projektmanagement seitens der Geschäftsleitung]{Bedürfnisse 
        an das neue Projektmanagement seitens der Geschäftsleitung\footnotemark}
    \label{tab:beduerfnisse_stakeholder_partner}
\end{longtable}
\footnotetext{Eigene Darstellung}

\subsection{Mitarbeiter}
Die in der Tabelle \ref{tab:beduerfnisse_stakeholder_mitarbeiter} aufgelisteten 
Bedürfnisse beziehen sich auf die Sicht des Mitarbeiters von allink, wurden aber 
von der Geschäftsleitung formuliert.

\begin{longtable}{lp{3cm}p{10cm}}
    \toprule \textbf{Nr.} & \textbf{Bedürfnis} & \textbf{Beschreibung} \\
    \midrule \addtocounter{bcounter}{1}B\arabic{bcounter} & Klare Briefings & 
        Der Mitarbeiter wünscht sich klare Briefings, woraus hervorgeht, was
        das Ziel des Projektes und die Aufgabe des einzelnen Mitarbeiters ist.\\
    \midrule \addtocounter{bcounter}{1}B\arabic{bcounter} & Geregelte Arbeitszeiten  & 
        Ausserordentliche Einsätze zu Randzeiten sollten durch eine bessere
        Planung möglichst vermieden werden. Die Mitarbeiter sind sich zwar
        bewusst, dass dies vorkommen kann und auch bereit ihren Teil beizutragen.
        Trotzdem sollte es auf ein Minimum reduziert werden.\\
    \midrule \addtocounter{bcounter}{1}B\arabic{bcounter} & Definierte Ablagestruktur & 
        Zum Wohle aller wünscht sich der Mitarbeiter eine klarere Ablagestruktur
        von den aktuellen und archivierten Projekten. Dies kann viele Abläufe
        vereinfachen und unnötige Kommunikation vermeiden.\\
    \midrule \addtocounter{bcounter}{1}B\arabic{bcounter} & Klare Ansprechpartner & 
        Innerhalb eines Projektes, wie auch im Unternehmen, wünscht sich der
        Mitarbeiter klare Ansprechpartner.\\
    \midrule \addtocounter{bcounter}{1}B\arabic{bcounter} & Minimum an administrativen Arbeiten & 
        Damit sich der Mitarbeiter mit seinen Talenten möglichst gut auf 
        seine Aufgaben konzentrieren kann, sollten administrative Arbeiten
        wie das Rapportieren der Arbeitszeiten auf ein Minimum reduziert
        werden.\\
    \bottomrule
    \caption[Bedürfnisse an das neue Projektmanagement seitens der Mitarbeiter]{Bedürfnisse 
        an das neue Projektmanagement seitens der Mitarbeiter\footnotemark}
    \label{tab:beduerfnisse_stakeholder_mitarbeiter}
\end{longtable}
\footnotetext{Eigene Darstellung}

\subsection{Drittanbieter}
Die in der Tabelle \ref{tab:beduerfnisse_stakeholder_drittanbieter} aufgelisteten 
Bedürfnisse beziehen sich auf die Sicht eines Drittanbieters, wie zum Beispiel
einer Druckerei. Sie wurden ebenfalls von der Geschäftsleitung der allink 
formuliert.

\begin{longtable}{lp{3cm}p{10cm}}
    \toprule \textbf{Nr.} & \textbf{Bedürfnis} & \textbf{Beschreibung} \\
    \midrule \addtocounter{bcounter}{1}B\arabic{bcounter} & Klare Aufträge & 
        Die Aufträge an den Drittanbieter sollten sauber definiert sein und
        alle nötigen Unterlagen zur Ausführung beinhalten.\\
    \midrule \addtocounter{bcounter}{1}B\arabic{bcounter} & Klare Ansprechpartner & 
        Wie auch der Mitarbeiter wünscht sich ein Drittanbieter für einen 
        Auftrag einen klaren Verantwortlichen innerhalb von allink.\\
    \midrule \addtocounter{bcounter}{1}B\arabic{bcounter} & Rechnungen bezahlen & 
        Die Rechnungen sollten rechtzeitig bezahlt werden. Wo möglich sollte
        man auch vorhanden Skonto\footnote{Als Skonto bezeichnet man einen 
        Preisnachlass auf einen Rechnungsbetrag bei einer frühen Bezahlung.}
        Optionen nutzen, wovon beide Parteien profitieren.\\
    \bottomrule
    \caption[Bedürfnisse an das neue Projektmanagement seitens der Drittanbieter]{Bedürfnisse 
        an das neue Projektmanagement seitens der Drittanbieter\footnotemark}
    \label{tab:beduerfnisse_stakeholder_drittanbieter}
\end{longtable}
\footnotetext{Eigene Darstellung}

\clearpage

\section{Kennzahlen}
\newcounter{kcounter}
Die nachfolgend definierten Kennzahlen sind in Zusammenarbeit mit der Geschäftsleitung
entstanden. Die Benennung der einzelnen Kennzahlen entspringen keinem Lehrbuch,
sondern wurden mit Hinblick auf die Bedürfnisse der allink formuliert. Gewisse
Begriffe sind der Geschäftsleitung bereits gängig und wurden deshalb der Einfachheitshalber 
so belassen. Durch die jeweilige Beschreibung sollte es dem Leser jedoch klar
werden, wie die Kennzahl zustande kommt.

\subsection{Projekt}
Die in der Tabelle \ref{tab:proj_kennzahlen_anforderungen_projektmanagement} abgebildeten
Projekt-Kennzahlen basieren einerseits auf den Anforderungen und auf zusätzlichen
Diskussionen mit der Geschäftsleitung. Die Kennzahlen sollen in Zukunft auf
Projektebene gemessen werden können.

\begin{longtable}{lp{2cm}p{3cm}p{8cm}}
    \toprule \textbf{Nr.} & \textbf{Stufe} & \textbf{Kennzahl} & \textbf{Beschreibung} \\
    \midrule \addtocounter{kcounter}{1}K\arabic{kcounter} & Projekt & Bruttokosten &
        Die Bruttokosten eines Projektes setzen sich aus allen dem Kunden 
        gegenüber offerierten Beträgen zusammen.\\
    \midrule \addtocounter{kcounter}{1}K\arabic{kcounter} & Projekt & Nettokosten &
        Das Nettokosten eines Projektes wird berechnet, indem man die Bruttokosten
        abzüglich den externen Kosten von Drittanbietern rechnet.\\
    \midrule \addtocounter{kcounter}{1}K\arabic{kcounter} & Projekt & Zielstundensatz &
        Der Zielstundensatz ist der durchschnittlich zu erreichende Stundensatz,
        den man in diesem Projekt erreichen will.\\
    \midrule \addtocounter{kcounter}{1}K\arabic{kcounter} & Projekt & Stundenmaximum &
        Das Stundenmaximum eines Projektes wird berechnet, indem man das Nettobudget
        durch den Zielstundensatz teilt.\\
    \midrule \addtocounter{kcounter}{1}K\arabic{kcounter} & Projekt & Total Stunden &
        Die Total Stunden setzten sich aus allen rapportierten Stunden
        von allen Mitarbeitern, die am Projekt mitgearbeitet haben, zusammen.\\
    \midrule \addtocounter{kcounter}{1}K\arabic{kcounter} & Projekt & Eigentlicher Stundensatz &
        Der eigentliche Stundensatz ist der durchschnittlich erreichte Stundensatz im Projekt
        und berechnet sich, indem man das Nettobudget durch die tatsächlich
        rapportierten Stunden teilt.\\
    \bottomrule
    \caption[Projekt-Kennzahlen-Anforderungen an das neue Projektmanagement]{Projekt-Kennzahlen-Anforderungen 
        an das neue Projektmanagement\footnotemark}
    \label{tab:proj_kennzahlen_anforderungen_projektmanagement}
\end{longtable}
\footnotetext{Eigene Darstellung}

Ein Rechenbeispiel der in der Tabelle \ref{tab:proj_kennzahlen_anforderungen_projektmanagement}
definierten Projekt-Kennzahlen ist in der nachfolgenden Tabelle \ref{tab:proj_kennzahlen_anforderungen_projektmanagement_bsp}
abgebildet.

\begin{longtable}{lp{3cm}p{3cm}p{7cm}}
    \toprule \textbf{Nr.} & \textbf{Kennzahl} & \textbf{Betrag} & \textbf{Beschreibung} \\
    \midrule K1 & Bruttokosten & 20'000 CHF & 
        Gegenüber dem Kunden offerierter Betrag. \\
    \midrule K2 & Nettokosten & 16'400 CHF &
        Bruttokosten abzüglich geplanter Kosten von Drittanbietern, zum
        Beispiel einer Druckerei.\\
    \midrule K3 & Zielstundensatz & 150 CHF & 
        Am Ende des Projektes ist das Ziel dem Kunden 150 CHF pro rapportierte
        Stunde verrechnet zu haben. \\
    \midrule K4 & Stundenmaximum & 109 Stunden & 
        Das bedeutet es stehen theoretisch 109 Stunden zur Verfügung. \\
    \midrule K5 & Total Stunden & 135 Stunden & 
        Rapportiert wurden schlussendlich insgesamt 135 Stunden. \\
    \midrule K6 & Eigentlicher Stun- densatz & 121 CHF & 
        Das bedeutet man hat am Ende die Stunden für 121 CHF pro Stunde
        verrechnen können. \\
    \bottomrule
    \caption[Rechenbeispiel der Projekt-Kennzahlen]{Rechenbeispiel der 
        Projekt-Kennzahlen\footnotemark}
    \label{tab:proj_kennzahlen_anforderungen_projektmanagement_bsp}
\end{longtable}
\footnotetext{Eigene Darstellung}

In diesem praxisnahen Beispiel sieht man, dass der Zielstundensatz nicht erreicht
wurde. Dank diesen Kennzahlen kann man dies nun besser erkennen und nach den 
dafür verantwortlichen Gründen suchen. Auch kann man bei einer laufenden 
Betrachtung während eines Projektes besser abschätzen, ob man die Planung
einhalten wird, oder nicht.

\subsection{Liquiditätsplanung}
Die in der Tabelle \ref{tab:liq_kennzahlen_anforderungen_projektmanagement} abgebildeten
Liquiditäts-Kennzahlen basieren ebenfalls auf den Anforderungen und auf zusätzlichen
Diskussionen mit der Geschäftsleitung. Diese Kennzahlen sollen in Zukunft
projektübergreifend gemessen und zur Liquiditätsplanung des Unternehmens
verwendet werden können.

\begin{longtable}{lp{2cm}p{3cm}p{8cm}}
    \toprule \textbf{Nr.} & \textbf{Stufe} & \textbf{Kennzahl} & \textbf{Beschreibung} \\
    \midrule \addtocounter{kcounter}{1}K\arabic{kcounter} & Kalender- monat & Bruttokosten &
        Die Bruttokosten pro Monat lassen sich durch die Summe aller Bruttokosten
        aller Projekte berechnen.\\
    \midrule \addtocounter{kcounter}{1}K\arabic{kcounter} & Kalender- monat & Nettokosten &
        Die Nettokosten pro Monat lassen sich durch die Summe aller Nettokosten
        aller Projekte berechnen.\\
    \midrule \addtocounter{kcounter}{1}K\arabic{kcounter} & Kalender- monat & Fremdkosten &
        Die Fremdkosten pro Monat werden durch die Summe der Differenz der Brutton-
        und Nettokosten aller Projekte berechnet.\\
    \midrule \addtocounter{kcounter}{1}K\arabic{kcounter} & Kalender- monat & Offen &
        Die noch offenen Beträge pro Monat werden durch die Summe in diesem
        Monat geplanter, aber noch nicht in Rechnung gestellter Kosten berechnet.\\
    \midrule \addtocounter{kcounter}{1}K\arabic{kcounter} & Planung & Nächste 30 Tage &
        Die noch offenen Beträge in den nächsten 30 Tagen werden durch die Summe,
        der in den nächsten 30 Tagen geplanten und noch nicht in Rechnung
        gestellten Kosten berechnet.\\
    \midrule \addtocounter{kcounter}{1}K\arabic{kcounter} & Planung & Weiter 30 Tage &
        Die noch offenen Beträge weiter weg als 30 Tage werden durch die Summe,
        der in den Tagen weiter weg als 30 Tagen geplanten und noch nicht in
        Rechnung gestellten Kosten berechnet.\\
    \midrule \addtocounter{kcounter}{1}K\arabic{kcounter} & Planung & Auf sicher ungeplant &
        Die auf sicher geplanten Beträge lassen sich durch die Summe aller
        noch nicht geplanten und noch nicht in Rechnung gestellten Kosten
        berechnen.\\
    \midrule \addtocounter{kcounter}{1}K\arabic{kcounter} & Planung & Offeriert &
        Die offerierten Beträge werden durch die Summe aller offerierten 
        Beträge über alle Projekte berechnet.\\
    \bottomrule
    \caption[Liquiditäts-Kennzahlen-Anforderungen an das neue Projektmanagement]{Liquiditäts-Kennzahlen-Anforderungen 
        an das neue Projektmanagement\footnotemark}
    \label{tab:liq_kennzahlen_anforderungen_projektmanagement}
\end{longtable}
\footnotetext{Eigene Darstellung}

Mit diesen Liquiditäts-Kennzahlen sollte es der Geschäftsführung möglich sein
eine saubere und relativ genaue Liquiditätsplanung auf zwei Monate hinaus
erstellen zu können. Voraussetzung für die Genauigkeit der Daten ist eine
saubere Erfassung und Pflege in einem passendem System.

\clearpage

\section{Anforderungen}\label{chap:sec_anforderungen}
\newcounter{acounter}
Die aufgenommenen und beschriebenen Bedürfnisse und Kennzahlen werden nun in der Formulierung
der Anforderungen berücksichtigt und in der Tabelle \ref{tab:anforderungen_projektmanagement} 
dargestellt. Es liegt in der Natur der Sache, dass Bedürfnisse von mehreren 
Anforderungen abgedeckt und nicht alle Bedürfnisse eins zu eins berücksichtigt 
werden können. Sie werden jedoch bei der Ausarbeitung des Lösungsansatzes im Kapitel \ref{chap:konzept}
erneut beachtet und zum Ende kontrolliert, welche Bedürfnisse abgedeckt wurden.

Jede Anforderung wird beschrieben und die darin abgedeckten Bedürfnisse aufgelistet. 
Die Anforderungen werden zudem priorisiert, indem sie in Muss- und Kann-Anforderungen 
kategorisiert werden.\footnote{\citealp*[Vgl.][S. 32]{hobel2006gabler}}

\begin{longtable}{llp{6cm}p{1cm}p{1cm}l}
    \toprule \textbf{Nr.} & \textbf{Anforderung} & \textbf{Beschreibung} & \textbf{Bed.} & \textbf{Ken.} & \textbf{Prio.} \\
    \midrule \addtocounter{acounter}{1}A\arabic{acounter} & Projektorganisation &
        Für jedes Projekt muss ein Hauptverantwortlicher Projektleiter definiert
        werden. Falls dies nicht ein Partner der Geschäftsleitung ist, muss
        zusätzlich noch ein Verantwortlicher Partner definiert werden. & 
        B2, B8, B9, B13, B16 & 
        - & 
        Muss \\
    \midrule \addtocounter{acounter}{1}A\arabic{acounter} & Projektbrief &
        Zu jedem Projekt muss ein klar formulierter Projektbrief erstellt
        werden, der die Ziele des Projektes klar aufzeigt. & 
        B10 & 
        - & 
        Muss \\
    \midrule \addtocounter{acounter}{1}A\arabic{acounter} & Projektplanung &
        Es muss eine zentrale Verwaltung der Projekte existieren, wo Meilensteine
        und deren Arbeitspakete (Todos) verwaltet werden können. & 
        B1, B10 & 
        - & 
        Muss \\
    \midrule \addtocounter{acounter}{1}A\arabic{acounter} & Zeiterfassung &
        Es muss eine möglichst einfache Zeiterfassungs-Lösung
        existieren, wo Mitarbeiter die aufgewendeten Stunden auf Stufe 
        Projekt rapportieren können. & 
        B6, B14 & 
        K5 & 
        Muss \\
    \midrule \addtocounter{acounter}{1}A\arabic{acounter} & Projektcontrolling &
        Es muss eine zentrale Projektverwaltung existieren, wo die wichtigsten
        Kennzahlen eines Projektes betrachtet werden können. & 
        B6, B7 & 
        K1 bis K6 & 
        Muss \\
    \midrule \addtocounter{acounter}{1}A\arabic{acounter} & Kostenüberblick &
        Es muss eine zentrale und Projektübergreifende Kostenverwaltung existieren, wo
        Offerten, externe Kosten und Rechnungen geplant und gepflegt werden können. & 
        B6, B7, B17 & 
        K1 bis K14 & 
        Muss \\
    \midrule \addtocounter{acounter}{1}A\arabic{acounter} & Projektablage &
        Es muss eine klare und möglichst selbsterklärende Struktur für die
        Ablage von laufenden und archivierten Projekten definiert werden. & 
        B8, B12 & 
        - & 
        Muss \\
    \midrule \addtocounter{acounter}{1}A\arabic{acounter} & Qualitätskontrolle &
        Die Qualitätskontrolle muss im Projektablauf klar verankert sein und ein
        Verantwortlicher muss pro Projekt definiert werden. 
        & 
        B3, B4 & 
        - & 
        Muss \\
    \bottomrule
    \caption[Anforderungen an das neue Projektmanagement]{Anforderungen an das 
        neue Projektmanagement\footnotemark}
    \label{tab:anforderungen_projektmanagement}
\end{longtable}
\footnotetext{Eigene Darstellung}

Die Bedürfnisse B5, B11 und B15 konnten nicht direkt den erarbeiteten Anforderungen
zugeordnet werden. Sie können jedoch durch das Erfüllen aller anderen Anforderungen
besser befriedigt werden. Die gewünschten Kennzahlen konnten alle abgedeckt werden.
Zu diesem Zeitpunkt wurden alle Anforderungen als Muss-Anforderungen priorisiert.

\clearpage

\section{Akzeptanzkriterien}
\newcounter{akcounter}
Die in der nachfolgenden Tablle \ref{tab:akzeptanzkriterien} definierten Akzeptanzkriterien 
dienen zur Kontrolle der im Kapitel \ref{chap:sec_anforderungen} definierten Anforderungen.
Nach der Erarbeitung des Lösungsansatzes kann anhand diesen Kriterien geprüft
werden, ob die Anforderungen umgesetzt wurden. Bei allen Kriterien wird angegeben
welche Anforderung damit geprüft wird.

\begin{longtable}{lp{10cm}p{3cm}}
    \toprule \textbf{Nr.} & \textbf{Kriterium} & \textbf{Anforderung} \\
    \midrule \addtocounter{akcounter}{1}AK\arabic{akcounter} &
        Wurde für das Projekt ein Hauptverantwortlicher Projektleiter definiert? &
        A1 \\
    \midrule \addtocounter{akcounter}{1}AK\arabic{akcounter} &
        Wurde für das Projekt ein Hauptverantwortlicher Partner definiert? &
        A1 \\
    \midrule \addtocounter{akcounter}{1}AK\arabic{akcounter} &
        Existiert für das Projekt ein Projektbrief? &
        A2 \\
    \midrule \addtocounter{akcounter}{1}AK\arabic{akcounter} &
        Können für das Projekt Meilensteine und Arbeitspakete definiert werden? &
        A3 \\
    \midrule \addtocounter{akcounter}{1}AK\arabic{akcounter} &
        Können die Mitarbeiter auf das Projekt Stunden rapportieren? &
        A4 \\
    \midrule \addtocounter{akcounter}{1}AK\arabic{akcounter} &
        Sind die definierten Projektkennzahlen in einem System ersichtlich? &
        A5 \\
    \midrule \addtocounter{akcounter}{1}AK\arabic{akcounter} &
        Sind die definierten Liquiditäts-Kennzahlen in einem System ersichtlich? &
        A6 \\
    \midrule \addtocounter{akcounter}{1}AK\arabic{akcounter} &
        Existiert eine klare Struktur zur Ablage der Projektdaten? &
        A7 \\
    \midrule \addtocounter{akcounter}{1}AK\arabic{akcounter} &
        Wurde für das Projekt ein Hauptverantwortlicher für die Qualitätssicherung definiert? &
        A8 \\
    \bottomrule
    \caption[Akzeptanzkriterien der definierten Anforderungen]{Akzeptanzkriterien 
        der definierten Anforderungen\footnotemark}
    \label{tab:akzeptanzkriterien}
\end{longtable}
\footnotetext{Eigene Darstellung}

Nun muss das neue Projektmanagement und der neue Projektablauf im Kapitel \ref{chap:konzept}
so erarbeitet werden, dass alle Anforderungen abgedeckt und anhand den Akzeptanzkriterien
geprüft werden können. 