Das Schreiben einer Diplomarbeit bedeutet nebst dem baldigen Abschluss des Studiums
auch einen enormen Aufwand. Deshalb ist die Wahl des Themas und meine
Motivation in diese Arbeit die nötige Zeit und Qualität zu investieren enorm
wichtig. Aus diesem Grund möchte ich kurz erläutern, was mich zu dieser
Arbeit geführt hat.

Ein eigenes Unternehmen zu gründen und erfolgreich zu führen war schon immer
einer meiner Träume. Nach meiner Lehre im Jahre 2004 machte ich mich deshalb
selbstständig und gründete 2005 die SiSprocom GmbH\footnote{Website der SiSprocom GmbH, \url{http://sisprocom.ch/}}. 
Im ersten Jahr bestanden
die Tätigkeitsfelder überwiegend aus Webdesign und Schulungen. Im Bereich
Webdesign lag der Fokus hauptsächlich auf der Programmierung. Dieser Bereich 
entwickelte sich immer stärker in Richtung 
Applikationsentwicklung und dank Aufträge einer Zürcher Grossbank 
lag der Fokus bald nur noch darauf.

Durch dieses grössere Mandat wuchs die SiSprocom GmbH zwischenzeitlich
auf 3 Mitarbeiter. Dies führte zu massiv mehr Aufwand in der Administration
und schnell wurde klar, dass zwingend Stunden rapportiert und das Schreiben
von Rechnungen vereinfacht werden musste. Wir bedienten uns damals Google 
Docs\footnote{Googles Online Office Suite, \url{http://docs.google.com/}} um die rapportierten Stunden zentral zu 
verwalten und einer selbst geschriebenen Software um vereinfacht Rechnungen 
verwalten und schreiben zu können.
So lehrreich und spannend die Arbeit in dieser Grossbank auch war, so kompliziert
waren auch ihre Abläufe und Prozesse. Zu diesem Zeitpunkt schwor ich mir, dies
in meiner Unternehmung einmal besser zu lösen.

Ende 2009 realisierte ich, dass ich zwar ein eigenes Unternehmen hatte und
selbständig war, jedoch fast ausschliesslich für einen Kunden arbeitete.
Ich kam zum Schluss, dass dies nicht mein eigentliches Ziel meiner Selbständigkeit 
war und begriff
relativ schnell, dass ich mich von dieser Abhängigkeit nur lösen konnte, wenn
ich mich als Person vollständig aus diesem Mandat zurückziehe.
Dies tat ich dann anfangs 2010 auch und mietete ein Büro in den Räumlichkeiten
der allink GmbH\footnote{Website der allink GmbH, \url{http://allink.ch/}}. Deren IT hatte zu diesem Zeitpunkt 
ein paar grössere
Herausforderungen zu bewältigen und ich bot meine Hilfe an. Schnell wurde
daraus eine Partnerschaft und die SiSprocom fusionierte mit der allink, indem zwischen
den Partnern Anteile der jeweiligen Firma ausgetauscht wurden.

Heute im März 2011, knapp ein Jahr danach, ist Die SiSprocom GmbH um 30\% 
gewachsen und arbeitet ausschliesslich für die erwähnte Grossbank. Alle anderen Projekte
haben wir in die allink GmbH übernommen. Die administrativen Aufgaben der
SiSprocom GmbH wurden
meinem Vater abgegeben, der zusammen mit einem Treuhänder das laufenden
Mandat und die drei Mitarbeiter gut handhaben kann. So konnte ich mich ganz
auf die neuen Herausforderungen bei der allink konzentrieren.

Die allink GmbH ist inzwischen um 70\% auf 17 Mitarbeiter gewachsen und muss sich dadurch
neuen Herausforderungen in der Organisation und Projektabläufe stellen.

Da mich die Selbständigkeit auch durch das ganze Studium begleitet hat, möchte 
ich meine Diplomarbeit nutzen um unser Unternehmen zu optimieren.