\section{Ausgangslage}
Die Agentur allink.creative ist im letzten Jahr stark gewachsen. Von zehn
Mitarbeitern im Februar 2010 auf siebzehn Mitarbeiter im Februar 2011. Dies hat 
zur Auswirkung, dass gewisse Abläufe und Prozesse neu definiert und bestehende
überarbeitet werden müssen, um weiterhin effizient, oder wenn möglich noch 
effizienter, arbeiten zu können. Die Agentur arbeitet zurzeit überwiegend mit Apple
Computern und setzt gewisse Software ein, die die Geschäftsleitung beibehalten 
möchte. Es soll jedoch innerhalb dieser Arbeit geprüft werden, welche Software
weiterhin Sinn macht und welche man möglicherweise ersetzen oder neu anschaffen
bzw. sogar selbst entwickeln möchte.

\section{Problemstellung}
Durch den schnellen Wachstum der Agentur stösst sie bei der Abwicklung ihrer
Projekte an Grenzen. Die Partner, die bisher den Überblick über alle
Projekte und deren Abläufe im Auge behalten konnten, sind bei dieser Grösse
nicht mehr in der Lage dies beizubehalten. Deshalb muss mehr Struktur geschaffen und
den Mitarbeitern mehr Verantwortung und Kompetenzen abgegeben werden. Und trotzdem
soll es der Geschäftsleitung mit Hilfe von Controling Tools möglich bleiben,
einen Überblick über die Lage der Agentur zu behalten.

Die Partner erhoffen sich dadurch ein gesundes Wachstum der Agentur ermöglichen
zu können. Zusätzlich wird vermutet, dass durch Optimierungen im Projektablauf
auch Kosten gespart werden können und die ganze Agentur im Allgemeinen 
professionalisiert werden kann.

\section{Zielsetzung}
Allen Partnern bei der allink GmbH ist klar, dass der
heutiger Projektablauf nicht optimal ist. Zu oft sehen sie sich mit gleichen
Problemen konfrontiert, die in anderen Projekt schon einmal angetroffen und gelöst wurden.
Jedes Mal wird versucht daraus zu lernen, ohne etwas konkret festzuhalten oder
wirklich zu verändern. Das liegt meist daran, dass zu viel ansteht und
die internen Verbesserungen hinter die Aufträge und Wünsche der Kunden
gestellt werden.

Ziel dieser Arbeit ist es, den aktuellen Projektablauf der
allink GmbH genauer zu untersuchen und auf dessen Vorteile und Nachteile einzugehen.
Danach wird versucht die eigentlichen Anforderungen der verschiedenen Stakeholdern
des Projektablaufes aufzunehmen und Kennzahlen zu definieren, die in Zukunft
bei einem verbesserten Projektablauf erfüllt und gemessen werden sollen.
Daraus werden Varianten eines neuen Projektablaufes erstellt und versucht 
diese so zu bewerten, dass die Geschäftsleitung der allink GmbH als Auftraggeber 
dieser Arbeit einen
Entscheid fällen kann, welchen Projektablauf man in Zukunft einsetzen und 
verfeinern möchte. Der neue Projektablauf soll abschliessend in einem ``Proof of Concept'', also
anhand eines konkreten Projektes, getestet werden. Natürlich wird ein einziger
``Proof of Concept'' nicht ausreichen um vollständig sicherzustellen, dass der
neue Projektablauf optimal ist. Dies wird sich aber dann im Laufe der Zeit zeigen.

\section{Aufbau der Arbeit}
Die Arbeit ist in 6 Teile gegliedert. Im ersten Teil werden dem Leser die
Grundlagen zu Projektmanagement und Projektprozesse näher gebracht. Neben
den jeweiligen Begriffserklärungen werden wichtige Modelle und Tools aus der
Theorie und Praxis aufgezeigt.

Im zweiten Teil der Arbeit wird der aktuelle Projektablauf bei der allink
GmbH genauer analysiert und aufgezeigt. Mit Hilfe der Prozessdarstellung wird
das aktuelle Vorgehen dargestellt und mit Beispielen untermalt. In einer
Marktanalyse wird eine ähnlich grosse Agentur interviewt und deren Vorgehen
als Vergleich herangezogen.

Im dritten Teil der Arbeit werden die Anforderungen aller Stakeholders die in
den Projektablauf involviert sind aufgenommen und funktionale wie nicht funktionale
Anforderungen definiert. Zusätzlich werden Kennzahlen definiert, die in Zukunft
während und nach den Projekten gemessen werden sollen.

Im vierten Teil der Arbeit werden mehrere Varianten eines neuen Projektablaufes
evaluiert und mithilfe von Nutzwertanalysen bewertet. Auf dieser Grundlage
entscheidet sich der Auftraggeber für einen neuen Projektablauf. Dieser wird in 
im fünften Teil der Arbeit in einem ``Proof of Concept'' getestet.

Im sechsten und letzten Teil wird die Arbeit in Form eines Fazits zusammengefasst.
Zudem wird ein Ausblick auf die zukünftige Verwendung und den Einsatz des neuen
Projektablaufes abgegeben.

Die folgende Abbildung \ref{pic:01_gliederung_arbeit} beschreibt die Gliederung der 
Arbeit in graphischer Form.

\begin{figure}[htbp]
\begin{center}
\includegraphics[width=0.6\textwidth,angle=0]{./bilder/einleitung/01_gliederung_arbeit.pdf}
\caption{Aufbau der Diplomarbeit}
\label{pic:01_gliederung_arbeit}
\end{center}
\end{figure}

% \section{Methodische Vorgehensweise}

\section{Inhaltliche Schwerpunkte}
Die inhaltlichen Schwerpunkte dieser Arbeit liegen in den folgenden Bereichen:

\begin{itemize}
    \item Einarbeitung Theorie Projektmanagement und Projektprozesse
    \item Analyse des aktuellen Projektablaufes der allink GmbH
    \begin{itemize}
        \item Darstellung des Projektablaufes
        \item Aufzeigen der Stärken und Schwächen
        \item Zurzeit verwendete Software und Hilfsmittel
    \end{itemize}
    \item Variantenbildung neuer Projektabläufe
    \begin{itemize}
        \item Erarbeitung von praxisnahen Varianten
        \item Bewertung und Beurteilung der möglichen Varianten
    \end{itemize}
    \item Überprüfung und Test des neu gewählten Projektablaufes
    \item Fazit und Reflektion der Arbeit
\end{itemize}

Die Arbeit soll dem Leser und anderen Agenturen in einer ähnlichen Situation
aufzeigen, was für Herausforderungen bei einem Wachstum entstehen und wie
sie mit Hilfe von strukturierteren Abläufe bewältigt werden können.

\section{Inhaltliche Ein- und Abgrenzung}
Die Arbeit fokussiert sich auf eine Agentur. Es wird vertieft auf deren Probleme
und Herausforderungen eingegangen. Deshalb sind die Schlüsse die darin gezogen
werden nicht grundsätzlich für jede Agentur anwendbar. Es wird jedoch versucht, das
ganze so global wie möglich zu betrachten. Da zum Schluss aber eine für die
Agentur in der Praxis anwendbare neue Lösung gesucht wird, passt diese wohl
kaum in jede Firmenkultur.

Zusätzlich grenzt sich die Arbeit von folgenden Punkten klar ab:

\begin{itemize}
    \item Die Analysen beschränken sich auf Recherchen im Internet und Büchern.
    \item Umfragen, Erhebungen sowie Feldstudien werden nur begrenzt im Rahmen
        von Interviews durchgeführt.
    \item Die definierten zu messenden Kennzahlen können in den Bereich der Betriebswirtschaft
        und Buchhaltung fallen. Es wird aber nicht näher auf deren Theorien eingegangen.
\end{itemize}

