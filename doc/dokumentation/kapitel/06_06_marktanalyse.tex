\subsection{Fragekatalog}
Der in der Tabelle \ref{tab:fragekatalog} dargestellte Fragekatalog wurde für 
das Interview mit einer anderen Unternehmung aufgebaut. Die einzelnen Themenblöcke 
und die Zusammenstellung der Fragen wurden gezielt auf die Struktur der bestehenden 
Analyse der allink zusammengestellt um einen möglichst guten Vergleich darstellen 
zu können.

\newcounter{qcounter}
\begin{center}
    \begin{longtable}{lp{14cm}}
        \toprule \textbf{Nr.} & \textbf{Frage} \\
        \midrule & \textbf{Allgemeine Fragen} \\
        \midrule \addtocounter{qcounter}{1}\arabic{qcounter} & Wer ist mein Interviewpartner? \\
        \midrule \addtocounter{qcounter}{1}\arabic{qcounter} & Was ist die Funktion meines Interviewpartners? \\
        \midrule \addtocounter{qcounter}{1}\arabic{qcounter} & Was sind die Aufgaben meines Interviewpartners? \\
        \midrule & \textbf{Unternehmen} \\
        \midrule \addtocounter{qcounter}{1}\arabic{qcounter} & Wann wurde das Unternehmen gegründet? \\
        \midrule \addtocounter{qcounter}{1}\arabic{qcounter} & Wie viele Partner mit Mitspracherecht existieren? \\
        \midrule \addtocounter{qcounter}{1}\arabic{qcounter} & Wie ist das Organigramm des Unternehmens aufgebaut? \\
        \midrule \addtocounter{qcounter}{1}\arabic{qcounter} & Wie viele Vollzeitangstellte werden beschäftigt? \\
        \midrule \addtocounter{qcounter}{1}\arabic{qcounter} & Sind Praktikanten angestellt? \\
        \midrule \addtocounter{qcounter}{1}\arabic{qcounter} & Bietet das Unternehmen Lehrstellen an? \\
        \midrule & \textbf{Kunden} \\
        \midrule \addtocounter{qcounter}{1}\arabic{qcounter} & Wie viele Kunden sind Kleinstunternehmen? \\
        \midrule \addtocounter{qcounter}{1}\arabic{qcounter} & Wie viele Kunden sind kleine Unternehmen? \\
        \midrule \addtocounter{qcounter}{1}\arabic{qcounter} & Wie viele Kunden sind mittlere Unternehmen? \\
        \midrule \addtocounter{qcounter}{1}\arabic{qcounter} & Wie viele Kunden sind grosse Unternehmen? \\
        \midrule & \textbf{Projektablauf} \\
        \midrule \addtocounter{qcounter}{1}\arabic{qcounter} & Wie viele Projekte werden akquiriert? \\
        \midrule \addtocounter{qcounter}{1}\arabic{qcounter} & Wie viele Projekte entstehen durch direkte Anfragen? \\
        \midrule \addtocounter{qcounter}{1}\arabic{qcounter} & Wie werden die Aufwände eines potenziellen Projektes geschätzt? \\
        \midrule \addtocounter{qcounter}{1}\arabic{qcounter} & Wie wird auf Änderungen während des Projektes reagiert? \\
        \midrule \addtocounter{qcounter}{1}\arabic{qcounter} & Kommt es vor, dass Projekte während der Durchführung abgebrochen werden? \\
        \midrule \addtocounter{qcounter}{1}\arabic{qcounter} & Von wem werden die Arbeitspakete zusammengestellt? \\
        \midrule \addtocounter{qcounter}{1}\arabic{qcounter} & Wie werden die Arbeitspakete verteilt? \\
        \midrule \addtocounter{qcounter}{1}\arabic{qcounter} & Wie werden die Aufwände rapportiert? \\
        \midrule \addtocounter{qcounter}{1}\arabic{qcounter} & Wer kommuniziert direkt mit einem Kunden? \\
        \midrule \addtocounter{qcounter}{1}\arabic{qcounter} & Wie wird das Feedback eines Kunden verarbeitet? \\
        \midrule \addtocounter{qcounter}{1}\arabic{qcounter} & Wie wird mit zusätzlich zu verrechneten Anforderungen verfahren? \\
        \midrule \addtocounter{qcounter}{1}\arabic{qcounter} & Wie werden die Projektdaten archiviert? \\
        \midrule & \textbf{Verwendete Software} \\
        \midrule \addtocounter{qcounter}{1}\arabic{qcounter} & Auf welches Betriebssystem setzt das Unternehmen? \\
        \midrule \addtocounter{qcounter}{1}\arabic{qcounter} & Welche Office Suite setzt das Unternehmen ein? \\
        \midrule \addtocounter{qcounter}{1}\arabic{qcounter} & Was für Projektmanagement-Software wird verwendet? \\
        \midrule & \textbf{Stärken und Schwächen} \\
        \midrule \addtocounter{qcounter}{1}\arabic{qcounter} & Wo liegen die Stärken des Unternehmens? \\
        \midrule \addtocounter{qcounter}{1}\arabic{qcounter} & Wo sieht das Unternehmen ihre Chancen? \\
        \midrule \addtocounter{qcounter}{1}\arabic{qcounter} & Wo liegen die Schwächen des Unternehmens? \\
        \midrule \addtocounter{qcounter}{1}\arabic{qcounter} & Mit was für Risiken sieht sich das Unternehmen konfrontiert? \\
        \bottomrule
        \caption[Fragekatalog zur Marktanalyse]{Fragekatalog zur Marktanalyse\footnotemark}
        \label{tab:fragekatalog}
    \end{longtable}
\end{center}
\footnotetext{Eigene Darstellung}

Der erstellte Fragenkatalog dient als Leitfaden im Interview. Die Antworten
werden entgegengenommen, diskutiert und anschliessend in beschreibender Form
dokumentiert. Die verweisenden Fragen werden in Klammern an der dazugehörigen
Textstelle angegeben. Falls gewissen Fragen nicht beantwortet wurden, wird dies 
erwähnt und begründet.

\subsection{Panter IIc}
Das Unternehmen Panter IIc mit Sitz in Zürich hat sich freundlicherweise dazu
bereiterklärt mit dem Studierenden ein Interview durchzuführen. Der Interview
Partner ist Syrus Mozafar. Er ist Teilhaber der Panter IIc und Mitglied der
Geschäftsleitung (\textbf{1}).

Er ist laut eigenen Angaben auch für das Projektmanagement zuständig (\textbf{2})
und oft selbst Projektleiter. Sein Aufgabenbereich liegt speziell in der 
Projektplanung, Konsolidierung der Ressourcenplanung und das Auszahlen von
Löhnen in der Buchhaltung (\textbf{3}).

\subsubsection{Unternehmen}
Das Unternehmen ist im Jahre 2005 als GmbH gegründet worden. Die Rechtsform
ist bis heute erhalten geblieben (\textbf{4}). Es existieren fünf Teilhaber,
jedoch kann in ihrer Firmenkultur jeder Mitarbeiter bei grösseren Entscheidungen
des Unternehmens mitentscheiden (\textbf{5}).

Es existiert zur Zeit kein richtiges Organigramm, die Struktur ist somit flach
und jeder Mitarbeiter hat gewisse Kompetenzen und Aufgaben (\textbf{6}). 
Insgesamt hat Panter zwölf Mitarbeiter und zusätzliche sechs Mitarbeiter im
Personalverleih. Die Anstellungspensum variiert zwischen 20\% bis 80\% (\textbf{7}).
Zur Zeit werden weder Praktikanten noch Lehrlinge beschäftigt und die Geschäftsleitung
wird dies zu diesem Zeitpunkt nicht ändern, da sie darin eher einen Mehraufwand
als Nutzen sehen (\textbf{8} und \textbf{9}).

\subsubsection{Kunden}
Panter verfügt über einen relativ grossen Kundenstamm. Die Verteilung
der Unternehmensgrössen der Kunden unterteilen sich in ca. 40\% Kleinstunternehmen,
20\% kleine Unternehmen, 10\% mittlere Unternehmen und 30\% grosse
Unternehmen (\textbf{10} bis \textbf{13}).

\subsubsection{Projektablauf}
Bei Panter werden ca. 20\% der Projekte selbst akquiriert (\textbf{14}) und ca.
80\% entstehen durch direkte Anfragen oder Folgeaufträge (\textbf{15}). Kleinere
Projekte mit einem Umsatzvolumen bis 20'000 CHF werden nur grob im Alleingang
des Projektleiters geschätzt. Bei grösseren Projekten werden auf Grund der
Technologiewahl Experten innerhalb von Panter oder ausserhalb herbeigezogen
um eine möglichst genaue Schätzung abgeben zu können (\textbf{16}). Sie verwenden
dazu keine speziellen Verfahrenstechniken und setzen überwiegend die Software
Microsoft Excel ein. Bei grösseren Änderungen während eines Projektes wird
erneut eine Schätzung vorgenommen und dem Kunden die zusätzlichen Aufwände
offeriert (\textbf{17}). Es wird von Fall zu Fall entschieden ob eine Änderung 
zusätzlich verrechnet oder dem Kunden ``geschenkt'' wird (\textbf{24}). Bis zu 
diesem Zeitpunkt kam es erst einmal zu einem vollständigen Abbruch eines Projektes 
während dessen Durchführung (\textbf{18}).

Die Arbeitspakete werden überwiegend während der Erstellung der Schätzung und 
der Offerte vom Projektleiter und den Experten zusammengestellt (\textbf{19}).
Dabei ist im Normalfall immer auch eine Person der Geschäftsleitung vertreten.
Die Arbeitspakete werden dann auch gleich von dieser Gruppe an die für das 
Projekt eingeplanten Ressourcen verteilt (\textbf{20}).

Die Aufwände werden von den Mitarbeitern sehr exakt rapportiert (\textbf{21}),
da die Lohnsummen der einzelnen Mitarbeiter von den Anzahl geleisteten Stunden
abhängt. Es existieren somit keine Fixlöhne bei Panter.

Die Kommunikation mit dem Kunden erfolgt im Normalfall über den Projektleiter.
Es kommt aber auch vor, dass ein Mitarbeiter direkt bei einem Kunden zusätzliche
Informationen einfordert (\textbf{22}). Es existieren keine fixen Regeln dazu.
Das Feedback des Kunden wird im Unternehmens Wiki eingetragen. Da die zerstreute
Verteilung der Projektdaten in der Vergangenheit schon öfters bemängelt wurde, 
baut Panter zu diesem Zeitpunkt eine Struktur für ein Projektarchiv auf. Sie
verwenden zudem ein allgemeines E-Mail Konto um projektspezifische E-Mails
zentral und für alle Mitarbeiter zugänglich abzulegen (\textbf{23} und \textbf{25}).

\subsubsection{Verwendete Software}
Das Unternehmen setzt auf kein spezifisches Betriebssystem. Jeder Mitarbeiter
hat sein eigenes Gerät mit seinem präferierten System installiert. Einzig für
die unternehmenseigenen Server wird das Betriebssystem Debian\footnote{Debian 
ist ein frei verfügbares Betriebssystem, \url{http://www.debian.org/}} 
vorausgesetzt (\textbf{26}).

Also Office Suite setzt Panter auf die Open-Source
Lösung OpenOffice\footnote{OpenOffice ist eine frei verfügbare Office Suite, 
\url{http://de.openoffice.org/}}. Zusätzlich, um Interoperabilitätsprobleme
mit Kunden zu vermeiden, hat Panter eine Arbeitsstation mit Microsoft Windows
und Microsoft Office ausgerüstet (\textbf{27}).

Panter setzt überwiegend auf die Projektmanagement Software RedMine\footnote{RedMine
ist eine Webapplikation für Projektmanagement, \url{http://www.redmine.org/}},
wobei sie nicht vollständig sondern nur Teile davon nutzen (\textbf{28}). 
Das unternehmenseigene Wiki und diverse Exceldokumente unterstützen die in RedMine
verwalteten Projekte. 

\subsubsection{Stärken und Schwächen}

\subsection{FEINHEIT GmbH}

\subsubsection{Unternehmen}

\subsubsection{Kunden}

\subsubsection{Projektablauf}

\subsubsection{Verwendete Software}

\subsubsection{Stärken und Schwächen}
