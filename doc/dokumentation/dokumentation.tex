%  --------------------------------------------------------------------------
%  Diplomarbeit Dokumentation
%  Created by Silvan Spross on 2011-04-02.
%  --------------------------------------------------------------------------

%  --------------------------------------------------------------------------
%  Latex Document Settings
%  --------------------------------------------------------------------------
\documentclass[
11pt, % Schriftgrösse
a4paper, % A4 Papier
BCOR25mm, % Absoluter Wert der Bindekorrektur, z.B. BCOR1cm
DIV14, % Satzspiegel festlegen siehe
       % http://www.ctex.org/documents/packages/nonstd/koma-script.pdf
footsepline = false, % Trennlinie zwischen Textkörper und Fußzeile
                     % bei normalen Seiten
headsepline, % Trennlinie zwischen Kopfzeile und Textkörper
             % bei normalen Seiten
twoside, % Zweiseitig
openright,
%halfparskip, % Europäischer Satz mit Abstand zwischen den Absätzen
abstracton, % inkl. Abstract
listof=totocnumbered, % Abb.- und Tab.verzeichnis im Inhaltsverzeichnis
bibliography=totocnumbered % Lit.zeichnis in Inhaltsverzeichnis aufnehmen
]{scrreprt}

\usepackage[automark]{scrpage2} % Gestaltung von kopf- und Fußzeile
\usepackage[ngerman]{babel}
\usepackage[ngerman]{translator}
\usepackage{tocbasic}
\usepackage[utf8]{inputenc}
\usepackage{lmodern} % Latin Modern
\usepackage[T1]{fontenc}
\usepackage{hyphenat}
\usepackage{ae} % Schöne Schriften für PDF-Dateien

% Tradmark
\def\TTra{\textsuperscript{\texttrademark}}

%1.5 Zeilenabstand
\usepackage[onehalfspacing]{setspace}

% Festlegung des Seitenstils (scrpage2)
\pagestyle{scrheadings}
\clearscrheadfoot
\automark[chapter]{section}

\lehead{\sffamily\upshape\headmark}
\cehead{}
\rehead{}
\lefoot[\pagemark]{\upshape \pagemark}
\cefoot{}
\refoot{}
\lohead{}
\cohead{}
\rohead{\sffamily\upshape\headmark}
\lofoot{}
\cofoot{}
\rofoot[\pagemark]{\scshape \pagemark}

% Surround parts of graphics with box
\usepackage{boxedminipage}

% Package for including code in the document
\usepackage{listings}

% If you want to generate a toc for each chapter (use with book)
\usepackage{minitoc}

% Abkürzungsverzeichnis erstellen.
\usepackage[printonlyused]{acronym}

% schöne Tabelle zeichnen
\usepackage{booktabs}
\renewcommand{\arraystretch}{1.4} %Die Zeilenabstände in Tabllen angepasst.

% für variable Breiten
\usepackage{tabularx}

% Durchgestrichener Text
\usepackage[normalem]{ulem} %emphasize weiterhin kursiv

% This is now the recommended way for checking for PDFLaTeX:
\usepackage{ifpdf}

\usepackage[hyperfootnotes=false]{hyperref}
\hypersetup{
  bookmarks=true,         % show bookmarks bar?
  unicode=true,           % non-Latin characters in Acrobat’s bookmarks
  pdftoolbar=true,        % show Acrobat’s toolbar?
  pdfmenubar=true,        % show Acrobat’s menu?
  pdffitwindow=true,      % window fit to page when opened
  pdfstartview={FitH},    % fits the width of the page to the window
  pdftitle={Diplomarbeit},   
  pdfauthor={Silvan Spross},
  pdfsubject={Definition und Optimierung der Projektprozesse bei allink.creative},
  pdfcreator={TeX Live 2009},
  pdfproducer={pdfTeX, Version 3.1415926-1.40.10},
  pdfnewwindow=true,      % links in new window
  colorlinks=true,       % false: boxed links; true: colored links
  linkcolor=blue,          % color of internal links
  citecolor=green,        % color of links to bibliography
  filecolor=magenta,      % color of file links
  urlcolor=cyan          % color of external links
  % linkcolor=black,          % color of internal links
  % citecolor=black,        % color of links to bibliography
  % filecolor=black,      % color of file links
  % urlcolor=black          % color of external links
}

\ifpdf
    \usepackage[pdftex]{graphicx}
\else
    \usepackage{graphicx}
\fi

\makeatletter 
\let\orgdescriptionlabel\descriptionlabel 
\renewcommand*{\descriptionlabel}[1]{% 
  \let\orglabel\label 
  \let\label\@gobble 
  \phantomsection 
  \edef\@currentlabel{#1}% 
  %\edef\@currentlabelname{#1}% 
  \let\label\orglabel 
  \orgdescriptionlabel{#1}% 
} 
\makeatother 

%  --------------------------------------------------------------------------
%  Start Document
%  --------------------------------------------------------------------------
\title{Definition und Optimierung der Projektprozesse bei allink.creative}

\author{Diplomarbeit in Informatik\\
    \\
    Studierender - Silvan Spross\\
	Auftraggeber - Michael Walder, allink.creative\\
    Projektbetreuer - Beat Seeliger\\
    Experte - tbd\\
	\\
	HSZ-T - Technische Hochschule Zürich}

\date{März 2011 bis Juni 2011}

\begin{document}

  \ifpdf
    \DeclareGraphicsExtensions{.pdf, .jpg, .tif}
  \else
    \DeclareGraphicsExtensions{.eps, .jpg}
  \fi
  
  \pagenumbering{Alph}
  
  \maketitle
  \cleardoublepage

  \begin{abstract}
tbd
\end{abstract}
  \cleardoublepage

  \pagenumbering{roman}
  
  \tableofcontents
  \cleardoublepage
  
  \pagenumbering{arabic}
  
  \chapter{Rahmenbedingungen}
  Für das Informatik Diplomstudium an der Fachhochschule Zürich für Technik
HSZ-T wird von den Studenten verlangt eine Diplomarbeit eigenständig zu
verfassen.

\section{Sprache}
Die Semesterarbeit wurde in deutscher Sprache verfasst. Englische Ausdrücke 
wurden immer dort verwendet, wo diese im Sprachgebrauch in den verwendeten 
Programmen genau so gebraucht werden.

Aus Gründen der besseren Lesbarkeit der Semesterarbeit wurde teilweise auf 
die Nennung beider Geschlechter verzichtet. In diesen Fällen ist die 
weibliche Form ausdrücklich inbegriffen.
  
\section{Richtlinien}
Folgende Dokumente mit Richtlinien der Hochschule für Technik Zürich 
wurden für die Semesterarbeit berücksichtigt:

\begin{itemize}
    \item Reglement \cite{hsz_reglement}
    \item Ablauf \cite{hsz_ablauf}
    \item Bewertungskriterien \cite{hsz_bewertungskriterien}
\end{itemize}

  \cleardoublepage
  
  \chapter{Einführung}
  tbd

\section{Motivation}
tbd

\section{Zielsetzung}
tbd

\section{Danksagung}
tbd
  
  \cleardoublepage
  
  \chapter{Aufgabenstellung}
  %
%  Antrag, Aufgabenstellung Diplomarbet
%
%  Created by Silvan Spross on 2011-02-03.
%
\documentclass[]{scrreprt}
\usepackage[ngerman]{babel}

% Use utf-8 encoding for foreign characters
\usepackage[utf8]{inputenc}

% Setup for fullpage use
\usepackage{fullpage}

% Package for including code in the document
\usepackage{listings}

% If you want to generate a toc for each chapter (use with book)
\usepackage{minitoc}

% This is now the recommended way for checking for PDFLaTeX:
\usepackage{ifpdf}

\ifpdf
    \usepackage[pdftex]{graphicx}
\else
    \usepackage{graphicx}
\fi

\title{Aufgabenstellung\\
    Diplomarbeit in Informatik}
    
\author{Studierender - Silvan Spross\\
    Projektbetreuer - Beat Seeliger\\
    Auftraggeber - Michael Walder, allink.creative\\
    \\
    HSZ-T - Technische Hochschule Zürich}
    
\date{26. Februar 2011}

\begin{document}

    \ifpdf
        \DeclareGraphicsExtensions{.pdf, .jpg, .tif}
    \else
        \DeclareGraphicsExtensions{.eps, .jpg}
    \fi

    \maketitle

    \pagenumbering{arabic}

    % \tableofcontents

    \chapter{Aufgabenstellung Diplomarbeit}

    \section{Thema}
    Definition und Optimierung der Projektprozesse bei allink.creative

    \section{Ausgangslage}
    Die Agentur allink.creative ist im letzten Jahr stark gewachsen. Von zehn
    Mitarbeitern im Februar 2010 auf siebzehn Mitarbeiter im Februar 2011. Dies hat 
    zur Auswirkung, dass gewisse Funktionen und Prozesse neu definiert und bestehende
    überarbeitet werden müssen, um weiterhin effizient, oder wenn möglich noch 
    effizienter, arbeiten zu können. Die Agentur arbeitet überwiegend mit Apple
    Computern und setzt gewisse Software ein, die die Geschäftsleitung beibehalten 
    möchte. Die konkreten Vorstellungen und Vorgaben müssen von dem Studierenden
    in der Arbeit erfasst werden.

    \section{Ziel der Arbeit}
    Bereiche wie die Stundenrapportierung, die Projektplanung und das Projektcontrolling 
    können mit Hilfe von IT-Lösungen massgebend optimiert und vereinfacht werden. 
    In dieser Arbeit sollen die Herausforderungen, die der Auftraggeber in der 
    Planung und im Controlling eines Projektes zu bewältigen hat, erfasst und 
    Lösungsvorschläge evaluiert werden. Der Fokus liegt dabei auf der besseren 
    Messbarkeit des finanziellen Erfolges eines Projektes und des gesamten 
    Unternehmens.
    
    Die Arbeit grenzt sich ganz klar von der Finanzbuchhaltung ab, da
    diese keinen direkten Einfluss auf die Projekte hat und bei allink.creative 
    bei einen Treuhänder ausgelagert wurde.
    
    % Folgende Ziele sollen erreicht werden:
    % 
    % \begin{itemize}
    %     \item Bla
    % \end{itemize}
    % 
    % Folgende Punkte werden abgegrenzt, da es den Rahmen der Arbeit sprengen 
    % würde:
    % 
    % \begin{itemize}
    %     \item Bla
    % \end{itemize}

    \section{Aufgabenstellung}
    Folgende Aufgaben soll der Studierende während dieser Arbeit bewältigen:
    
    \begin{itemize}
        \item IST-Situation im Bereich Projektablauf der allink.creative erfassen
        \item Kennzahlen definieren, die in Zukunft auf Projektebene gemessen 
            werden sollen
        \item Neue Prozesse definieren und bestehende, sofern sinnvoll, überarbeiten
        \item Evaluation von IT-Lösungen, die diese Prozesse möglichst passend 
            für den Auftraggeber abbilden und die definierten Kennzahlen generieren können
    \end{itemize}

    \section{Erwartete Resultate}
    Der Studierende soll dem Auftraggeber ein Dokument erstellen, das folgende 
    Punkte beinhaltet: 
    
    \begin{itemize}
        \item Beschreibung der IST-Situation im Bereich Projektablauf
        \item Übersicht der bestehenden Software beim Auftraggeber
        \item Kennzahlen, die auf Projektebene gemessen werden sollen
        \item Darstellung der neuen und überarbeiteten Prozesse
        \item Übersicht der bestehenden Software in der neuen Prozesslandschaft
        \item Softwareempfehlungen für die komplette Prozessabbildung
        \item ``Make or Buy''-Entscheid mit dem Auftraggeber
    \end{itemize}
    
    Sofern noch zum Zeitpunkt dieser Arbeit möglich, soll der neue 
    Projektablauf anhand eines realen Projektes aufgezeigt werden.

    \section{Geplante Termine}
    Die Termine können zum Zeitpunkt des Antrages noch nicht definitiv 
    festgelegt werden. Sofern jedoch die Planung eingehalten werden kann und 
    freie Termine zur Verfügung stehen, sollten die Termine innerhalb der 
    angegebenen Monate liegen.

    \begin{tabbing}
        \hspace*{4cm}\= \kill
    	Kick-Off:               \> Anfangs März 2011\\
    	Review:                 \> Anfangs April 2011\\
    	Abgabe:                 \> Anfangs Juni 2011\\
    	Schlusspräsentation:    \> Mitte Juni 2011 \\
    \end{tabbing}

    \section{Genehmigung}
    Der Studierende, sein Projektbetreuer und der Studiengangsleiter 
    Informatik erklären sich mit der Aufgabenstellung einverstanden und geben 
    die Arbeit frei zur Erfassung im Einschreibesystem der Hochschule für 
    Technik Zürich.

    \begin{tabbing}
        \hspace*{10cm}\= \kill
    	Silvan Spross, Studierender \> Beat Seeliger, Projektbetreuer \\\\\\
        \line(1,0){150} \> \line(1,0){150} \\\\\\
    	Dr. Olaf Stern, Studiengangsleiter Informatik \\\\\\
        \line(1,0){150}
    \end{tabbing}
    
    \bibliographystyle{plain}
    \bibliography{literaturverzeichnis}
    
\end{document}
  \cleardoublepage
  
  %\chapter{Erreichte Ziele}
  %\input{./kapitel/erreichte_ziele.tex}
  %\cleardoublepage
  
  \appendix
  
  \chapter{Personalienblatt}
  \input{./kapitel/personalienblatt.tex}
  
  \chapter{Bestätigung}
  \input{./kapitel/bestaetigung.tex}
  
  \listoffigures
  \listoftables
  %\lstlistoflistings
  
  \bibliographystyle{alpha}
  \bibliography{literaturverzeichnis}

\end{document}